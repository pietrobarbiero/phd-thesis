%%%%%
%%
%% Sample document ``thesis.tex''
%%
%% Version: v0.2
%% Authors: Jean Martina, Rok Strnisa, Matej Urbas
%% Date: 30/07/2008
%%
%% Copyright (c) 2008-2011, Rok Strniša, Jean Martina, Matej Urbas
%% License: Simplified BSD License
%% License file: ./License
%% Original License URL: http://www.freebsd.org/copyright/freebsd-license.html
%%%%%

% Available documentclass options:
%
%   <all `report` document class options, e.g.: `a5paper`>
%   withindex   - enables the index. New index entries can be added through `\index{my entry}`
%   glossary    - enables the glossary.
%   techreport  - typesets the thesis in the technical report format.
%   firstyr     - formats the document as a first-year report.
%   times       - uses the `Times` font.
%   backrefs    - add back references in the Bibliography section
%
% For more info see `README.md`
% \documentclass[withindex,glossary]{cam-thesis}
\documentclass[withindex,glossary]{cam-thesis}

% Citations using numbers
\usepackage[numbers]{natbib}
\usepackage{bibentry}
\usepackage{url}

\usepackage{algorithm}
\usepackage{algorithmic}
\usepackage{subcaption}
\usepackage{amsmath}
\usepackage{amsthm}
\usepackage{amssymb}
\usepackage{paralist}
\usepackage{dsfont}

\usepackage{amssymb,amsfonts}
\usepackage{booktabs}
\usepackage{xcolor}
\usepackage{multirow}
\usepackage{caption}
\usepackage{subcaption}
\usepackage{longtable}
\usepackage{amsmath}
\usepackage{float}
%\usepackage{fontawesome}
\usepackage{makecell}
\usepackage{listings}
\usepackage{csquotes}


\usepackage{streams}
\usepackage{stringdiagrams}


\usepackage{tabularx}


\newcommand{\mc}{\mathcal}

%%%%%%%%%%%%%%%%%%%%%%%%%%%%%%%%%
% THEOREMS
%%%%%%%%%%%%%%%%%%%%%%%%%%%%%%%%
\theoremstyle{plain}
\newtheorem{theorem}{Theorem}[section]
\newtheorem{proposition}[theorem]{Proposition}
\newtheorem{lemma}[theorem]{Lemma}
\newtheorem{corollary}[theorem]{Corollary}
\theoremstyle{definition}
\newtheorem{definition}[theorem]{Definition}
\newtheorem{assumption}[theorem]{Assumption}
\theoremstyle{remark}
\newtheorem{remark}[theorem]{Remark}
\newtheorem{example}[theorem]{Example}

\bibliographystyle{plainnat}


%%%%%%%%%%%%%%%%%%%%%%%%%%%%%%%%%%%%%%%%%%%%%%%%%%%%%%%%%%%%%%%%%%%%%%%%%%%%%%%%
%% Thesis meta-information
%%

%% The title of the thesis:
% \title{Interactive Concept-Aware Learning\\or\\Beyond the Accuracy-Explainability Trade-Off}
\title{Deep concept reasoning:\\beyond the accuracy-interpretability trade-off}

%% The full name of the author (e.g.: James Smith):
\author{Pietro Barbiero}

%% College affiliation:
\college{Clare College}

%% College shield [optional]:
% \collegeshield{CollegeShields/Christs}
% \collegeshield{CollegeShields/Churchill}
\collegeshield{CollegeShields/Clare}
% \collegeshield{CollegeShields/ClareHall}
% \collegeshield{CollegeShields/CorpusChristi}
% \collegeshield{CollegeShields/Darwin}
% \collegeshield{CollegeShields/Downing}
% \collegeshield{CollegeShields/Emmanuel}
% \collegeshield{CollegeShields/Fitzwilliam}
% \collegeshield{CollegeShields/Girton}
% \collegeshield{CollegeShields/GonCaius}
% \collegeshield{CollegeShields/Homerton}
% \collegeshield{CollegeShields/HughesHall}
% \collegeshield{CollegeShields/Jesus}
% \collegeshield{CollegeShields/Kings}
% \collegeshield{CollegeShields/LucyCavendish}
% \collegeshield{CollegeShields/Magdalene}
% \collegeshield{CollegeShields/MurrayEdwards}
% \collegeshield{CollegeShields/Newnham}
% \collegeshield{CollegeShields/Pembroke}
% \collegeshield{CollegeShields/Peterhouse}
% \collegeshield{CollegeShields/Queens}
% \collegeshield{CollegeShields/Robinson}
% \collegeshield{CollegeShields/Selwyn}
% \collegeshield{CollegeShields/SidneySussex}
% \collegeshield{CollegeShields/StCatharines}
% \collegeshield{CollegeShields/StEdmunds}
% \collegeshield{CollegeShields/StJohns}
% \collegeshield{CollegeShields/Trinity}
% \collegeshield{CollegeShields/TrinityHall}
% \collegeshield{CollegeShields/Wolfson}
% \collegeshield{CollegeShields/CUniNoText}
% \collegeshield{CollegeShields/FitzwilliamRed}

%% Submission date [optional]:
% \submissiondate{November, 2042}

%% You can redefine the submission notice [optional]:
% \submissionnotice{A badass thesis submitted on time for the Degree of PhD}

%% Declaration date:
\date{My Month, My Year}

%% PDF meta-info:
\subjectline{Computer Science}
\keywords{one two three}


% too long, , 1 liner
% P1: open problem, 
% P2: current solutions -> problem remains
% P3: research questions (reporting on completed work)
%   - max 2 questions
%   - key innovations
% LENS: sparse attention mechanism for concepts -> predictions + truth table (logic) -> explainability by design
% CEM: concept embeddings -> break Acc-vs-Expl trade-off
% how do you measure?
% when we don't have labels?
% gaining interpretability without sacrificing accuracy

% remind story at the beginning and end of each chapter
% abstract: 1 page long
% write asbtract first
% write core chapters second
% chapter 2: highlight gaps + concepts ~20% + assume DL
% 
% ask x viva to check: story + what to do with mix pubs

%%%%%%%%%%%%%%%%%%%%%%%%%%%%%%%%%%%%%%%%%%%%%%%%%%%%%%%%%%%%%%%%%%%%%%%%%%%%%%%%
%% Abstract:
%%
\abstract{%
% The opaque reasoning of neural networks induces a lack of human trust.
Human trust in deep neural networks is currently an open problem as their decision process is opaque. 
Current solutions addressing this problem either (i) provide post-hoc, qualitative, and local explanations, or (ii) increase models' transparency compromising their accuracy.
To address these limitations, in this work I demonstrated how to design ``explainable-by-design'' neural models providing quantitative and global explanations without sacrificing accuracy. My key innovations enabling this progress are (i) a sparse attention mechanism and (ii) a high-dimensional representation for concepts learnt during training. On the one hand, the sparse attention mechanism allowed concept bottleneck models to solve each task using a small subset of relevant concepts and to learn simple logic-based explanations at train time. On the other hand, the high-dimensional representation of concepts breaks the information bottleneck of these concept-based models allowing them to gain accuracy without sacrificing interpretability. To make the proposed approaches more general, I devised an unsupervised concept layer which reduced training costs and allowed to train these models in absence of concept labels. As existing metrics were not always directly applicable, I proposed new performance scores to validate the impact of my innovations and to compare them with the current state of the art. The results of my experiments demonstrated how the proposed approaches significantly outperformed state-of-the-art models in predictive performance while providing accurate global explanations, thus breaking the current accuracy-vs-explainability trade-off and laying the foundations for human trust in deep learning.
}

% current limitations/knowledge gaps in XAI
% - accuracy and explainability
%   - visual (qualitative), local (instance-based), low-level (input-based), and post-hoc explanations
%   - interpretable models make accuracy worse (trade-off)
% - concepts make explanations more stable, but (1) require (expensive) annotations, and (2) do not solve the other issues
% my goals are
% - build models which are (a) more accurate and (b) more trustworthy than current models
% - provide models which (a) are explainable-by-design (i.e., they do not require a post-hoc external XAI method), (b) provide quantitative, global, and concept-based (high-level) explanations humans can easily understand and interact with



%%%%%%%%%%%%%%%%%%%%%%%%%%%%%%%%%%%%%%%%%%%%%%%%%%%%%%%%%%%%%%%%%%%%%%%%%%%%%%%%
%% Acknowledgements:
%%
% \acknowledgements{%
%   My acknowledgements ...
% }



%%%%%%%%%%%%%%%%%%%%%%%%%%%%%%%%%%%%%%%%%%%%%%%%%%%%%%%%%%%%%%%%%%%%%%%%%%%%%%%%
%% Glossary [optional]:
%%
% \newglossaryentry{HOL}{
%     name=HOL,
%     description={Higher-order logic}
% }



%%%%%%%%%%%%%%%%%%%%%%%%%%%%%%%%%%%%%%%%%%%%%%%%%%%%%%%%%%%%%%%%%%%%%%%%%%%%%%%%
%% Contents:
%%
\begin{document}



%%%%%%%%%%%%%%%%%%%%%%%%%%%%%%%%%%%%%%%%%%%%%%%%%%%%%%%%%%%%%%%%%%%%%%%%%%%%%%%%
%% Title page, abstract, declaration etc.:
%% -    the title page (is automatically omitted in the technical report mode).
\frontmatter{}



%%%%%%%%%%%%%%%%%%%%%%%%%%%%%%%%%%%%%%%%%%%%%%%%%%%%%%%%%%%%%%%%%%%%%%%%%%%%%%%%
%% Thesis body:
%%

%%%%%%%%%%%%%%%%%%%%%%%%%%%%%%%%%%%%%%%%%%%%%%%%%%%%%%%%%%%%%%%%%%%%%%%%%%%%%%%%
%% Introduction:
%%
\chapter{Introduction} \label{chapter:intro}
\textbf{Research: completed. Status: drafted. Difficulty: low. Priority: low.}

\textit{I will start my thesis with an overview of the state of the art of AI in research and companies and the potential impact of this field on humankind in the next decades. Next, I will discuss the current limitations and knowledge gaps which question the practical deployment of AI in high-stakes decision settings. I will conclude this chapter with an overview of my work describing how it advances the field AI and it contributes to a fairer and safer interaction with humankind.}


\section{Deep learning and explainability}

The rise of Artificial Intelligence (AI) poses ethical~\citep{duran2021afraid, lo2020ethical} and legal~\citep{wachter2017counterfactual, gdpr2017} questions, that are more and more compelling as the deployment of such technologies becomes commonplace in practice. Philosophical concerns turn into pressing needs in safety-critical domains which require accurate and trustworthy AI agents~\cite{rudin2019stop,shen2022trust}. 


Deep learning researchers stockpile ground-breaking achievements almost as fast as they find (consistently similar!) flaws in their models~\citep{marcus2022very}. The extremely high learning capacity may allow deep learning to achieve super-human performances on some tasks at the cost of making impossible even for researchers to trace back and explain incorrect predictions. As this trend got worse, lawmakers started questioning the ethical and legal ramifications of their deployment in safety-critical domains. As a response, the research community intensified the effort in developing trustworthy, fair and reliable models. This effort lead to relevant innovations aiming at explaining the inner workings of deep neural networks. However, after years of research, trustworthy deep learning models are still outside our reach.

\section{Limitations and knowledge gaps}

The key requisite for human trust is for an agent to show consistent and reliable behavior~\citep{shen2022trust}. The assessment of agents' behavior is commonly measured in terms of (i) task performance i.e., the capacity of the agent to provide \textbf{\textit{accurate predictions}} for test samples, and (ii) rationale i.e., the capacity of the agent to give \textbf{\textit{explanations}} for its predictions. While intense efforts lead to consistent advances in terms of explaining trained ``black-box'' models, most of these approaches turned out to be subject of similar limitations: they are mostly qualitative (mostly visual), local (instance-based), low-level (input-based), and post-hoc (they do not make a model trustworthy by design, they try to check if a model can be trusted). A first sign of change came only recently when \citet{koh2020concept} proposed to supervise the last hidden layer of neurons with human annotated concepts. This allowed the network to (i) be aware of ground-truth human concepts at training time, (ii) use learnt concepts to provide more intuitive high-level explanations, and (iii) interact with human experts correcting mispredicted concepts at test time. While this design significantly improved human trust, it did not solve the issue as (i) the explanations were still mostly local and qualitative and (ii) enforcing concept supervisions during training lead to worse task performance. As a result, finding a good compromise between accurate predictions and robust explanations remains one of the fundamental open problems in deep learning.


\section{Summary and contributions}

This work aims at improving the current trade off between accuracy and explainability by proposing novel model designs (i) showing higher task performance compared to the state of the art, and (ii) providing quantitative and global explanations for their predictions. This work opens with an introduction to deep learning (Chapter~\ref{chapter:intro}) and explainability literature with an in-depth discussion on the main limitations of current methods including concept-based and logic-based models, metrics, datasets, and benchmarks (Chapter \ref{chapter:background}). The central chapters describe the main technical advances of this work. Chapter~\ref{chapter:metrics} discusses quantitative metrics that will be used to analyze and compare architectures throughout this work in terms of their task performance and explanation quality. Chapter~\ref{chapter:lens} addresses the ``\textit{explainability problem}'' through Logic Explained Networks i.e., neural networks providing quantitative and global concept-based explanations for their predictions. Chapter~\ref{chapter:cem} addresses the ``\textit{trade off problem}'' through Concept Embedding Models i.e., concept-based models going beyond the current compromise between accuracy and explainability. Chapter~\ref{chapter:unsupervised} extends the methods described in previous chapters to settings where (expensive!) concept annotations are not available, and must then be learnt in an unsupervised way. Chapter~\ref{chapter:applications} showcases real-world case studies and results in diverse settings from vision to biomedical applications. The last chapter provides a summary of the advances, drawing conclusion and future perspectives (Chapter~\ref{chapter:conclusion}).

\bigskip

\textbf{PAPERS}
\nobibliography*
\begin{enumerate}
    \item Pietro Barbiero, ..., Giuseppe Marra. Interpretable Neural Symbolic Concept Reasoning. \textit{arXiv preprint arXiv:XXXX.YYYYY},2023
    
    \item Pietro Barbiero, ..., Elena Di Lavore. Categorical Foundations of Explainable AI: A Unifying Theory of Structures and Semantics. \textit{arXiv preprint arXiv:XXXX.YYYYY},2023
    
    \item \bibentry{barbiero2021entropy}
    
    \item \bibentry{zarlenga2022concept}
    
    \item \bibentry{ciravegna2021logic}
    
    \item \bibentry{magister2022encoding}
    
    \item \bibentry{zarlenga2021quality}

    \item \bibentry{azzolin2022global}
    
    \item Federico Siciliano, Pietro Barbiero, ..., Pietro Lio'. Explaining Neural Networks Using a Ruleset Based on Interpretable Concepts. \textit{arXiv preprint arXiv:XXXX.YYYYY},2023
    
    \item \bibentry{georgiev2022algorithmic}
    
    \item Dmitry Khazdan, ..., . Graph Concept Interpretation. \textit{arXiv preprint arXiv:XXXX.YYYYY},2023
    
    \item \bibentry{jain2022extending}
    
    \item Han Xuanyuan, Pietro Barbiero, ..., and Pietro Lio' Analysing the Neurons of Graph Neural Networks: Towards Concept-Based Global Interpretability. \textit{arXiv preprint arXiv:XXXX.YYYYY}, 2023
    
    \item Pietro Barbiero, and Pietro Lio'. Logic-based Deep Learning Clinical models: Down Syndrome case study. \textit{arXiv preprint arXiv:XXXX.YYYYY}, 2023
    
    \item \bibentry{kidwai2023forecasts}
    
    \item \bibentry{barbiero2021graph}
\end{enumerate}



%%%%%%%%%%%%%%%%%%%%%%%%%%%%%%%%%%%%%%%%%%%%%%%%%%%%%%%%%%%%%%%%%%%%%%%%%%%%%%%%
%% Related works:
%%
\chapter{Categorical foundations of explainable AI and concept learning} \label{chapter:background}

% Story of chapter:
% \begin{itemize}
%     \item motivation: why do we need a categorical foundations of XAI? no existing sound theory of XAI, this has bad implications. Examples. Here we provide the first theory which will also set the scene for the next chapters. Why category theory? Because it is general and goes to the essence. We can really build the foundations.
%     \item elements of category theory: feedback monoidal categories, cartesian streams, signatures. 
%     \item syntax (learning agent, explainable learning agent) and semantics (translator functor, explanation, understanding) of explainable AI
%     % \item categorical taxonomy of explainable AI: 
%     \item Concept learning: why concepts? what is a concept (definition)? examples of concepts. Concept research focuses on: concept quality (completeness and purity, Chapter 3), concept representations (chapter 4), concept explanations (chapter 5), and interpretable concept models (chapter 6). This thesis makes progress in all these research directions.
% \end{itemize}

\textbf{Motivation}---Explainable AI (XAI) research aims to address the human need for accurate and trustworthy AI through the design of interpretable AI models and algorithms able to explain uninterpretable AI models~\cite{arrieta2020explainable}. 
Some of these methods are so effective that their impact now deeply affects other research disciplines such as medicine~\cite{jimenez2020drug}, physics~\cite{schmidt2009distilling,cranmer2019learning}, and even pure mathematics~\cite{davies2021advancing}. 

A considerable number of works attempted to describe key methods and notions in this fast-growing literature~\cite{adadi2018peeking,Das2020OpportunitiesAC,arrieta2020explainable,dovsilovic2018explainable,tjoa2020survey,gunning2019xai,hoffman2018metrics,palacio2021xai}. 
However, none of these works are grounded on a solid and unifying theory of explainability, but they rather rely on qualitative descriptions, preventing them from drawing truly universal conclusions. Current surveys acknowledge this problem and grumble that key fundamental notions of explainable AI still lack a formal definition, and that the field as a whole is missing a unifying and sound formalism~\cite{adadi2018peeking,palacio2021xai}: The very notion of ``\textit{explanation}'' represents a pivotal example as it still lacks a proper mathematical formalization. The followings represent an example of some of the best definitions currently available in literature:
{\small
\begin{displayquote}
\textit{``An explanation is an answer to a `}why?\textit{' question.''~\cite{miller2019explanation}}

\textit{``An explanation is additional meta information, generated by an external algorithm or by the machine learning model itself, to describe the feature importance or relevance of an input instance towards a particular output classification.''~\cite{Das2020OpportunitiesAC}}

\textit{``An explanation is the process of describing one or
more facts, such that it facilitates the understanding of aspects related to said facts (by a human
consumer).''}~\cite{palacio2021xai}.
\end{displayquote}
}

As the interest for XAI methods rises inside and outside academic environments, the need for a sound formalization and encompassing taxonomy of the field grows quickly, as an essential precondition to welcome a wider audience. Indeed, the absence of a mathematical formalization of key explainable AI notions may severely undermine this research field, as it could lead to ill-posed questions, induce re-discovery of the same ideas, and make it difficult for new researchers to approach the domain.

\textbf{Solution}---To fill this knowledge gap, in this chapter we introduce the elements of the first formal theory of explainable AI and concept learning aiming to:
\begin{itemize}
    \item formalize key XAI notions for the first time;
    \item set the scene and motivate the work of the next chapters.
\end{itemize}
The \textbf{key innovation} of this chapter is the use of categorical structures to formalize XAI notions and processes.
We use category theory as it provides a sound and abstract formalism to study general structures and systems of structures, avoiding contingent details and focusing on their very essence. For this reason, category theory represents now the standard formalism of many mathematical disciplines, including algebra~\cite{eilenberg1945general}, geometry~\cite{bredon2012sheaf}, logic~\cite{johnstone2014topos}, computer science~\cite{goguen1992institutions}, and more recently machine learning~\cite{cruttwell2022categorical,ong2022learnable}.
%This is why in this work we decided to describe XAI notions using this formal language.

In this chapter we set the scene for the next chapters defining the key notions and presenting the notation we will follow in the rest of this work in the language of category theory. We will first discuss the basic elements of category theory analyzing the main categorical structures we will use in the following chapters (Section~\ref{sec:pre}). We will then use such categories to formally define the syntax and the semantics of explainable AI agents and notions (Section~\ref{sec:framework} and~\ref{sec:syExp}). Among available semantics, we will focus on the human-friendly semantics based on concept learning (Section~\ref{sec:concept-learning}). Finally, we will present the main knowledge gaps in concept learning which motivate the next chapters (Section~\ref{sec:gaps-concept-learning}).

% CEMs is a fully supervised high-dimensional concept representation. This high-dimensional representation increases the capacity of CEMs at concept level. The increased model capacity allows to encode more information in each concept beyond the probability of a concept being active/inactive, including contextual nuances which CEMs can use to have a deeper understanding of each concept and to solve tasks more efficiently.

% We will first present CEM's architecture~\ref{sec:cem} and then we will demonstrate how CEMs fill the key CBMs knowledge gaps we discussed with a set of experiments of increasing complexity~\ref{sec:cem}.

% For this reason, in this chapter we introduce the elements of the first formal theory of explainable AI and concept learning. This way we set the scene for the next chapters defining the key notions and presenting the notation we will follow in the rest of this work in the language of category theory. 

% Specifically, we use category theory as it provides a sound and abstract formalism to study general structures and systems of structures, focusing on their essence. 
% For this reason, many mathematical disciplines are affected by category theory, including algebra~\cite{eilenberg1945general}, geometry~\cite{bredon2012sheaf}, logic~\cite{johnstone2014topos}, and more recently machine learning~\cite{cruttwell2022categorical,ong2022learnable}.


% \section{Notation and basic definitions}
\section{Elements of category theory}
\label{sec:pre}
To make this work self-contained, this section introduces the minimal set of definitions that we will later need to formalize XAI systems i.e., feedback monoidal categories and the category of signatures. In particular, we will use feedback monoidal categories as a syntax to model structures sharing some of the key properties of AI systems, being able to: observe inputs, provide outputs, and receive feedback dynamically. Cartesian streams provide a semantics for these models. We will use the category of signatures to model the structure of ``explanations''.
% Further details can be found in the apendix.

\subsection{Monoidal categories}
The process interpretation of monoidal categories~\cite{Coecke2017,fritz2020} sees morphisms in monoidal categories as modelling processes with multiple inputs and multiple outputs.
Monoidal categories also provide an intuitive syntax for them through string diagrams~\cite{joyal1991geometry}.
The coherence theorem for monoidal categories~\cite{maclane78} ensures that string diagrams are a sound and complete syntax for them and thus all coherence equations for monoidal categories correspond to continuous deformations of string diagrams. One of the main advantages of string diagrams is that they make reasoning with equational theories more intuitive.

We recall the definitions of category and monoidal category.
Categories provide a syntax for processes that can be composed \emph{sequentially}.
\begin{definition}[\citet{eilenberg1945general}]
    A \emph{category} \(\cat{C}\) is given by a class of \emph{objects} $\obj{\cat{C}}$ and, for every two objects \(X,Y \in \obj{\cat{C}}\), a set of \emph{morphisms} $\hom(X,Y)$ with input type \(X\) and output type \(Y\). A morphism \(f \in \hom(X,Y)\) is written \(f \colon X \to Y\).
    For all morphisms \(f \colon X \to Y\) and morphisms \(g \colon Y \to Z\) there is a \emph{composite} morphisms \(f \dcomp g \colon X \to Z\).
    For each object \(X \in \obj{\cat{C}}\) there is an \emph{identity} morphism \(\id{X} \in \hom(X,X)\).
    Composition needs to be associative, i.e. there is no ambiguity in writing \(f \dcomp g \dcomp h\), and unital, i.e. \(f \dcomp \id{Y} = f = \id{X} \dcomp f\).
    % (\Cref{fig:category-string-diagrams}, bottom).
\end{definition}
%\begin{figure}[h!]
%    \centering
    $\sequentialFig{}$ \qquad \(\identityXFig\)\\[4pt]
    \compositionUnitalFig{}
    % \caption{String diagrams for the composition of two morphisms (top, left), the identity morphism (top, right), and the unitality condition (bottom).}\label{fig:category-string-diagrams}
%\end{figure}
A mapping between two categories $\cat{C}_1$ and $\cat{C}_2$ that preserves compositions and identities is called a \emph{functor} and maps objects and morphisms of $\cat{C}_1$ into objects and morphisms of $\cat{C}_2$.

% A \emph{category} consists of a class of \emph{objects} $\mathcal{O}$ and a class of \emph{morphisms} $\mathcal{H}$.
% \mike{maybe the $\mathcal{O}$ and $\mathcal{H}$ be in the end of the name because now it cut the flow of reading. Something like class of objects $\mathcal{O}$ .....}. 
% Each morphism $f\in\mathcal{H}$ has a domain object $X\in\mathcal{O}$ and a codomain object $Y\in\mathcal{O}$, written \(f \colon X \to Y\). Morphisms with a common interface can be composed and composition satisfies few axioms (see~\citet{maclane78} for more details). 
\begin{example}
$\Set$ is a category whose objects are sets (e.g., $X = \{\text{tree}, \text{sky}\}$ or $Y = \{\text{green}, \text{blue}\}$) and whose morphisms are functions between sets (e.g., $f: X \rightarrow Y$ such that $\text{blue} = f(\text{sky})$ and $\text{green} = f(\text{tree})$). 
\end{example}

Monoidal categories~\cite{maclane78} are categories endowed with extra structure, a monoidal product and a monoidal unit, that allows morphisms to be composed \emph{in parallel}.
The monoidal product is a functor \(\tensor \colon \cat{C} \times \cat{C} \to \cat{C}\) that associates to two processes, \(f_1 \colon X_1 \to Y_1\) and \(f_2 \colon X_2 \to Y_2\), their parallel composition \(f_1 \tensor f_2 \colon X_1 \tensor X_2 \to Y_1 \tensor Y_2\) (\Cref{fig:string-diagrams-monoidal-cat}, right).
The monoidal unit is an object \(\monoidalunit \in \obj{\cat{C}}\).
A monoidal category is \emph{symmetric} if there is a morphism \(\swap{X,Y} \colon X \tensor Y \to Y \tensor X\), for any two objects \(X\) and \(Y\), called the \emph{symmetry}.
%\begin{figure}[h!]
%    \centering
    $\morphismManyWiresFig{} \qquad \parallelFig{} \qquad \swapXYFig{X}{Y}$
    % \caption{A morphism with multiple inputs and outputs (left), the parallel composition of two morphisms (center), and the symmetry (right) in a monoidal category.}
%    \label{fig:string-diagrams-monoidal-cat}
%\end{figure}
A symmetric monoidal structure on a category is required to satisfy some coherence conditions \cite{maclane78} (see~\Cref{app:mon-cat} for more details), which ensure that string diagrams are a sound and complete syntax for symmetric monoidal categories \cite{joyal1991geometry}. 
Like functors are mappings between categories that preserve their structure, \emph{symmetric monoidal functors} are mappings between symmetric monoidal categories that preserve the structure and axioms of symmetric monoidal categories.


Some symmetric monoidal categories have the additional property of allowing resources and processes to be \emph{copied} and \emph{discarded}.
These are called \emph{Cartesian categories}.
A monoidal category \(\cat{C}\) is Cartesian whenever there are two morphisms, the copy \(\cp_X \colon X \to X \times X\) and the discard \(\discard_X \colon X \to 1\) (\Cref{fig:copy-discard}), that commute with all morphisms in \(\cat{C}\)~\cite{fox76} (see~\Cref{app:cat} for details).
When a monoidal category is Cartesian, it is customary to indicate with \(\times\) the monoidal product given by the Cartesian structure, and with \(1\) the corresponding monoidal unit. 
%\begin{figure}[h!]
    \[\copyXFig{} \qquad \qquad \discardXFig{}\]
    % \caption{The copy and discard morphisms in a Cartesian category.}
%    \label{fig:copy-discard}
%\end{figure}


%\paragraph{Monoidal category}
%A category with an associative unital tensor product, subject to certain coherence conditions, is said a \emph{Monoidal Category}~\cite{maclane78}. More specifically, a monoidal category has an additional structure defined by an operation $\times$ on both objects, like $X \times Y$ and with $I$ denoting the identity object, and morphisms, such that if $f \colon X_1 \rightarrow Y_1$ and $g \colon  X_2\rightarrow Y_2$ then 
%$f \times g \colon X_1 \times X_2 \rightarrow Y_1 \times Y_2$ 
%(satisfying the axioms in Appendix~\ref{app:cat}).
%In a monoidal category, a morphism \(f \colon X \to Y\) represents a process with input type \(X\) and output type \(Y\) (e.g. with possibly multiple inputs $X_1,X_2,X_3$ and outputs $Y_1,Y_2$). String diagrams can \emph{formally} describe a morphism as a box with input and output wires~\cite{joyal1991geometry}: 
%\[
%\morphismFig{} \hspace{1cm}
%\morphismManyWiresFig{}\]
% In monoidal categories, processes can have multiple inputs and multiple outputs:
% For example a morphism \(h \colon X_1 \tensor X_2 \tensor X_3 \to Y_1 \tensor Y_2\) is depicted as a box with multiple input and output wires: 
%The structure of a monoidal category allows two kinds of composition of processes: sequential and parallel.
%Given two morphisms \(f \colon X \to Y\) and \(g \colon Y \to Z\), they can be composed sequentially to obtain \(f\dcomp g \colon X \to Z\) which represents the process of applying \(f\) and then applying \(g\) to the output of \(f\) i.e.,
% This is depicted by connecting the output wire of \(f\) with the input wire of \(g\)
%$\sequentialFig{}$. Given \(f \colon X \to Y\) and \(f' \colon X' \to Y'\), they can also be composed in parallel to obtain \(f \times f' \colon X \times X' \to Y \times Y'\) i.e.,
% This is depicted by writing \(f\) on top of \(f'\).
%$\parallelFig{}$. 
%For every object \(X\), there exists an identity morphism \(\id{X} \colon X \to X\), i.e. such that \(f \dcomp \id{Y} = f = \id{X} \dcomp f\), given \(f \colon X \to Y\).
%This is depicted by a wire with type \(X\): \(\identityXFig\).
%The equation is intuitive in string diagrams:

% An example is \Cref{fig:composition-unital}.
% Another example is the middle-four interchange law \((f \tensor f') \dcomp (g \tensor g') = (f \dcomp g) \tensor (f' \dcomp g')\).
% These two expressions have one representation in terms of string diagrams.
% \[\interchangeLawFig{}\]
%The coherence theorem~\cite{maclane78} for monoidal categories ensures that string diagrams are a sound and complete syntax for morphisms in monoidal categories.



\subsection{Feedback monoidal categories and Cartesian streams}
With symmetric monoidal categories we can model AI systems that observe inputs and produce outputs.
Most AI learning algorithms, however, rely on \emph{feedback} to adjust their learning parameters: the learning phase for AI models is often dynamic, observing inputs, producing outputs, and getting feedback over and over again. 
Feedback monoidal categories provide a structure, on top of the structure of symmetric monoidal categories, to model this dynamic behaviour.

\begin{definition}[\cite{katis02,di2021canonical}]
A feedback monoidal category is a symmetric monoidal category \(\cat{C}\) endowed with an endofunctor $F:\cat{C} \rightarrow \cat{C}$, and an operation \(\fbk[S] \colon \hom (X \times F(S), Y \times S) \to \hom (X,Y)\) for all objects \(X,Y,S\) in \(\cat{C}\), which satisfies a set of axioms (see \Cref{app:feedback-cat}).
\end{definition}
% \elena{write diagrams! and fix itemize}\\
% \elena{feedback functors}\\
% \elena{free feedback monoidal categories}\\
%In addition, the learning phase for AI models is often dynamic, observing inputs, producing outputs, and getting feedback over and over again. To model this dynamic behavior, we can use the category of Cartesian streams~\cite{uustalu05,katsumata19}. 
%To define this category we need a Cartesian category i.e., a special monoidal structure allowing for resources to be copied \(\cp_X \colon X \to X \times X\) and discarded \(\discard_X \colon X \to 1\)~\cite{fox76}:
% We indicate with \(\times\) the monoidal product given by the Cartesian structure, and with \(1\) the corresponding monoidal unit. 
% for every object \(X\), with two morphisms \(\cp_X \colon X \to X \times X\) (\textit{copy}) and \(\discard_X \colon X \to 1\) (\textit{discard}):
%\[\copyXFig{} \qquad \text{and} \qquad \discardXFig{}\]
Feedback monoidal functors are mappings between feedback monoidal categories that preserve the structure and axioms of feedback monoidal categories.
Feedback monoidal categories are the \emph{syntax} for processes with feedback loops.
When the monoidal structure of a feedback monoidal category is cartesian, we call it feedback cartesian category.
Their \emph{semantics} can be given by monoidal streams~\cite{monoidalStreams}.
In cartesian categories, these have an explicit description.
We refer to them as cartesian streams, but they have appeared in the literature multiple times under the name of ``stateful morphism sequences''~\cite{katsumata19} and ``causal stream functions''~\cite{uustalu05}.

\begin{definition}[\citet{uustalu05}]
A \emph{cartesian stream} \(\stream{f} \colon \stream{X} \to \stream{Y}\), with \(\stream{X} = (X_0, X_1, \dots)\) and \(\stream{Y} = (Y_0, Y_1, \dots)\), is a family of functions \(f_n \colon X_n \times \cdots \times X_0 \to Y_n\) indexed by natural numbers.
Cartesian streams form a category \(\Stream{\Set}\).
For the details of this construction see \Cref{app:streams}.
\end{definition}
% \begin{proposition}[\citet*{monoidalStreams}]
% Streams form a feedback monoidal category.
% \end{proposition}

The main purpose for using Cartesian streams is their capability of capturing an entire flow of a training process while using the framework of a category.
A morphism in the category of cartesian streams encodes a process that receives an input \(X_n\) and produces an output \(Y_n\) at each time step \(n\).
%In this setting \(\stream{X} = (X_0, X_1, \dots)\) and \(\stream{Y} = (Y_0, Y_1, \dots)\) have the meaning to encode the flow of the input and output sets of the process $\stream{f}$ during the training.

% \todo{To check}

\begin{proposition}[\citet*{monoidalStreams}]
Cartesian streams form a feedback monoidal category denoted by \(\Stream{\Set}\).
\end{proposition}

As a result, we can use the category of Cartesian streams as semantics for structures sharing some of the key properties of AI systems, i.e. being able to: observe inputs, provide outputs, and receive feedback dynamically.


\subsection{Free monoidal categories and syntax}
%\elena{mention functorial semantics somewhere~\cite{lawvere1963functorial}, cite something for free constructions, for adjoints~\cite{kan1958adjoint}}
A syntax is a way of reasoning abstractly about structures without the need to know the details of any given structure.
In the same way a traditional syntax is defined by a set of symbols and some rules to combine them, free symmetric monoidal categories and free feedback monoidal categories are defined by a set of generators for objects and for morphisms.
The rules to combine them are given by the structure and axioms of symmetric monoidal categories and feedback monoidal categories, respectively.

When reasoning with a syntax, we want to ensure that the reasoning carried out still holds in the semantics.
This is done by symmetric monoidal functors, in the case of symmetric monoidal categorie, and feedback functors, in the case of feedback monoidal categories.
In fact, by definition of free symmetric monoidal (resp. feedback monoidal) category, once we fix the semantics of the generators, there exist a unique symmetric monoidal (feedback monoidal) functor to the semantics category.

We will employ free feedback monoidal categories as syntax for learning agents, and take semantics in the feedback monoidal category \(\Stream{\Set}\) of cartesian streams.

% \stefano{Proposta def free category: A \emph{free category} generated by a set of morphisms $\mathcal{G}$ is a category whose objects are source or target objects of generators in $\mathcal{G}$ and whose morphisms are the tuples of composable functions of $\mathcal{G}$. Intuitively, given a set of axioms for the composition, a free category generated by a set of morphisms is the most "general" category containing the generators and satisfying the chosen axioms.}



% \usepackage{graphicx}
\begin{table}[!t]
\centering
\caption{Reference for notation. List of main operations, objects, morphisms, categories, and functors.}
\label{tab:notation}
\resizebox{\columnwidth}{!}{%\small
\begin{tabularx}{\columnwidth}{cX}
% {\small
\toprule
\multicolumn{1}{c}{\textbf{\textsc{Symbol}}} & \multicolumn{1}{l}{\textbf{\textsc{Description}}} \\ \midrule
% \multicolumn{2}{c}{\textbf{Operations}}  \\
\multicolumn{2}{c}{\textbf{Objects}}  \\
$X$ & \textit{Input}: the input type of a model. \\
$Y$ & \textit{Output}: the output type of a model. \\
$P$ & \textit{Parameter}: the type of a model state. \\
$E$ & \textit{Explanation}: the output type of an explainer. \\
\\
\multicolumn{2}{c}{\textbf{Morphisms}}  \\
$\id{Z}$ & \textit{Identity}: the identity operation on $Z$. \\
$\hat{g}$ & \textit{Model}: given an input $X$ and parameter $P$, returns an output $Y$. \\
$\nabla_Y$ & \textit{Optimizer}: given a reference $Y$, a model output $Y$ and a parameter $P$, returns the updated parameter $P$. \\ 
$\hat{f}$ & \textit{Explainer}: given an input $X'$ and a parameter $P$, returns an output $Y'$ and an explanation $E$. \\ 
% $\tau$ & Translator: given an explanation $\mathcal{E}$, returns an output prediction $Y$. \\
\\
\multicolumn{2}{c}{\textbf{Categories}}  \\
\(\Stream{\Set}\) & \textit{Cartesian streams}: feedback monoidal category of Cartesian streams on $\Set$. \\
\(\mathsf{Learn}\) & \textit{Category of learners}: free feedback monoidal category generated by the objects $X$, $Y$, $P$, and the morphisms $g$ and $\nabla$.\\
\(\mathsf{XLearn}\) & \textit{Category of explainable learners}: free feedback monoidal category generated by the objects $X$, $Y$, $P$, $E$ and the morphisms $\eta$, $\nabla_Y$, and $\nabla_E$. \\
\(\mathsf{Sign}\) & \textit{Category of signatures}: category generated by the object $\Sigma$ and the morphism $\phi$. \\
\\
\multicolumn{2}{c}{\textbf{Functors \& Operators}}  \\
$\fbk[S]$ & \textit{Feedback}: the operation which brings an output back to the input. \\
\(Sen\) & \textit{Sentence}: functor from the category of signatures $\mathsf{Sign}$ to $\Sigma$-sentences over $\Set$. \\
$T$ & \textit{Interpreter}: functor from the category of learners \(\mathsf{Learn}\) to Cartesian streams $\Stream{\Set}$ over $\Set$. \\
\bottomrule
\end{tabularx}%
}
\end{table}
% \mike{This Table is a good idea to introduce it in the begin of the terminologies so the reader have it as a reference when he/she is confused. as there are a lot of terminologies. }
% \section{Categories of Learning Agents: Explainable and Concept-based Learners}

\subsection{Category of signatures}
In order to model objects of type ``explanation'', we will use the category of signatures~\cite{goguen1992institutions}.
% is  categorical theory of institutions and abstract formal concepts~\cite{goguen2005concept} study abstract syntactic structures. 
In institution theory~\cite{goguen1992institutions}, a signature $\Sigma$ constitutes the ``syntax'' of a formal language which serves as ``context'' or ``interpretant'' in the sense of classical logic~\cite{goguen2005concept}. Simple examples of signatures are given by First-Order Logic (FOL) theories and equational signatures.
% \begin{definition}[Signature~\citep{goguen1992institutions}]
% A \emph{signature} $\Sigma = (S_f, S_r, \text{ar})$ is a collection of:
% \begin{itemize}
%     \item a set of function symbols $S_f$
%     \item a set of relation symbols (or predicates) $S_r$
%     \item a morphism $\text{ar}: S_f \cup S_r \rightarrow \mathbb{N}$, which assigns a natural number called \textit{arity} to every function or relation symbol \alberto{The notation ``ar'' sucks, it is kinda incoherent with the rest. But I guess we are forced to use it because of previous papers, right?}
% \end{itemize}
% \end{definition}
% \stefano{I would not use the word morphism in this definition. Morphisms in mathemathics and in the paper have a specific meaning. I would use "function"}
Signatures form a category $\mathsf{Sign}$ whose objects are signatures and whose morphisms $\phi: \Sigma \rightarrow \Sigma'$ are interpretations between signatures corresponding to a ``change of notation''~\cite{goguen1992institutions}.
From this abstract vocabulary, institution theory defines abstract statements as sentences obtained from a vocabulary $\Sigma$~\cite{goguen1992institutions}.
\begin{definition}[$\Sigma$-sentence~\cite{goguen2005concept}]
There is a functor $Sen: \mathsf{Sign} \rightarrow \Set$ mapping each signature $\Sigma$ to the set of statements $Sen(\Sigma)$.
A $\Sigma$-\emph{sentence} is an element of \(Sen(\Sigma)\).
\end{definition}

\begin{example}\label{ex:sigma-sentence}
Let $\Sigma$ be a signature of propositional logic 
with $\{x_{\textit{flies}},$ $x_{\textit{animal}},$ $x_{\textit{plane}},$ $x_{\textit{dark\_color}},$ $\ldots\} = \textit{VAR}$, being $\textit{VAR}$ an infinite set of propositional variables and the standard connectives of Boolean logic, i.e. $\neg,\wedge,\vee,\rightarrow$. Then $x_{\textit{plane}} \wedge x_{\textit{flies}}$ is a $\Sigma$-sentence.
% Let $f$ be an explainer aiming at predicting an output in $\mc{Y}=\{x_{\textit{plane}}\}$ given an input in $\mc{X}\subseteq\textit{VAR}$.
\end{example}

% and each signature morphism $\sigma: \Sigma \rightarrow \Sigma'$ to the sentence translation map $Sen(\sigma): Sen(\Sigma) \rightarrow Sen(\Sigma')$.


%%%% moved in sec 3.4
% \citet{tarski1944semantic} and~\citet{goguen1992institutions} proved how the semantics of ``truth'' is invariant under change of signature. This means that we can safely use signature morphisms to %change ``notation'' 
% switch from one ``notation'' to another, inducing consistent syntactic changes in a $\Sigma$-sentence without impacting the ``meaning'' or the ``conclusion'' of the sentence~\cite{goguen1992institutions}. As a result, signature morphisms can translate a certain explanation between different signatures. 
%%%%%%%



% \begin{example}
% % A simple example of a $\Sigma$-sentence 
% A valid sentence in natural language is ``All men are mortal. Socrates is a man.
% Therefore, Socrates is mortal.''. Using a functor from natural language to first-order logic \elena{is there a functor from natural language to FOL? is natural language even a syntax in the mathematical sense?}, we can write the syllogism as ``$\forall x\ \text{man}(x) \rightarrow \text{mortal}(x)$, $\text{man}(Socrates) \Longrightarrow \text{mortal}(Socrates)$'', without changing its validity.
% \end{example}



\section{Syntax and semantics of explainable AI}
\label{sec:framework}
Here we formalize XAI structures and semantics using feedback monoidal categories and the category signatures $\mathsf{Sign}$. 
To this end, we first formalize the notions of ``learning agent'' (Section~\ref{sec:learning-agent}) and ``explainable learning agent'' (Section~\ref{sec:x-learning-agent}) as morphisms in free feedback monoidal categories generated by a \emph{model}, an \emph{optimizer}, and an \emph{explainer}.  Then we describe a functor translating these abstract notions into concrete instances in the feedback monoidal category of \(\Stream{\Set}\) (Section~\ref{sec:interpretation}). 
%%% OLD
Finally, we formalize the notion of ``explanation'' as a $\Sigma$-theory and of ``understanding'' as a signature morphism in (Section~\ref{sec:syExp}).
% To this end, we first describe the
% abstract entity we call “learning agent” which represents an
% abstraction for any human or artificial learner (Section 3.1).
% Secondly, from learning agents we derive another abstract
% entity we call “explainable learning agent” which represents
% an abstraction for special instances of learning agents which
% generate objects called “explanations” (Section 3.2). Finally,
% we describe a functor which translates these abstract entities
% into the concrete category of Set (Section 3.4). For guid-
% ance, Table 1 contains the main reference to the notation we
% use in this and following sections.
% \stefano{I would remove the distinction between concrete and non-concrete categories since it doesn't make sense in category theory}

% \todo{Overview of aims}:
% \begin{itemize}
%     \item aim 1: define learning agents
%     \item aim 2: define explainers
%     \item aim 3: interaction between learning agents / explainers
%     \item aim 4: define data for explanations
% \end{itemize}



\subsection{Syntax of learning agents}
\label{sec:learning-agent}
% \paragraph{What is learning?}
% We provide the formal definition of a learning agent from an abstract description of its features. In general, learning is a dynamic process which depends on a set of parameters $P$, like e.g. the weights of a neural network. More specifically, 
Generalizing~\citet{cruttwell2022categorical} to non-gradient-based systems, learning involves the following processes:
\begin{itemize}
    \item \textbf{Observing} a pair of objects $X$ and $Y$. In AI the objects $X$ and $Y$ represent input and output data of models.
    \item \textbf{Predicting} objects of type $Y$ from objects of type $X$, given a parameters of type $P$.
    \item \textbf{Updating} parameters $P$ according to a loss function. 
    % \alberto{But we are not updating the states $P$, we are updating the states of the parameters $P$, right?}
\end{itemize}
% \mike{Please kindly make more clear the $P$ what is represent because it is confusing. 'we ofter refer to the 'states P' as "parameters" ' or at P as "parameters". If it is the second, then in predicting terminology the 'parameters P' is redundancy}

Using this informal description as guidance,  we describe an abstract learning agent
% \(s\) \todo{do we use this symbol later?} 
as a morphism in the free feedback monoidal category generated by two morphisms: a model \(\hat{g} \colon X \times P \to Y\) and an optimizer \(\nabla_Y \colon Y \times Y \times P \to P\). The model $g$ produces an output of type \(Y\) given an input of type \(X\) and a parameter of type \(P\), while $\nabla_Y$ updates the parameters of the model \(P\) given a reference of type \(Y\), the predicted output of type \(Y\), and the parameters \(P\).
In order to specify the syntax of abstract learning agents we define the free category $\mathsf{Learn}$ of abstract models and optimizers.
% \todo{here we generate a free category because we want to generate the syntax of our systems}
\begin{definition}[Abstract model and optimizer]
The category $\mathsf{Learn}$ is the free feedback cartesian category generated by three objects, the input type \(X\), the output type \(Y\) and the parameter type \(P\), and by two morphisms, the model \(\hat{g} \colon X \times P \to Y\) and the optimizer \(\nabla_Y \colon Y \times Y \times P \to P\).
\[\learnModel \quad \learnOptimizer\]
% \todo{FIGURE: Add a few space between the letters}
\end{definition}
\begin{remark}
The output of the model and the reference may contain different elements, but they do have the same type, which is why we use the same type-symbol $Y$ to represent both objects. The same argument applies in the following whenever we have conceptually different inputs/outputs but denoting objects of the same type.
\end{remark}
Having defined the free category for abstract models and optimizers, we formalize an abstract learning agent as a morphism in $\mathsf{Learn}$.
% \stefano{do we need a $Y'$ for the optimizer?}
% \mike{the optimizer has to Y. If i understand correctly the one is the out put of the g morphism and the other is the reference? ground truth of supervised learning?}
% \mike{the next string diagram is more clear than the above one. Can you name different the Y reference/groundtruth so the reader knows that is not the output of g?}
\begin{definition}[Abstract learning agent]
An abstract learning agent is the morphism in $\mathsf{Learn}$ given by the composition $\fbk[P]\left((\id{Y}\times\id{X}\times\nu_P);(\id{Y}\times \hat{g} \times \id{P});\nabla_Y\right):$
% $(\id{Y} \times g) \dcomp \nabla)$:
% $\fbk[P]((\id{Y} \times g) \dcomp \nabla)$
\[\learnCat\]
% \fg{I think the full dot should be before the split with $g$ and I'll represent somehow the $\fbk[P]$?} 
\end{definition}
% \todo{do we need this proposition?}
% \begin{proposition}
% % \elena{Consider whether we want to do this, maybe with the Para construction?}
% Learning agents form a feedback monoidal category. 
% \end{proposition}


\subsection{Syntax of explaining agents}
\label{sec:x-learning-agent}
% \paragraph{What is explainable learning?}
% After defining what an abstract learning agent is, w
% Here describe what makes a learning agent explainable. As for learning agents, we derive a formal definition of an explainable learning agent from an abstract description of its features. \todo{add running example} 
Compared to learning, learning to explain involves the following processes:
\begin{itemize}
    \item \textbf{Observing} a pair of objects $X$ and $Y$.
    \item \textbf{Predicting} objects of type $Y$ from the observed $X$ given a set of parameters $P$.
    \item \textbf{Predicting} explanation objects of type $E$
    % with signature $\Sigma$, 
    from the observed $X$ given the parameters $P$.
     \item \textbf{Updating} the parameters $P$ according to a loss function over the predicted objects $Y$ or $E$.
% \mike{It is a bit tricky the 'updating' here. There are explainable learning agents that they did not need any loss minimization and training process (local XAI like attribution (shap,LRP,lime) and sensitivity methods (GradCam etc). except if by explainable learning agents you mean something different that i did not understand.??? }
% \fg{An explaining learning agent should assume a training procedure, while a simple explainer can just be used to predict an explanation. We can make an example referring then to the next section where there are all the differet instantiations.}
\end{itemize}
\begin{remark}
    Not all XAI methods have or need to update parameters. In our formalism, we describe the updating process of these ``static'' systems with an identity morphism.
\end{remark}
From this informal description we conclude that learning to explain extends learning processes manipulating an extra object, called ``explanation''. To define the category of explainable learning agents we need: (i) a new morphism $\hat{f}: X \times P \rightarrow Y \times E$, called \textit{explainer}, that provides both predictions $Y$ and explanations $E$,
% $\nabla_{Y'}: Y' \times Y' \times P' \rightarrow P'$ 
and (ii) a new morphism $\nabla_E: E \times E \times P \rightarrow P$ to optimize the agent parameters through its explanations $E$.
% \mike{again same comments as in learning agent. E and E I will kindly suggest to be different like $E_r$ or $E_{GT}$ or something that shows reference or ground truth.}
 \begin{definition}[Abstract explainer and optimizer]
     The category $\mathsf{XLearn}$ is the free feedback cartesian category generated by four objects, the input type $X$, the output types $Y$ and $E$, the parameter type $P$, and by three morphisms, the explainer $\hat{f}: X \times P \rightarrow Y\times E$, 
     % , the translator $\tau: \mathcal{E} \rightarrow \mathcal{Y}$, 
     and the optimizer $\nabla_{Y}: Y \times Y \times P \rightarrow P$: 
     % and $\nabla_E: E \times E \times P \rightarrow P$:
\[\learnExplainer \quad \learnOptimizerX \]%\quad \learnOptimizerExplanations \]
 \end{definition}
% Notice how the composition of an explainer and a translator $\eta \circ \tau$ yields a model of type $g$, and thus the category $\mathsf{XLearn}$ is a subcategory of $\mathsf{Learn}$ \todo{please check this!!}.
% \stefano{As free monoidal categories should be the other way round since the set of generators of XLearn is bigger than Learn but I'm not sure}
Through these new morphisms we formalize an abstract learning agent manipulating explanations.
% \todo{sometimes the explanation is a by-product}
% \stefano{no def of XLearn}
% \mike{again E, E , Y, Y some readers will think is the same}
\begin{definition}[Abstract explainable learning agent]
An abstract explainable learning agent is the morphism in $\mathsf{XLearn}$ given by the composition:
\[ \fbk[P]\left((\id{Y\times X}\times\nu_{P});(\id{Y}\times \hat{f} \times \id{P}); (\id{Y\times Y}\times\swap{E,P}) ;(\nabla_{Y}\times \discard_E)\right)\]
% $\fbk[P]((\id{Y} \times \eta ) \dcomp \nabla)$:
\[\xLearnCat\]
% \elena{change the diagram to send E as output to the top right, same for Y below}
% or % by the composition 
% ${\scriptstyle \fbk[P]\left((\id{E\times X}\times\nu_{P});(\id{E}\times \eta \times \id{P}); (\id{E}\times \swap{Y,E\times P});(\nabla_{E}\times \discard_Y)\right):}$
% % $\fbk[P]((\id{E} \times \eta ) \dcomp \nabla_E)$:
% \[\xLearnCatExplanations\]
% \[\fbk[P]((\id{E} \times \eta ) \dcomp \nabla_E).\]
% \[\xLearnCat\]
\end{definition}
\begin{remark}
    Our formalism allows for two different optimization processes to update the parameters of explainable learning agents. The first option optimizes explanations indirectly by applying a standard optimizer $\nabla_{Y}$ over the predictions of the explainer and reference labels of type $Y$. The second option instead optimizes explanations directly by applying the optimizer $\nabla_E$ over the explanations of the explainer and reference explanations of type $E$.
\end{remark}
% \todo{add definition of agreement?}
% \begin{proposition}
% % \elena{Consider whether we want to do this, maybe with the Para construction?}
% Explainable learning agents form a feedback monoidal category. 
% \end{proposition}
% Notice how in our formalism ``explaining'' is not necessarily a dynamic process as learning. In general, explaining is a static process if the explainer is detached from the translator. Thus, in our framework the primary function of the translator is to make the explaining process dynamic by mapping explanations $\mathcal{E}$ into outputs $Y$ which an optimizer $\nabla$ can process to update the parameters $P$.

% \begin{remark}
%     To be more general, we defined the free categories of $\mathsf{Learn}$ and $\mathsf{XLearn}$ by using different object types $X$, $Y$, $P$. 
%     However, we can always define a unified free category with an explainer working on the same input/output type of a learning agent (as often happens in the practice). To generate this category we can use the objects $X,Y,P,E$ and the morpisms $g:X\times P\rightarrow Y$, $\eta:X\times P\rightarrow Y\times E$, $\nabla_Y:Y\times Y\times P\rightarrow P$, $\nabla_E:E\times E\times P\rightarrow P$.
% \end{remark}




\subsection{Semantics of AI agents}
\label{sec:interpretation}
So far we described key notions of XAI in an abstract way, i.e. in terms of free feedback monoidal categories. However, we need these notions to be ``concrete'' before using them in typical AI settings. To this end, we specify the semantics of abstract agents via a functor from the free category $\mathsf{Learn}$ to the concrete category of Cartesian streams over \(\Set\), i.e. \(\Stream{\Set}\). With this functor we can model the specific types of inputs, outputs, parameters, explanations, and signature we need in a specific context. We refer to such a functor as ``translator'' and define it as follows.
\begin{definition}[Translator]

\begin{itemize}
\item[]
    \item A functor \textbf{$\mathrm{T}_A$} from $\mathsf{Learn}$ to \(\Stream{\Set}\) is said an agent translator.
    \item Given a signature $\Sigma$, a functor $\mathrm{T}_{\Sigma}$ from $\mathsf{XLearn}$ to \(\Stream{\Set}\), where $\mathrm{T}_{\Sigma}(E)\subseteq Sen(\Sigma)$ is said an explainer translator.
\end{itemize}
\end{definition}

Through these functors we can specify concrete semantics and describe typical AI functions and objects. For instance, we can map abstract objects and morphisms from $\mathsf{Learn}$ and $\mathsf{XLearn}$ to concrete streams of sets and functions. To simplify the notation we will use the following shortcuts: $\mathcal{X} = \mathrm{T}_A(X)$, $\mathcal{X'} = \mathrm{T}_{\Sigma}(X)$, $\mathcal{Y} = \mathrm{T}_A(Y)$, $\mathcal{Y'} = \mathrm{T}_{\Sigma}(Y)$, $\mathcal{P} = \mathrm{T}_A(P)$, $\mathcal{P'} = \mathrm{T}_{\Sigma}(P)$, and $\mathcal{E} = \mathrm{T}_{\Sigma}(E)$. In AI these objects often take values in vector spaces e.g., $\mathcal{X} \subseteq \mathbb{R}^n$, $\mathcal{P} \subseteq \mathbb{R}^p$, $\mathcal{Y} \subseteq \mathbb{R}^t$. In addition, for clarity we will denote $g=T_A(\hat{g})$, $f=T_{\Sigma}(\hat{f})$ and $\hat{\nabla}=T(\nabla)$.
Translator functors allow us to model different types of real-world learners, including AI agents.
\begin{definition}[Concrete learning agent]
Given a translator $\mathrm{T}_A$ between $\mathsf{Learn}$ and $\mathsf{Stream_{Set}}$, we call concrete learning agent, or simply an AI agent, the image $\mathrm{T}_A(\alpha)$ of the abstract learning agent, where
${\displaystyle \alpha = \fbk[P]\left((\id{Y}\times\id{X}\times\nu_P);(\id{Y}\times g \times \id{P});\nabla_Y\right).}$
% \todo{not sure how to frame this one...}
% \textit{Human agent} A human agent is a concrete learning agent whose objects are in \(\Set\) and whose morphisms are functions in the human brain.
\end{definition}
% Using the translator functor $\mathrm{T}$, we can also describe some of the most common learning paradigms in AI such as supervised and unsupervised learning.
% % \elena{All the following definitions should be given in terms of the shape of the semantics functor}\todo{what do you mean?}
% \begin{example}[Learning Paradigms]
% \textit{Supervised Learning. } Supervised Learning is a concrete learning process whose morphisms are functions in \(\Set\), whose objects are in \(\Set\), and whose optimizer updates the parameters $P$ such that the output $Y$ is close to the input labels $Y$.
% \textit{Unsupervised Learning. } Unsupervised learning is a concrete learning process whose morphisms are functions in \(\Set\), whose objects are in \(\Set\), and whose input $Y$ is empty.
% \end{example}
% \stefano{If we mention morphisms we have to mention their category (Learn). I would not use concrete}
% \begin{example}[Gradient-based Learning Agent]
% Using $\mathrm{T}_A$, we can also model the categorical definition of gradient-based learning proposed by~\citet{cruttwell2022categorical} as a special case:
% % , which includes some of the most studied objects in explainable AI i.e., neural networks~\cite{lecun2015deep}.
% A gradient-based learning agent is an AI agent whose optimizer generates parameters updates using the gradient of a differentiable loss function over the parameters.
% % of type $\mathcal{L}(Y,g(X))$ w.r.t. the current parameter states $P_t$ i.e., $P_{t+1} = P_{t} - \alpha \frac{\partial \mathcal{L}(Y,g(X))}{\partial P_t}$.
% \end{example}
We can use this formalism to provide a more precise view of the dynamic process of learning.
% \begin{definition}[Learning]
% Learning is the process of a learning agent which keeps updating its parameters until it reaches a stationary state. 
% where  $\frac{\partial P}{\partial t} \rightarrow 0$ 
% i.e. $\nabla_Y \approx (\epsilon_Y \times \epsilon_Y \times\id{P}$).
% \fg{to check}
% \todo{Improve the description by cartesian streams? example of learning agent?
% exists n such taht.... then convergence}
% \todo{add example showing how we can update the paramters}
% \todo{e se lo cambiassimo direttamente in esempio di una rete neurale? e gli si aggancia direttamente come esce il learning sullo stream}
In fact we can describe learning as the process of a concrete learning agent which keeps updating its parameters until it eventually reaches a stationary state. Given a learning agent $L$ this process is represented by the image of $L$ through the translator functor $T_A$. A learning process is convergent if there exist $k$ such that the cartesian stream $T_A(L)$ has $g_{n+1} \approx g_n| X_n \times \cdots \times X_0$ and $X_{n+1} = X_n$ for $n > k$.
% \todo{definizione di convergenza del learning: c'è convergenza quando $\nabla(y,y,p)=p$ per ogni $y\in \mc{Y}, p\in \mc{P}$
% Non credo ci sia convergenza in codesto caso. Per dare la definizione di convergenza dobbiamo usare gli streams perchè è li che è codificato il training}
% \end{definition}


\begin{definition}[Concrete explainable agent]
Given a translator $\mathrm{T}_{\Sigma}$ between $\mathsf{XLearn}$ and $\mathsf{Stream_{Set}}$, we call concrete explainable agent, or simply an AI agent, the image $\mathrm{T}_{\Sigma}(\alpha)$ of the abstract explainable agent, where:
${\scriptstyle \alpha=\fbk[P]\left((\id{Y\times X}\times\nu_{P});(\id{Y}\times f \times \id{P}); (\id{Y\times Y}\times\swap{E,P}) ;(\nabla_{Y}\times \discard_E)\right)}$. 
% or ${\scriptstyle \alpha=\fbk[P]\left((\id{E\times X}\times\nu_{P});(\id{E}\times \eta \times \id{P}); (\id{E}\times \swap{Y,E\times P});(\nabla_{E}\times \discard_Y)\right)}$
% \[
% \begin{array}{l}
% \fbk[P]((\id{Y\times X}\times\nu_{P});(\id{Y}\times \eta \times \id{P});\\
% ;(\id{Y\times Y}\times\swap{E,P}) ;(\nabla_{Y}\times \discard_E))
% \end{array}
% \]
% \todo{not sure how to frame this one...}
% \textit{Human agent} A human agent is a concrete learning agent whose objects are in \(\Set\) and whose morphisms are functions in the human brain.
\end{definition}



%\begin{example}[Gradient-based Learning Agent]
    %This definition can be seen a special case of an instantiation of $\mathsf{Learn}$ through a translator $\mathrm{T}_A$, so as the category $\mathsf{ParaLens}$ they refer to is a special case of $\mathsf{Stream}_{\mathsf{Set}}$ category \todo{provide reference for a theorem, or change the sentence} describing only one step of the training. For what concern string diagrams representation, we notice that our diagram represent a generalization of the one used in~\citet{cruttwell2022categorical}, where both the optimizer and the loss function are included in the $\nabla_Y$ morphism. 
    % (see Fig. \ref{fig:gblavsala}).
    % \begin{figure}
    %     \centering
    %     \includegraphics[width=0.5\textwidth]{figs/string diagrams.png}
    %     \caption{Gradient-based learning agents proposed by \cite{cruttwell2022categorical} are a special case of a concrete learning agents.}
    %     \label{fig:gblavsala}
    % \end{figure}

%\end{example}



% \begin{itemize}
%     \item we need now to go from abstract categories to concrete categories
%     \item first we define an interpreter as a functor mapping free cat in concrete cat
%     \item then we can start discussing: learning paradigms (supervised vs unsupervised), human vs AI learner, gradient-based learner, classifiers\&regressors, etc
% \end{itemize}

% \todo{The following are examples}

% \elena{All the following definitions should be given in terms of the shape of the semantics functor}

% \begin{definition}[Adversarial Learning]
% Adversarial learning is a learning process where the optimizer updates the parameters $P$ such that the predictions $B$ are far from the input supervisions $B$.
% \end{definition}

% Figure \ref{fig:learning_agent} formally describes the category of learning agents \(\Learn{\cat{Stream}}\) by means of its string diagrams. Figure \ref{fig:learning_agent_composition} formally describes the composition of learning agents in series and in parallel.
% The composition of learning agents is still a learning agent. The composition of models is still a model. A model can be a learning agent by itself (with its own internal model and optimizer).
% \begin{figure}[!h]
%     \centering
%     \includegraphics[width=0.2\textwidth]{figs/cat4xai_3-simple block.png}
%     \includegraphics[width=0.2\textwidth]{figs/cat4xai_3-learning.png}
%     \caption{Learning agent: $s = g \circ \nabla$.}
%     \label{fig:learning_agent}
% \end{figure}
% \begin{figure}[!h]
%     \centering
%     \includegraphics[width=0.2\textwidth]{figs/cat4xai_3-composition.png}
%     \includegraphics[width=0.2\textwidth]{figs/cat4xai_3-comp parallel.png}\\
%     \includegraphics[width=0.2\textwidth]{figs/cat4xai_3-learning_comp.png}
%     \caption{Compositions of agents: in series and in parallel.}
%     \label{fig:learning_agent_composition}
% \end{figure}

% \todo{The following are examples}

% \elena{I am not sure what \(\Learn{\Stream{}}\) should be}

% \begin{example}[Learning Classifier]
% A learning classifier is a concrete learning agent where supervisions and predictions are objects $A,B \in [0,1]^k$.
% \end{example}


% From objects to concepts 

% \section{``Explain'' the Explainable AI Literature}


% Add:
% \begin{itemize}
%     \item agreement (concordant explanations): $\frac{\partial P}{\partial t} \rightarrow 0$ with $\nabla_E$
% \end{itemize}
% \mike{Please consider to include a small introduction to explain the difference of an explainable agent and explainable method, so the reader can know what will read in the section 3 and 4. Right now the manuscript is a bit confusing as you start immidiatly with literature review and terminology. As i was reading the manuscript I was expected to read about XAI methods and generalized terminology. In section 3 that you speak about the XAI agents, i was thinking about the XAI methods which you explain in 4 and this confused me a lot. Just include a small paragraph of 3 sentences that explain the XAI framework with the XAI agency, the method and the communication for the explanation before start the terminology (a scheme will be even better!)}


\section{Explanations and understanding}
\label{sec:syExp}
\subsection{What is an explanation?}
So far we just considered an ``explanation'' as a special object generated by an explainer morphism $\hat{f}$, depending on its input $X$ and/or parameters $P$. 
% According to the introduced definition, an explainable learning agent distinguishes from a learning agent as it also generates a special object $E$, called ``explanation''. 
We are now interested in analyzing the properties of this special object.
As previously discussed, the XAI research community agrees in considering explanations as ``answers to \textit{why?} questions''~\cite{miller2019explanation}. 
% In our categorical framework, an explanation is an output of the explainer $\eta$, depending on its input $X'$ and/or parameters $P'$. %~\cite{Das2020OpportunitiesAC}.
% A typical explanation can thus describe a causal relationships between events. Humans typically disambiguate events in ``causes'' and ``effects'' such that whenever the ``causes'' occur, then the ``effects'' must follow~\cite{todo}. 
% In our formalism all objects  the learning agent parameters $P$, the data $X$ are the ``causes'' while the agent predictions $Y$ are the ``effects''.
Here we generalize this idea providing the first formal definition of the term ``explanation'', which embodies the very essence and purpose of explainable AI.
% \todo{change notation for single explanation and explanation object type}
\begin{definition}[Explanation]
    Given a $\Sigma$ signature and a concrete explainer $f=T_{\Sigma}(\hat{f}):\mc{X}\times \mc{P}\rightarrow \mc{Y}\times \mc{E}$,
    % over the input $\mc{X}$ and parameters $\mc{P}$, 
    an explanation $\mc{E} = T_{\Sigma}(E)$ in a signature $\Sigma$ is a set of $\Sigma$-sentences (i.e. a $\Sigma$-theory). 
\end{definition}

% \begin{example}
%         Let assume  $\Sigma$ to be a signature of Propositional Logic 
%     with $\{x\} = \textit{VAR}$, being $\textit{VAR}$ the set of propositional variables, a set of relations $R = \{Files(x), Animal(x), Plane(x), \dots\}$ and the standard connectives of Boolean Logic, i.e. $\neg,\wedge,\vee,\rightarrow$. For instance, $\hat{\eta}$ could be an explainer aiming at predicting an output in $\mc{Y}=\{\textit{plane}\}$ given an input in $\mc{X}$.    
%     Then an explanation could consist in a $\Sigma$ sentence, like 
%     % $\{\textit{Plane}(x) \rightarrow p_{\textit{flies}}$, $p_{\textit{flies}} \wedge \neg p_{\textit{animal}}$, $
%     $\varepsilon = Files(x) \wedge\neg Animal(x)\rightarrow Plane(x)$.
% \end{example}





% \begin{remark}
Our definition of explanation generalizes and formalizes the best definitions currently available in literature such as the ones we presented at the beginning of this chapter. In fact, existing definitions informally represent special forms of explanations. For example, according to~\citet{Das2020OpportunitiesAC} and~\citet{palacio2021xai} an explanation provides additional meta information to describe facts related to the explainer, including the feature relevance of an input. This represent the simplest form of explanation and corresponds to a pure description of the most relevant inputs. Seminal XAI methods typically provide this form of descriptions by showing the most relevant input attributes for the prediction of a given sample, as it happens in saliency maps~\cite{simonyan2013deep}, Concept Activation Vectors~\cite{kim2018interpretability}, and SHapley Additive exPlanations~\cite{lundberg2017unified}. The following example illustrates this form of explanation.
\begin{example}
\label{ex:logrel}
Let $\Sigma'$ be a vocabulary extending the one in Example~\ref{ex:sigma-sentence} with two additional symbols $R=\{\textit{relevant}, \textit{irrelevant}\}$. We consider as sentences in this language, expressions of kind $(x_1:r_1,\ldots,x_n:r_n)$, where $x_i\in \textit{VAR}$ and $r_i\in R$ for $n\in\mathbb{N}$, $i=1,\ldots,n$. Let $f$ be an explainer aiming at predicting an output in $\mc{Y}=\{x_{\textit{plane}}\}$ given an input in $\mc{X}\subseteq\textit{VAR}$. Then an explanation describing the most relevant inputs is a $\Sigma'$-sentence such as $\varepsilon'=(x_{\textit{flies}}:\textit{relevant},x_{\textit{animal}}:\textit{relevant},x_{\textit{dark\_color}}:\textit{irrelevant})$.
\end{example}
A more advanced form of explanation describes specific combinations of attributes leading to specific predictions. This form of explanation is common in rule-based systems such as decision trees~\cite{breiman1984classification} and Generalized Additive Models~\cite{hastie2017generalized}. Explanations of this form may represent an answer to a ``why?'' question such as to why a specific input instance leads to a specific output~\cite{miller2019explanation,Das2020OpportunitiesAC}, as we illustrate in the following example.
\begin{example}
\label{ex:logsig}
    Let $\Sigma$ be the vocabulary in Example~\ref{ex:sigma-sentence}. Let $f$ be an explainer aiming at predicting an output in $\mc{Y}=\{x_{\textit{plane}}\}$ given an input in $\mc{X}\subseteq\textit{VAR}$. Then the $\Sigma$-sentence $\varepsilon = x_{\textit{flies}}\wedge\neg x_{\textit{animal}}\rightarrow x_{\textit{plane}}$ explains why the input is classified as type ``plane''.
\end{example}

% \end{remark}
% As a simple example of a $\Sigma$-sentence in natural language is ``All men are mortal. Socrates is a man.
% Therefore, Socrates is mortal.''. 
% Using a functor from natural language to first-order logic, we can write the syllogism as ``$\forall x\ \text{man}(x) \rightarrow \text{mortal}(x)$, $\text{man}(Socrates) \Longrightarrow \text{mortal}(Socrates)$'', without changing its validity.
% \todo{PIANO: esempio della rilevanza+tarski e co + def comunicazione tra explainer? + esempio conversione di spiegazioni.. .e quindi understanding!}
% The simplest form of explanation is a pure description of the most relevant inputs~\cite{Das2020OpportunitiesAC}. Seminal XAI methods typically provide this form of descriptions by showing the most relevant input attributes for the prediction of a given sample, as it happens in saliency maps~\cite{simonyan2013deep}, Concept Activation Vectors~\cite{kim2018interpretability}, and SHapley Additive exPlanations~\cite{lundberg2017unified} . A more advanced form of explanation describes specific combinations of attributes leading to specific predictions. This form of explanation is common in rule-based systems such as decision trees~\cite{breiman1984classification} and Generalized Additive Models~\cite{hastie2017generalized}. 
% Finally, following existing surveys~\cite{Das2020OpportunitiesAC}, we can further distinguish the semantics of explanations on whether they hold for a single sample (as in LIME~\cite{ribeiro2016should}) or for a more general group of samples (as in decision trees~\cite{breiman1984classification}).

% \todo{esempio da togliere, non è una signature per come la intende Gougen}

\begin{remark}
\citet{tarski1944semantic} and~\citet{goguen1992institutions} proved how the semantics of ``truth'' is invariant under change of signature. This means that we can safely use signature morphisms to %change ``notation'' 
switch from one ``notation'' to another, inducing consistent syntactic changes in a $\Sigma$-sentence without impacting the ``meaning'' or the ``conclusion'' of the sentence~\cite{goguen1992institutions}. As a result, signature morphisms can translate a certain explanation between different signatures.
\end{remark}
While signature morphisms do not change the meaning of an explanation, they may have a great impact on human observers as we discuss in the next section.


% \paragraph{Semantics of Explanations}
% While data semantics supplies the raw material for $\Sigma$-sentences, the form of a sentence determines the . 


%%%%%%work on it for the rebuttal
%%%%%%%%%%%%%%%%%%%%%%%%%%%%%%%%%%%%%%%%%%%%%%%%%
%%% PIE IN CASO TAGLIA TUTTO DA QUA ALLA FINE DELLA SEZ
% \todo{TOCHECK!! and in case remove (I would say also the understanding part)}
% On the other hand, in our framework an explanation is tightly related to the specific inputs and outputs of an explainable agent, then it is fundamental for comparing explanations in different signatures, to define how two explainers  can ``communicate" between them.
% \begin{definition}[Explainers' communication]
% Given two explainer $\hat{\eta}_1:\mc{X}_1\times\mc{P}_1\rightarrow \mc{Y}_1\times\mc{E}_1$ and $\hat{\eta}_2:\mc{X}_2\times\mc{P}_2\rightarrow \mc{Y}_2\times\mc{E}_2$ with signatures $\Sigma_1$ and $\Sigma_2$ respectively, we say that $\hat{\eta}_1$ can communicate with $\hat{\eta}_2$ if it exists a function $\Phi:\mc{X}_1\times\mc{P}_1\times\mc{Y}_1\times\mc{E}_1\rightarrow \mc{E}_2$, such that $\Phi_{|\mc{E}_1}$ is a morphism between $Sen(\Sigma_1)$ and $Sen(\Sigma_2)$.
% \end{definition}
% \begin{example}
%     As an example of communication between explainable agents, and following Example \ref{ex:logsig} and \ref{ex:logrel}, we can consider $\Phi(x,p,y,\varepsilon)=(x_i:\textit{relevant},\ldots,x_j:\textit{irrelevant})$ if $x_i$s occur in $\varepsilon$ whereas $x_j$s do not.
% \end{example}
% \todo{To make more correct or delete it}
%%%%%%%%%%%%%%%%%%%%%%%%%%%%%%%%%%%%%%%%%%






\subsection{What is understanding?}
Tightly connected to explanation morphisms, ``understanding'' is another key notion in explainable AI which currently lacks a mathematical formalization. In the context of explainable AI, we are often interested in a specific type of understanding which~\citet{pritchard2009knowledge} refers to as \textit{understanding-why}. This form of understanding is often called \textit{explanatory understanding} and is ascribed in sentences that take the form ``I understand why Z'', where Z is an explanation (for example, ``I understand why the bread burnt as I left the oven on''). 
% Notice how this informal description is tightly connected with the informal description of an explanation as ``an answer to a \textit{why?} question''~\cite{miller2019explanation}. 
Using this intuition, we can formally define understanding as follows.
% \todo{agreement vs understanding}
% \todo{change to interpretation}
%%%% TO FIX DEPENDING ON THE PREVIOUS PART
\begin{definition}[Understanding]
    An explainable learning agent providing explanations in a signature $\Sigma'$ can understand the explanation $\mathcal{E}$ in the signature $\Sigma$ if and only if it exists at least one signature morphism $\phi: \Sigma \rightarrow \Sigma'$.
    % from the signature $\Sigma$ of the explanation and the signature $\Sigma'$ of the agent.
\end{definition}

\begin{remark}
    Notice that the existence of this morphism is not always guaranteed. This means that in some cases human observers may not be able to understand certain AI explanations. This happens even among human beings talking in two different (natural) languages. In other situations, a partial morphism may exist allowing a form of partial understanding. This happens for example in translating natural languages to formal languages.
    % Nevertheless, when two agents share the same ``language'', to allow for a kind of understanding to exist. \todo{add discussion on partial understanding e.g., from natural language to formal language}
\end{remark}

For this reason, choosing a good signature is key and often more important for human understanding than developing state-of-the-art explainers. In fact, signatures based on ambiguous syntax (e.g., natural language) may significantly degrade human understanding as bits and pieces of explanations might get lost in the change of notation. Conversely, signatures of formal languages (e.g., propositional logic) are robust under translation in other languages, including informal languages, providing stronger guarantees for human observers. The second aspect of a good signature is the choice of the symbols providing the raw material for compound explanations. The next section illustrates how the choice of symbols plays a crucial role for human understanding.

% Another interesting research line in AI consists in studying multi-agents learning environments, with possibly the human in the loop.  
% According to our proposed formalization, for instance we can define  the notion of ``agreement'' between two explainers by co-optimizing their explanations until convergence, or evaluating their communicated explanations.
% % \begin{definition}[Agreement]
% %     Given two explainable learning agents $\hat{\eta}$ and $\hat{\eta}'$ providing explanations in a signature $\Sigma$ and $\Sigma'$ if and only if exists a communication function $\Phi$ between them, such that $\nabla_{E'}$ converges if provided $\mc{E}'$ and $\Phi(\mc{E})$ as inputs.
% % \end{definition}
% % \todo{a questo punto 1 explainer capisce l'altro quando tramite il morfismo glielo posso attaccare al $\nabla_E$ e mi da convergenza..?}

% % \subsection{Semantics of Explanations}
% \label{sec:xai-semantics}
% While the taxonomy of XAI models mostly depends on how explainer and model functions are combined, the semantics of explanations mostly depends 
% % on the content of the data objects and the signature used to form sentences. 
% % The semantics of the explanation depends 
% on two factors: the semantics of the data and the form of a valid $\Sigma$-sentence.
% % While category theory formalizes explainable AI structures, the semantics of explanations mostly depends on the content of the objects transformed by the morphisms. From a categorical perspective, all the objects handled by AI models are ``formal concepts'' in \(\Set\), as we discussed in Section~\ref{sec:framework}. Given this general structure, we can still differentiate the semantics of data with regards to the semantics of explanations, as follows.

% % \paragraph{Semantics of data} 



\section{Concepts: semantics for human understanding}
\label{sec:concept-learning}

\subsection{Data semantics and human understanding}
The semantics of data forms the raw material for the semantics of explanations and plays a crucial role for human understanding. We describe the semantics of data in terms of the set of attributes used to characterize each sample and on the set of values each attribute can take. We usually refer to data objects as feature and label matrixes, corresponding to input objects $\mathcal{X}$ and target $\mathcal{Y}$ respectively. The semantics of a feature matrix varies depending on the attributes which typically represent pixels in images~\cite{kulkarni2022explainable}, relations in graphs~\cite{li2022survey}, words in natural language~\cite{danilevsky2020survey}, or semantically-meaningful variables (such as ``temperature'', ``shape'', or ``color'') in tabular data~\cite{di2022explainable}. Notice how different data types do not change the architecture of an explainable AI system. However, choosing a specific data type can lead to significantly different levels of human understanding. In fact, human understanding does not depend directly on the structure of the explainable AI system, but rather on the existence and completeness of a proper signature morphism from the explanation to the human observer. For example, humans lean towards explanations whose semantics is based on meaningful, human-understandable notions (such as ``temperature'', ``shape'', or ``color''), rather than explanations whose semantics is based on pixels.
% , as pointed out in seminal works in concept learning~\cite{kim2018interpretability}. 
In fact several works show how humans do not reason in terms of low-level attributes like pixels, but rather in terms of high-level ideas~\cite{goguen2005concept,ghorbani2019interpretation}. Thus explanations based on such semantics might significantly improve human understanding~\cite{ghorbani2019interpretation}.
This observations have roots in cognitive sciences (e.g., Representational Theory of the Mind). According to these theories ``\emph{concepts are the basic building blocks of human thoughts}''~\citep{margolis2007ontology}: following simple rules the human mind can combine finite stocks of basic concepts over and over again to create increasingly complex representations~\citep{margolis2007ontology}. For instance, the mind can combine the basic concepts ``roof'' and ``walls'' to generate the concept ``house''. 

\subsection{What is a concept?}
The relationship between data semantics and human understanding motivated~\citet{kim2018interpretability} to open a research line in concept learning in AI to increase human understanding. The objective of this field is to increase human trust by making AI use ``the same building blocks of human thought'' as opposed to other XAI approaches~\citep{kim2018interpretability}. Informally, we can define a concept as a human-understandable property shared by a set of objects. For instance ``roof'' is a property shared by all objects of type ``house''. Likewise, we can say that all objects of type ``house'' share the property of having a ``roof''.
% In particular concept-based XAI has the following aims:
% \begin{itemize}
%     \item discover concepts from trained models to explain their decisions in ``human'' terms (post-hoc explainability)
%     \item train models to learn specific concepts and compose them to solve tasks (self-explainability)
% \end{itemize} 
% Compared to other XAI approaches, concept-based XAI has solid foundations in cognitive sciences and a precise formalization in universal algebra.
Following~\citet{ganter1997formal} and~\citet{goguen2005concept}, we formalize the notion of ``concept'' as follows.
\begin{definition}[(Formal) Context~\citep{ganter1997formal}]
A formal context $\mc{K} := (\mc{A},\mc{B},\mc{I})$ consists of a set of objects $\mc{A}$, a set of attributes $\mc{B}$, and a set of relations $\mc{I}$ between $\mc{A}$ and $\mc{B}$.
\end{definition}
In order to express that an object $a \in \mc{A}$ is in relation with an attribute $b \in \mc{B}$, we write $(a, b) \in \mc{I}$ and read it as ``the object $a$ has the attribute $b$". For a set $\mc{A}' \subseteq \mc{A}$ of objects we can define the set of attributes common to the objects in $\mc{A}$: $\mc{A}^* := \{b \in \mc{B} \ | \ \forall a \in \mc{A}', \ (a,b) \in \mc{I} \}$. Similarly, we can define the the set of objects having all attributes in $\mc{B}' \subseteq \mc{B}$: $\mc{B}^* := \{b \in \mc{B} \ | \ \forall b \in \mc{B}', \ (a,b) \in \mc{I} \}$.
\begin{definition}[(Formal) Concept~\citep{ganter1997formal}]
A concept of the context $\mc{K} := (\mc{A},\mc{B},\mc{I})$ is a pair $(\mc{A}',\mc{B}')$ such that $\mc{A}' \subseteq \mc{A}$, $\mc{B}' \subseteq \mc{B}$, $\mc{A}^* = \mc{B}$, and $\mc{B}^* = \mc{A}$.
\end{definition}
\citet{ganter1997formal} refers to $\mc{A}'$ as the \textit{extent} and to $\mc{B}'$ as the \textit{intent} of the concept $(\mc{A}',\mc{B}')$. In AI, we often represent (formal) contexts using matrices where the rows are headed by sample identifiers, the columns are headed by attribute names, and the value of a cell represents the binary relation between a sample and an attribute. In these settings we often discriminate among different type of contexts depending on their use. Following common practice, we call ``feature matrix'' the context corresponding to the input type of an AI agent and we represent this context with the set $\mc{X} \subseteq \mathbb{R}^{d}$. We call ``label matrix'' the context corresponding to the output type of an AI agent and we represent this context with the set $\mc{Y} \subseteq \mathbb{R}^{l}$. 
% \begin{itemize}
%     \item $A \subseteq \{0,1\}^{n \times n}$ commonly known as adjacency matrix
%     \item $Y \subseteq \mathbb{R}^{n \times l}$ commonly known as the matrix of ``ground-truth labels''
%     % \item $C \subseteq \mathbb{R}^{n \times k}$ commonly known as the matrix of ``ground truth attribute labels''
% \end{itemize}
\begin{remark}
The same concept can have different representations depending on its intent $\mc{B}'$. In particular the intent plays a key role in assigning specific semantics to the context, thus affecting the semantics of explanations and human understanding, as illustrated in the following example.
\end{remark}
% For instance, two images representing the concept ``house'' need to have the exact same pixels representing the ``house''. This is because the context of ``pixels'' is not translation or rotation invariant. 
\begin{example}
Consider the concept ``house'' for a single sample described using different attributes i.e., pixels (image), node neighbors (graph), or labels (table):
\begin{figure}[H]
    \centering
%    \includegraphics{image_concept.pdf}
%    \includegraphics{graph_concept.pdf}
%    \includegraphics{label_concept.pdf}
    % \caption{input matrix: segment of image (subset of pixels), adjacency matrix: motif of graph (subset of nodes and edges), label matrix: a rectangle with a non-zero elements of the table (subset of rows and columns)}
    \label{fig:my_label}
\end{figure}
Notice how the form of the explanations is considerably different and some are less intuitive than others.
\end{example}
In particular,~\citet{kim2018interpretability} observe that when the intent is less ``structured'' (e.g., attributes represent pixels of an image) explanations can be quite difficult to grasp for humans. This is why~\citet{kim2018interpretability} propose to increase human understanding by providing explanations based on contexts where the individual attribute names are semantically meaningful and human-understandable (as it often happens in tabular data). For this reason, when the intent of the feature matrix is not human-understandable,~\citet{kim2018interpretability} propose to transform the original intent into a more semantically meaningful set of attributes where concepts and explanations are more intuitive for human observers. 
\begin{remark}
    For brevity and simplicity,~\citet{kim2018interpretability} refers to human-understandable attributes as ``high-level concepts'' or simply as ``concepts''. From now on we will follow this convention and we refer to ``human-understandable attributes'' as ``concepts'' highlighting the distinction with ``formal concepts'' when appropriate.
\end{remark}
The following definition will simplify our description in the next chapters.
\begin{definition}[High-level concept~\cite{kim2018interpretability}]
    A high-level concept (or simply ``concept'') is a human-understandable and semantically meaningful attribute name.
\end{definition}
In the next section we describe the general structure of AI agents providing explanations in human-understandable semantics based on high-level concepts.

\subsection{Concept-based models}
Concept-based models are explainable AI agents generating predictions using human-understandable concepts as input~\cite{kim2018interpretability,chen2020concept,koh2020concept}. Through the input concepts, concept-based models aim to increase human trust by allowing their users to trace back predictions directly to human-understandable concepts thus making the whole decision process of the AI agent more transparent~\cite{rudin2019stop,shen2022trust}. For instance, a concept-based model can make the prediction $\mc{Y}=\{x_{bird}\}$ using the concepts $x_{flies}$ and $x_{animal}$ allowing a human observer to verify that the set of concepts used to make the prediction matches their experience.

Concept-based models $f: \mathcal{C} \times \mc{P} \rightarrow \mathcal{Y}$ learn a map from a set of semantically meaningful concepts $\mc{C}$ to a set of tasks $\mathcal{Y}$~\cite{yeh2020completeness}. This way humans can interpret this mapping by tracing back predictions to the most relevant concepts~\cite{ghorbani2019interpretation}. When the features of the input space are hard for humans to reason about (such as pixel intensities), we may still apply concept-based models on the output of a ``concept-encoder'' i.e., a mapping $g: \mc{X} \times \mc{P}' \rightarrow \mc{C}$ from the input space $\mc{X}$ to the concept space $\mc{C}$~\cite{ghorbani2019towards,koh2020concept}. 
Using our categorical constructions we can formally describe a concept-based model as follows.
\begin{definition}[Concept encoder]
    A concept encoder is an AI agent $g: \mathcal{X} \times \mathcal{P}' \rightarrow \mathcal{C}$ where the output object $\mc{C}$ represents a set of concepts\footnote{Concepts in the sense of~\citet{kim2018interpretability}.}.
\end{definition}
\begin{definition}[Concept-based model]
    Given a concept encoder $g: \mathcal{X} \times \mathcal{P}' \rightarrow \mathcal{C}$, 
    a concept-based model is a XAI model where the explainer $f: \mathcal{C} \times \mathcal{P} \rightarrow  \mathcal{Y}\times \mathcal{E}$ takes as input object the set of concepts $\mathcal{C}$ generated by the concept encoder: 
    % \todo{fix tikz}
\[ \xaiCBM \]
\end{definition}
% % \stefano{Is XAI system defined?}
% % \paragraph{Agreement between explainers}
% This example demonstrates how category theory allows to reason about structures of structures more easily freeing XAI researchers from contingent details. 
Training a concept-based model may require a dataset 
% composed of tuples in $\mathcal{D} = (\mathcal{X} \times \mathcal{C} \times \mathcal{Y})\subseteq(X \times C \times Y)$,
where each sample consists of input features $\mathbf{x}\in \mathcal{X} \subseteq \mathbb{R}^n$ (e.g., an image's pixels), $k$ ground truth concepts $\mathbf{c}\in  \mathcal{C} \subseteq \{0, 1\}^k$ (i.e., a binary vector with concept annotations, when available) and $t$ task labels $\mathbf{y} \in  \mathcal{Y} \subseteq \{0, 1\}^t$ (e.g., an image's classes).
During training, a concept-based model is encouraged to align its predictions to task labels i.e., $\mathbf{y} \approx \mathbf{\hat{y}}=f(g(\mathbf{x}))$. Similarly, a concept encoder can be supervised when concept labels are available i.e., $\mathbf{c} \approx \mathbf{\hat{c}} = g(\mathbf{x})$. 
When concept labels are not available, unsupervised concept encoders extract concepts by associating concept labels to clusters found in the embeddings of pre-trained models as proposed by~\citet{ghorbani2019towards,magister2021gcexplainer}. We indicate concept and task predictions as $\hat{c}_i=(g(\mathbf{x}))_i$ and $\hat{y}_j=(f(\mathbf{\hat{c}}))_j$ respectively.


% \paragraph{Concept representations}
% Usually, concept-based models represent concepts using their truth degree, that is, $\hat{c}_1,\ldots,\hat{c}_k\in [0,1]$.
% % the concept encoder $g$ learns $k$ different scalar concept representations $\hat{c}_1,\ldots,\hat{c}_k\in [0,1]$, which represent the predicted truth degree of each concept. 
% However, this representation might significantly degrade task accuracy as observed by~\citet{mahinpei2021promises} and~\citet{zarlenga2022concept}. To overcome this issue, concept-based models may represent concepts using concept embeddings $\mathbf{\hat{c}}_i \in \mathbb{R}^m$ alongside their truth degrees $\hat{c}_i \in [0,1]$.\footnote{%For the sake of simplicity, we abuse notation and 
% With an abuse of notation, we use the same symbol for a concept embedding and its corresponding truth degree, with the former in bold to distinguish it.} While this increase task accuracy of concept-based models~\cite{zarlenga2022concept}, it also weaken their interpretability as concept embeddings lack clear semantics.

\begin{remark}
In the following chapters we will omit the dependency on parameters for morphisms as they are all parametric i.e., instead of writing $f: \mathcal{C} \times \mc{P} \rightarrow \mathcal{Y}$ we will simply write $f: \mathcal{C} \rightarrow \mathcal{Y}$.
\end{remark}

% more examples: \url{https://wikious.com/en/Formal_concept_analysis}

\section{Knowledge gaps and aims}
\label{sec:gaps-concept-learning}
We can summarize the ultimate aim of XAI research on concepts as follows: To design trustworthy AI systems able to attain state-of-the-art performance in solving complex tasks while providing human-understandable explanations for their decisions.
To this end, XAI research on concepts focuses on four main research areas: models, representations, metrics, and explanations. Research in concept models aims to improve the architectures of concept-based models and their concept encoders to increase the performance of these models in learning concepts from raw features and in learning the task labels from the learnt concepts. Research in concept representations focuses on devising more efficient data structures to encapsulate the information of learnt concepts preserving their semantics but allowing for concept encoders to incorporate sample-specific information about specific concept instances. Research in concept metrics aims to assess the quality of learnt concepts in terms of preserved semantics and their predictive information for task labels. Finally, research in concept explanations targets the design of signatures and the forms of the explanations provided by concept-based models in order to make them more trustworthy.

However, XAI research on concepts is a relatively young field and current approaches represent only the first steps towards the ultimate goal of the field. In fact, current approaches struggle either to attain state-of-the-art performances in solving complex tasks or to preserve a clean semantics in learnt concept representations. In addition, state-of-the-art concept-based systems either provide simple explanations in non-formal languages (e.g., Concept Activation Vectors~\cite{kim2018interpretability} or Concept Bottleneck Models~\cite{koh2020concept}), which may mislead human observers, or are not differentiable thus impeding a joint training with concept encoders to learn better concepts depending on the task (e.g., decision trees~\cite{breiman1984classification} or Bayesian rule lists~\cite{letham2015interpretable}). 
% Last but not least, differentiable concept-based models currently generate explanations as a by-product of their decision process. 
We can then summarize the main research directions in this field as follows:
\begin{enumerate}
    \item[\textbf{Aim \#1}]--- Generate compound explanations in formal languages with differentiable concept-based models;
    \item[\textbf{Aim \#2}]--- Attain state-of-the-art performance in solving complex tasks while preserving clean concept semantics;
    % \item[\textbf{Aim \#3}]--- Design interpretable concept-based models whose decision process is based on formal explanations.
\end{enumerate}

The following chapters address some of the main knowledge gaps currently arising in different areas of XAI concept research. In particular, Chapter~\ref{chapter:lens} focuses on \textbf{Aim \#1} presenting Logic Explained Networks (LENs), a family of differentiable concept-based models generating compound explanations in the formal language of first-order logic. Chapter~\ref{chapter:cem} focuses on \textbf{Aim \#2} introducing concept embedding representations which allow concept-based models to attain state-of-the-art performance in solving complex tasks while preserving clean concept semantics. While addressing \textbf{Aim \#2}, existing concept-based models are not designed for concept embeddings and are unable to provide formal and semantically meaningful explanations based on this concept representation. To solve this limitation, Chapter~\ref{chapter:DCR} presents the Deep Concept Reasoner (DCR), the first interpretable concept-based model using concept embeddings. In particular DCR represents the first differentiable concept-based model attaining state-of-the-art performance in solving complex tasks while providing human-understandable and formal explanations for its decisions, thus representing a concrete step towards efficient and trustworthy AI systems.


% \subsection{Concept completeness}


% \subsection{Concept alignment}
% The Concept Alignment Score (CAS) aims to measure how much learnt concept representations can be trusted as faithful representations of their ground truth concept labels. Intuitively, CAS generalises concept accuracy by considering the predictions' homogeneity within groups of similar samples. More specifically, CAS applies a clustering algorithm $\kappa$ to find $\rho > 2$ clusters, assigning to each sample $\mathbf{x}^{(j)}$, for each concept $c_i$, a cluster label $\pi_i^{(j)} \in \{1, \cdots, \rho\}$, using the concept representation $\hat{\textbf{c}}_i$. Given $N$ test samples, the homogeneity score $h(\cdot)$~\citep{rosenberg2007v} then computes the conditional entropy $H$ of ground truth labels $C_i = \{c_i^{(j)}\}_{j=1}^{N}$ w.r.t. cluster labels $\Pi_i = \{\pi_i^{(j)}\}_{j=1}^{N}$, i.e., $h = 1$ when $H(C_i,\Pi_i)=0$ and $h = 1 - H(C_i, \Pi_i)/H(C_i)$ otherwise. The higher the homogeneity, the more a learnt concept representation is ``aligned'' with its labels, and can thus be trusted as a faithful representation. CAS averages homogeneity scores over all concepts, providing a normalised score $\text{CAS} \in [0,1]$:
% \begin{equation}
%     \text{CAS}(\mathbf{\hat{c}}_1, \cdots, \mathbf{\hat{c}}_k) \triangleq \frac{1}{N - 2}\sum_{p=2}^N \Bigg(\frac{1}{k} \sum_{i=1}^k h(c_i, \kappa_p(\hat{\textbf{c}}_i)) \Bigg)
%     % \text{CAS}(\mathbf{\hat{c}}_1, \cdots, \mathbf{\hat{c}}_k) := \frac{1}{k(N-2)} \sum_{p=2}^N \sum_{i=1}^k h(c_i, \kappa(\ha  t{\textbf{c}}_i, p)))
% \end{equation}
% % Notice how when the number of clusters $p$ equals the number of samples, CAS and concept accuracy are identical.
% % The concept alignment score is therefore maximal (i.e., $\text{CAS} = 1$) when all clusters contain only data points which are members of a single concept class (i.e., for all samples within a cluster, the label of any concept $i$ is either always $c_i = True$ or always $c_i = False$). 
% To tractably compute CAS in practice, we sum homogeneity scores by varying $p$ across $p \in \{2, 2 + \delta, 2 + 2 \delta, \cdots, N\}$ for some $\delta > 1$ (details in Appendix). Furthermore, we use k-Medoids~\citep{kaufman1990partitioning} for cluster discovery, similarly to~\citet{ghorbani2019interpretation} and~\citet{magister2021gcexplainer}, and use concept logits when computing the CAS for Boolean and Fuzzy CBMs. For Hybrid CBMs, we use $\hat{\mathbf{c}}_i \triangleq [\hat{\mathbf{c}}_{[k:k + \gamma]}, \hat{\mathbf{c}}_{[i:(i + 1)]}]^T$ as the concept representation for $c_i$ (i.e., the extra capacity is a shared embedding across all concepts).
% % Cluster and concept labels are matched by homogeneity score.
% % and use a simple majority class count for labeling a cluster.

% \begin{itemize}
%     \item Accuracy vs. scalability/noise
%     \item Accuracy vs. explainability
%     \item Accuracy vs. interpretability
% \end{itemize}





% A recent leap in XAI research happened in 2018 when \citet{kim2018interpretability} started the field of concept-based XAI. Their argument was:

% \begin{displayquote}
% A key difficulty, however, is that most ML models operate on features, such as pixel values, that do not correspond to high-level concepts that humans easily understand. Furthermore, a model’s internal values (e.g., neural activations) can seem incomprehensible. We can express this difficulty mathematically, viewing the state of an ML model as a vector space $E_m$ spanned by basis vectors $e_m$ which correspond to data such as input features and neural activations. Humans work in a different vector space $E_h$ spanned by implicit vectors $e_h$ corresponding to an unknown set of human-interpretable concepts. [...] To address these challenges, we [...] provide an interpretation of a neural net's internal state in terms of human-friendly concepts.
% \end{displayquote}

% \section{Notation}
% \begin{itemize}
%     \item input space
%     \item output space
%     \item concept space
%     \item ground truth labels
%     \item predictions / representations
%     \item concept encoder
%     \item concept decoder / label predictor
% \end{itemize}

% \section{The Dark Side of Deep Learning}
% Standard intro w/ motivation for XAI: trade-off between accuracy and explainability: 
% \begin{itemize}
%     \item classical ML models are explainable, but they may have poor performances on some data
%     \item DL has high performances, but is less explainable generating a lack of human trust
% \end{itemize}

% Researchers are trying to: make explainable models more accurate and black boxes more interpretable


% \section{Concept Learning: A New Hope}
% A recent leap in XAI research happened in 2018 when \citet{kim2018interpretability} started the field of concept-based XAI. Their argument was:

% \begin{displayquote}
% A key difficulty, however, is that most ML models operate on features, such as pixel values, that do not correspond to high-level concepts that humans easily understand. Furthermore, a model’s internal values (e.g., neural activations) can seem incomprehensible. We can express this difficulty mathematically, viewing the state of an ML model as a vector space $E_m$ spanned by basis vectors $e_m$ which correspond to data such as input features and neural activations. Humans work in a different vector space $E_h$ spanned by implicit vectors $e_h$ corresponding to an unknown set of human-interpretable concepts. [...] To address these challenges, we [...] provide an interpretation of a neural net’s internal state in terms of human-friendly concepts.
% \end{displayquote}

% The motivation for this research line has roots in cognitive sciences (e.g., Representational Theory of the Mind). According to these theories \textbf{concepts are the basic building blocks of human thoughts}~\citep{margolis2007ontology}: following simple rules the human mind can combine finite stocks of basic concepts over and over again to create increasingly complex representations~\citep{margolis2007ontology}. For instance, the mind can combine the basic concepts ``roof'' and ``walls'' to generate the concept ``house''. 

% These concept theories motivated~\citet{kim2018interpretability} opening a research line on concept learning in ML to increase human trust in AI. The objective of this field is to increase human trust by making AI use ``the same building blocks of human thought'' as opposed to other XAI approaches~\citep{kim2018interpretability}. In particular concept-based XAI has the following aims:
% \begin{itemize}
%     \item discover concepts from trained models to explain their decisions in ``human'' terms (post-hoc explainability)
%     \item train models to learn specific concepts and compose them to solve tasks (self-explainability)
% \end{itemize} 
% Compared to other XAI approaches, concept-based XAI has solid foundations in cognitive sciences and a precise formalization in universal algebra.

% % concepts ontology (philosophy):

% % intuitively, what is a concept? formal concept is defined to be a pair (A, B), where A is a set of objects (called the extent) and B is a set of attributes (the intent) such that the extent A consists of all objects that share the attributes in B, and dually the intent B consists of all attributes shared by the objects in A.

% \section{Elements of Logic(s) and (Formal) Concept Theory}
% % \begin{definition}[Signature~\citep{goguen2005concept}]
% % A signature $\Sigma = (S_f, S_r, \text{ar})$ is a collection of:
% % \begin{itemize}
% %     \item a set of \textbf{function symbols} $S_f$
% %     \item a set of \textbf{relation symbols} (or predicates) $S_r$
% %     \item \textbf{a morphism} $\text{ar}: S_f \cup S_r \rightarrow \mathbb{N}$, which assigns a natural number called \textit{arity} to every function or relation symbol
% % \end{itemize}
% % \end{definition}

% Intuitively, what is a concept? It's (mental) representation describing objects sharing some common properties \citep{margolis2007ontology}. In the following we formalize this idea and provide some concrete examples.

% \subsection{Institutions and Abstract Formal Concept}

% Let $\mathbb{C}at^{op}$ represent the opposite of the category of small categories.
% \begin{definition}[Institution~\citep{goguen2005concept}]
% An institution $\mathbb{I}$ consists of:
% \begin{itemize}
%     \item an abstract category $\mathbb{S}ign$ of signatures
%     \item a functor $Sen: \mathbb{S}ign \rightarrow \mathbb{S}et$ giving for each signature $\Sigma$ the set of sentences $Sen(\Sigma)$, and for each signature morphism $\sigma: \Sigma \rightarrow \Sigma'$, the sentence translation map $Sen(\sigma): Sen(\Sigma) \rightarrow Sen(\Sigma')$
%     \item a functor $Mod: \mathbb{S}ign \rightarrow \mathbb{C}at^{op}$ giving for each signature $\Sigma$ the category of models $Mod(\Sigma)$, and for each signature morphism $\sigma: \Sigma \rightarrow \Sigma'$, the reduct functor $Mod(\sigma): Mod(\Sigma') \rightarrow Sen(\Sigma)$
%     \item a satisfaction relation $\models_\Sigma \subseteq |Mod(\Sigma)| \times Sen(\Sigma)$ for each signature $\Sigma \in \mathbb{S}ign$ such that for each signature morphism, the following satisfaction condition holds (``truth under context morphisms''): $M' \models_{\Sigma'} Sen(\sigma) \iff Mod(\sigma)(M') \models_\Sigma \phi$ for each $M' \in Mod(\Sigma')$ and $\phi \in Sen(\Sigma)$.
% \end{itemize}
% \end{definition}

% We say $M \models_\Sigma T$ where T is a set of $\Sigma$-sentences, if $M \models_\Sigma \phi$ for all $\phi \in T$, and we say $T \models_\Sigma \phi$ if for all $\Sigma$-models $M$, $M \models_\Sigma T$ implies $M \models_\Sigma \phi$.

% \begin{corollary}
% First order logic is an institution.
% \end{corollary}

% \begin{displayquote}
% Other logics follow a similar pattern, including modal logics, temporal logics, many sorted logics, equational logics, order sorted logics, description logics, higher order logics, etc. Database systems of various kinds are also institutions, where database states are contexts, queries are sentences, and answers are models.
% \end{displayquote}

% Given an institution $\mathbb{I}$, a theory is a pair $(\Sigma,T)$, where $T$ is a set of $\Sigma$-sentences. The collection of all $\Sigma$-theories can be given a lattice structure, under inclusion: $(\Sigma,T) \leq (\Sigma',T') \iff \Sigma \subseteq \Sigma' \wedge T \subseteq T'$.

% For any signature $\Sigma$ of an institution $I$, there is a Galois connection between its $\Sigma$-theories and its sets of $\Sigma$-models i.e., $(\Sigma,T)^\bullet = \{M \ | \ M \models_\Sigma T\}$, and if $\mathcal{M}$ is a collection of $\Sigma$-models, let $\mathcal{M}^\bullet = \{\phi \ | \ \mathcal{M} \ \models_\Sigma \phi\}$.

% \begin{definition}[(Abstract) Formal Concept~\citep{goguen2005concept}]
% An (abstract) formal concept of an institution $\mathbb{I}$ is a pair $(T,\mathcal{M})$ such that $T^\bullet = \mathcal{M}$ and $\mathcal{M}^\bullet = T$ (i.e., a closed theory).
% \end{definition}

% This is the most general definition of a concept embracing all current concept theories (e.g., John Sowa’s lattice of theories [61] (abbreviated LOT), the formal concept analysis (FCA) of Rudolf Wille [15], the information flow (IF) of Jon Barwise and Jerry Seligman [3], Gilles Fauconnier’s logic-based mental spaces [13], Peter Gardenfors geometry-based conceptual spaces [16], the conceptual integration (CI, also called blending) of Fauconnier and Turner, etc.). However, to make the notion of concept more concrete and directly applicable to standard ML language we will define a ``concrete'' formal concept in the (standard) institution of First Order Logic (FOL) and in the context of ML.


% \subsection{Concrete Formal Concepts and First Order Logic}

% In a more concrete setting, we will consider in the following entities in the institution of First Order Logic (FOL):
% \begin{itemize}
%     \item the function symbol $g$ (also known as \textbf{concept encoder})
%     \item the function symbol $f$ (also known as \textbf{concept decoder} or \textbf{label predictor})
%     \item the set of \textbf{formal objects} $G$ (FOL models)
%     \item the set of \textbf{formal attributes} $M$ (FOL sentences)
%     \item the binary relation between formal objects and attributes $I \subseteq G \times M$
% \end{itemize}


%% %%%%%%%%%%%%%%%%%%%%%%%%%%%%%%%%%%%%%%%%%%%%%%%%%%%%%%%%%%%%%%%%%%%%%%%%%%%%%%%%
%% %% Concept Quality:
%% %%
%% \chapter{Concept Quality} \label{chapter:metrics}
% \textbf{Research: completed. Status: drafted. Difficulty: low. Priority: low.}

% \textit{In this chapter I will discuss how to measure the quality of concept representations. In particular I will focus on my contribution in inventing the niche impurity score~\citep{zarlenga2021quality} which generalizes concept completeness~\citep{yeh2020completeness} to concept subsets. I will demonstrate how this metric is computationally efficient and does not require concept labels thus making it applicable in real-world supervised and unsupervised scenarios. I will conclude the chapter with experiments showing how the niche impurity score can be used in practice to evaluate the robustness of concept representations generated by state-of-the-art supervised and unsupervised concept learning methods.}

\textbf{Motivation---} Very few metrics available to assess concept quality. Hard to understand whether to trust concept-based models explanations based on learnt concepts.

\textbf{Solution---} Two new metrics to assess concept quality and robustness.

The \textbf{key innovation} consists in generalizing existing metrics to subset of concepts (niching) and to concept embeddings (alignment).

\section{Concept completeness}


\section{Concept niches and concept impurity}
\begin{definition}[Concept nicher] \label{def:nicher}
Given a set of concept representations $\hat{C} \subseteq \mathbb{R}^{d \times k}$, we define a concept nicher as a function $\nu: \{1, \cdots k\} \times \{1, \cdots k\} \mapsto [0, 1]$ that returns $\nu(i, j) \approx 1$ if the $i$-th concept $\mathbf{\hat{c}}_{(:, i)}$ is entangled with the $j$-th ground truth concept $c_j$, and $\nu(i, j) \approx 0$ otherwise.
\end{definition}

Our definition above can be instantiated in various ways, depending on how entanglement is measured. In favour of efficiency, we measure entanglement using absolute Pearson correlation $\rho$, as this measure can efficiently discover (a linear form of) association between variables~\cite{altman2015points}. We call this instantiation  \emph{concept-correlation nicher} (CCorrN) and define it as
$\text{CCorrN}(i, j) := \big| \rho\big(\{\mathbf{\hat{c}}^{(l)}_{(:, i)}\}_{l=1}^N, \{c^{(l)}_j\}_{l=1}^N\big) \big|$.

% The above definition is affected by how entanglement is defined. One efficient way of measuring the entanglement is to use the absolute Pearson correlation, denoted as $\rho$. We call such an instantiation a \emph{concept-correlation nicher} (CCorrN) and define it as:
% \[
%     \text{CCorrN}(i, j) := \big| \rho\big(\{\mathbf{\hat{c}}^{(l)}_{(:, i)}\}_{l=1}^N, \{\mathbf{\hat{c}}^{(l)}_j\}_{l=1}^N\big) \big|
% \]
If $\mathbf{\hat{c}}_{(:, i)}$ is not a scalar representation (i.e., $d > 1$), then for simplicity we use the maximum absolute correlation coefficient between all entries in $\mathbf{\hat{c}}_{(:, i)}$, and the target concept label $c_j$ as a representative correlation coefficient for the entire representation $\mathbf{\hat{c}}_{(:, i)}$. We then define a concept niche as: 
\begin{definition}[Concept niche]
The concept niche $N_j(\nu, \beta)$ for target concept $j$, determined by concept nicher $\nu(\cdot, \cdot)$ and threshold $\beta \in [0,1]$, is defined as $N_j(\nu, \beta) := \big\{i \ \ | \ \ i \in \{1, \cdots, k\} \text{ and } \nu(i, j) > \beta \big\}$.
\end{definition}

From this, the Niche Impurity (NI) measures the predictive capacity of the complement of concept niche $N_i(\nu, \beta)$, referred to as $\neg N_i(\nu, \beta) := \{1, \cdots, k\} \; \backslash \; N_i(\nu, \beta)$, for the $i$-th ground truth concept:
%Given the complement of concept niche $N_i(\nu, \beta)$, which we refer to as $\neg N_i(\nu, \beta) := \{1, \cdots, k\} \; \backslash \; N_j(\nu, \beta)$, the Niche Impurity (NI) measures its predictive capacity for the $i$-th ground truth concept.

\begin{definition}[Niche Impurity (NI)] \label{def:niche_impurity}
Given a classifier $f: \hat{C} \mapsto C$, concept nicher $\nu$, threshold $\beta \in [0, 1]$, and labeled concept representations $\{(\mathbf{\hat{c}}^{(l)}, \mathbf{c}^{(l)})\}_{l = 1}^n$, the Niche Impurity of the $i$-th output of $f(\cdot)$ is defined as $\text{NI}_i(f, \nu, \beta) := \text{AUC} \big( \{( f|_{\neg N_i(\nu, \beta)} \big( \mathbf{\hat{c}}^{(l)}_{(:, \neg N_i(\nu, \beta))} \big), c^{(l)}_i) \}_{l=1}^n \big)$, where $f|_{\neg N_j(\nu, \beta)}$
% : \hat{C} \mapsto C$
is the classifier resulting from masking all entries in $\neg N_j(\nu, \beta)$ when feeding $f$ with concept representations. 
\end{definition}

Although $f$ can be any classifier, in our experiments we use a ReLU MLP with hidden layer sizes $\{ 20, 20 \}$.
Intuitively, a NI of $1/2$ (random AUC of niche complement) indicates that the concepts inside the niche $N_i(\nu)$ are the only concepts predictive of the $i$-th concept, that is, concepts outside the niche do not hold any predictive information of the $i$-th concept.
% In contrast, a NI of $1$ suggests that concepts outside the nice $N_i(\nu)$ are still fully predictive of concept $i$.
Finally, the \textit{Niche Impurity Score} metric measures how much information apparently disentangled concepts
% (target concepts and their niche complements)
are actually sharing:

\begin{definition}[Niche Impurity Score (NIS)] \label{def:niche_impurity_score}
Given a classifier $f: \hat{C} \mapsto C$ and concept nicher $\nu$, the niche impurity score $\text{NIS}(f,\nu) \in [0,1]$ is defined as the summation of niche impurities across all concepts for different values of $\beta$: $\text{NIS}(f,\nu) := \int_{0}^{1} (\sum_{i=1}^{k} \text{NI}_i(f, \nu, \beta)/k) d\beta$.
\end{definition}

In practice, we estimate this integral using the trapezoid method with values in $\beta \in \{ 0.0, 0.05, \cdots, 1\}$. For efficiency, we parameterise $f$ as an MLP,
% one can very efficiently compute the NI for different concepts and values of $\beta$,
leading to a tractable impurity metric which easily scales as the number of concepts $k$ increases. Intuitively, a NIS of $1$ means that all the information to perfectly predict each ground truth concept is spread on many different and disentangled concept representations. In contrast, a NIS around $1/2$ (random AUC) indicates that no concept can be predicted by any concept representation subset.
% Intuitively, a NIS score of $1$ conveys perfect purity and means that the set of learnt concepts can be divided into disentangled sets, each of which is related to predicting a single ground truth concept, while the NIS score around $1/2$ conveys maximum impurity and indicates that information related to each ground truth concept is scattered across all assumed disentangled sets of learnt concepts.

\section{Concept alignment}
The Concept Alignment Score (CAS) aims to measure how much learnt concept representations can be trusted as faithful representations of their ground truth concept labels. Intuitively, CAS generalises concept accuracy by considering the predictions' homogeneity within groups of similar samples. More specifically, CAS applies a clustering algorithm $\kappa$ to find $\rho > 2$ clusters, assigning to each sample $\mathbf{x}^{(j)}$, for each concept $c_i$, a cluster label $\pi_i^{(j)} \in \{1, \cdots, \rho\}$, using the concept representation $\hat{\textbf{c}}_i$. Given $N$ test samples, the homogeneity score $h(\cdot)$~\citep{rosenberg2007v} then computes the conditional entropy $H$ of ground truth labels $C_i = \{c_i^{(j)}\}_{j=1}^{N}$ w.r.t. cluster labels $\Pi_i = \{\pi_i^{(j)}\}_{j=1}^{N}$, i.e., $h = 1$ when $H(C_i,\Pi_i)=0$ and $h = 1 - H(C_i, \Pi_i)/H(C_i)$ otherwise. The higher the homogeneity, the more a learnt concept representation is ``aligned'' with its labels, and can thus be trusted as a faithful representation. CAS averages homogeneity scores over all concepts, providing a normalised score $\text{CAS} \in [0,1]$:
\begin{equation}
    \text{CAS}(\mathbf{\hat{c}}_1, \cdots, \mathbf{\hat{c}}_k) \triangleq \frac{1}{N - 2}\sum_{p=2}^N \Bigg(\frac{1}{k} \sum_{i=1}^k h(c_i, \kappa_p(\hat{\textbf{c}}_i)) \Bigg)
    % \text{CAS}(\mathbf{\hat{c}}_1, \cdots, \mathbf{\hat{c}}_k) := \frac{1}{k(N-2)} \sum_{p=2}^N \sum_{i=1}^k h(c_i, \kappa(\ha  t{\textbf{c}}_i, p)))
\end{equation}
% Notice how when the number of clusters $p$ equals the number of samples, CAS and concept accuracy are identical.
% The concept alignment score is therefore maximal (i.e., $\text{CAS} = 1$) when all clusters contain only data points which are members of a single concept class (i.e., for all samples within a cluster, the label of any concept $i$ is either always $c_i = True$ or always $c_i = False$). 
To tractably compute CAS in practice, we sum homogeneity scores by varying $p$ across $p \in \{2, 2 + \delta, 2 + 2 \delta, \cdots, N\}$ for some $\delta > 1$ (details in Appendix). Furthermore, we use k-Medoids~\citep{kaufman1990partitioning} for cluster discovery, similarly to~\citet{ghorbani2019interpretation} and~\citet{magister2021gcexplainer}, and use concept logits when computing the CAS for Boolean and Fuzzy CBMs. For Hybrid CBMs, we use $\hat{\mathbf{c}}_i \triangleq [\hat{\mathbf{c}}_{[k:k + \gamma]}, \hat{\mathbf{c}}_{[i:(i + 1)]}]^T$ as the concept representation for $c_i$ (i.e., the extra capacity is a shared embedding across all concepts).
% Cluster and concept labels are matched by homogeneity score.
% and use a simple majority class count for labeling a cluster.

\section{Trade-offs in concept learning}
\begin{itemize}
    \item Accuracy vs. scalability/noise
    \item Accuracy vs. explainability
    \item Accuracy vs. interpretability
\end{itemize}


\section*{Papers}
\nobibliography*
\begin{itemize}
    \item \bibentry{zarlenga2021quality}
\end{itemize}


%%%%%%%%%%%%%%%%%%%%%%%%%%%%%%%%%%%%%%%%%%%%%%%%%%%%%%%%%%%%%%%%%%%%%%%%%%%%%%%%
%% Concepts for logic explanations:
%%
\chapter{Concept-based Logic Explanations} 
\label{chapter:lens}
\textbf{Research: completed. Status: drafted. Difficulty: low. Priority: low.}

\textit{In this chapter I will explain how concepts can be used to train more interpretable models. In particular, I will focus on my contributions in inventing Logic Explained Networks (LENs)~\citep{ciravegna2021logic}, a family of concept-based models providing first-order logic explanations for their predictions. I will describe LENs main architectures, learning paradigms, and logic rule extraction algorithms~\citep{barbiero2021entropy}. I will conclude the chapter showcasing LENs on a set of synthetic and real-world experiments demonstrating how LENs can provide highly accurate explanations with classification performances close to state-of-the-art models.}


Summary to this point: to solve the problem of human trust in AI we need reliable models that can be accurate and provide explanations for their decisions, we want to know how! In this chapter I will explain how to design self-explaining models providing global and quantitative explanations for their predictions.

\section{Motivation}

Knowledge gap/Motivation: current solutions provide post-hoc, local, and qualitative explanations. Post-hoc is bad because the damage is done: whatever biases the model learnt during training can be identified but not changed. Local is bad because local explanations are brittle: explain a cherry-picked example or an outlier does not generalize. Qualitative is bad because it's not clear how general are explanations: would they hold on unseen samples? for how many samples would they hold? how to compare XAI models? Hence we want: self-explainable models providing global and quantitative explanations. Concept bottleneck models tried to solve the post-hoc problem architecturally/by design: they propose an architecture where models' predictions are conditioned on concepts. This way: predictions can be explained in terms of concepts being active/inactive. However, explanations are still qualitative e.g., this concept is important, this one is less important. Is it possible to formalize the explanation in a way that is not ambiguous and shows how the model used the concepts? Also, explanations are still local: for this sample this concept A is active, concept B is inactive, etc. How to provide global explanations?

Contribution: a self-explainable model providing global and quantitative explanations.

Key innovation: a sparse attention mechanism for concept bottleneck models.

Expected outcome: The sparse attention allows the model to learn how to cherry-pick the most relevant concepts for each task and use only them for solving the task. This way the model learns how to solve the task using only a few concepts which can be active/inactive. This allows the extraction of logic explanations as the model is learning a logic mapping from (fuzzy) concepts to (classification) tasks.The logic explanations are simple thanks to the sparse attention mechanism. Without the attention mechanism the extraction of logic rules won't scale as the truth tables may explode (their size grows as exponentially with the number of concepts). Logic explanations can also be evaluated with quantitative metrics in terms of performance/interpretability: 

- performance: prediction accuracy (using the logic formulae to predict unseen samples)

- interpretability: complexity (the number of terms in a logic formula)

Research questions: how would this approach compare with existing approaches? is it interpretable as white boxes in terms of rule complexity? is it accurate as black boxes? what's the accuracy-vs-explainability trade-off of the approach? is there any advantage of using this approach over existing methods?


\section{Logic Explained Networks}

\subsection{Generate Concept-based Logic Explanations}
Any Boolean function can be converted into a logic formula in Disjunctive Normal Form (DNF) by means of its truth-table \citep{mendelson2009introduction}. 
% We indicate with $\hat{f}^i$ the Boolean function represented by the truth table $\mathcal{T}^i$, $\hat{f}^i: \hat{C}^i \mapsto Y^i$, being $Y^i$ the $i$-th component of $Y$.
%\sm{Following the notation defined at the end of Sec.~\ref{sec:con_awa}}, any $\bar{f}^i$ is a Boolean function over the set $\hat{C}^i$. %We denote with $\varphi_i$ the logic formula corresponding to the truth-table $\mathcal{T}^i$ of $\bar{f}^i$. 
Converting a truth table into a DNF formula provides an effective mechanism to extract logic rules of increasing complexity from individual observations
% , for cluster of samples,
to a whole class of samples. 
% In order to expliciteply write down the syntactic logic formula corresponding to any boolean function, we will use text strings in quotation marks corresponding to both concept and task symbols.
% The following steps are repeated for any task function $f^i$.
% [TODO] In the following, with a little abuse of notation, we will denote by $\bar{c}_j,\neg\bar{c}_j$ and $\bar{f}^i,\neg\bar{f}^i$ both the Boolean values and the human-understandable concept and task name and their negated, respectively, for every $j,i$. 
The following rule extraction mechanism is applied to any empirical truth table $\mathcal{T}^i$ for each task $i$.
% \begin{figure}[h]
%     \centering
%     \includegraphics[width=1\textwidth]{figs/truth-table.png}
%     \caption{Caption}
%     \label{fig:tt2form}
% \end{figure}

% %\vspace{-3mm}
\paragraph{FOL extraction.}
Each row of the truth table $\mathcal{T}^i$ can be partitioned into two parts that are a tuple of binary concept activations, $\hat{q}\in \hat{C}^i$, and the outcome of $\bar{f}^i(\hat{q}) \in \{0, 1\}$. 
An \textit{example-level} logic formula, consisting in a single minterm, can be trivially extracted from each row for which $\bar{f}^i(\hat{q})=1$, by simply connecting with the logic AND ($\wedge$) the true concepts and negated instances of the false ones. 
The logic formula becomes human understandable whenever concepts appearing in such a formula are replaced with human-interpretable strings that represent their name (similar consideration holds for $\bar{f}^i$, in what follows). For example, the following logic formula $\varphi^i_t$,
\begin{equation}
    \varphi^i_{t} = \textbf{c}_1\wedge\ \neg \textbf{c}_2 \wedge \ldots\wedge\textbf{c}_{m_i},
    \label{eq:locexp}
\end{equation}
is the formula extracted from the $t$-th row of the table where, in the considered example, only the second concept is false, being $\textbf{c}_z$ the name of the $z$-th concept.
Example-level formulas can be aggregated with the logic OR ($\vee$) to provide a \textit{class-level} formula, 
\begin{equation}
    \displaystyle\bigvee_{t \in S_i}\varphi^i_t, %= \displaystyle\bigvee_{\hat{c} \in S_i}\textbf{c}_1\wedge\ldots\wedge\textbf{c}_{m_i}
\label{eq:agg_exp}
\end{equation}
being $S_i$ the set of rows
 indices of $\mathcal{T}^i$ for which $\bar{f}^i(\hat{q}) = 1$, i.e. it is the support of $\bar{f}^i$.
%Once $\varphi^i$ is built, given a concept tuple $\bar{c} \in \hat{C}^i$, we may measure the satisfiability of $\varphi^i$ on $\bar{c}$ by the truth-value of $\varphi^i(\bar{c})$ obtained by replacing boolean values with concept names in Eq. \ref{eq:agg_exp}. 
%By construction the support of the class-level rule and of $\hat{f}^i$ coincide.
We define with $\phi^i(\hat{c})$ the function that holds true whenever Eq.~\ref{eq:agg_exp}, evaluated on a given Boolean tuple $\hat{c}$, is true.
Due to the aforementioned definition of support, we get the following class-level First-Order Logic (FOL) explanation for all the concept tuples,
\begin{equation}
\forall \hat{c} \in \hat{C}^i:\ \phi^i(\hat{c})\leftrightarrow\bar{f}^i(\hat{c}).
\label{eq:FOL_C}
\end{equation}
We note that in case of non-concept-like input features, we may still derive the FOL formula through the ``concept decoder'' function $g$ (see Sec. \ref{sec:background}),
\begin{equation}
\forall x \in X:\ \phi^i\left(\xi(\overline{g(x)},\mu^i)\right)\leftrightarrow\bar{f}^i\left(\xi(\overline{g(x)},\mu^i)\right)
\end{equation}
An example of the above scheme for both example and class-level explanations is depicted on top-right of Fig. \ref{fig:awareness}.

%\vspace{-3mm}
\paragraph{Remarks.} The aggregation of many example-level explanations may increase the length and the complexity of the FOL formula being extracted for a whole class. However, existing techniques as the Quine–McCluskey algorithm can be used to get compact and simplified equivalent FOL expressions \citep{mccoll1878calculus,quine1952problem,mccluskey1956minimization}. For instance, the explanation (\textit{person} $\wedge$ \textit{nose}) $\vee$ ($\neg$\textit{person} $\wedge$ \textit{nose}) can be formally simplified in \textit{nose}.
%\fg{mettere o omettere questo paragrafo?}
%\paragraph{Remarks.}
Moreover, the Boolean interpretation of concept tuples may generate colliding representations for different samples. For instance, the Boolean representation of the two samples $\{ (0.1, 0.7), (0.2, 0.9) \}$ is the tuple $\bar{c} = (0, 1)$ for both of them. This means that their example-level explanations match as well. %It is worth to notice that formally Eq. \ref{eq:FOL_C} relies on the fact that there are no two different concept tuples $c,d$ with $\bar{c}=\bar{d}$ such that $f^i(c)\neq f^i(d)$ for some $i$. 
However, a concept can be eventually split into multiple finer grain concepts to avoid collisions. Finally, we mention that the number of samples for which any example-level formula holds (i.e. the support of the formula) is used as a measure of the explanation importance. In practice, example-level formulas are ranked by support and iteratively aggregated to extract class-level explanations, until the aggregation improves the accuracy of the explanation over a validation set.
% For very large datasets, it may be useful to consider only the example-level formulas with the highest importance rate in order to drop outlier explanations out and to get simpler more focused explanations. 

\subsection{The Entropy Layer}
When humans compare a set of hypotheses outlining the same outcomes, they tend to have an implicit bias towards the simplest ones as outlined in philosophy \citep{soklakov2002occam,rathmanner2011philosophical},
% aristotlePosterior, %hoffmann1996ockham,
psychology \citep{miller1956magical,cowan2001magical}, and decision making \citep{simon1956rational,simon1957models,simon1979rational}.
% , information theory \citep{mackay2003information}, and natural sciences \citep{wiley2011phylogenetics,ma2014changing}. 
The proposed entropy-based approach encodes this inductive bias in an end-to-end differentiable model. The purpose of the entropy-based linear layer is to encourage the neural model to pick a limited subset of input concepts, allowing it to provide concise explanations of its predictions. The learnable parameters of the layer are the usual weight matrix $W$ and bias vector $b$. In the following, the forward pass is described by the operations going from Eq. \ref{eq:gamma} to Eq. \ref{eq:forward}
while % and 
the generation of the truth tables from which explanations are extracted 
is formalized by % in 
Eq. \ref{eq:sparse} and Eq. \ref{eq:truth-table}.

% \begin{figure}[t]
%     \centering
%     \includegraphics[width=0.65\columnwidth]{LaTeX/figs/awareness_only_layer.pdf}
%     \caption{A detailed view on one ``head'' of
%     the entropy-based linear layer for the $1$-st class, emphasizing the role of the $k$-th input concept as example: (i) the scalar $\gamma_k^1$ (Eq.~\ref{eq:gamma}) is computed from the 
%     set of weights connecting the $k$-th input concept 
%     to the output neurons of the entropy-based layer;
%     % $weight vector $W_k^1$, the weights associated to the $k$-th concept (Eq. \ref{eq:gamma}); 
%     (ii) the relative importance of each concept is summarized by the categorical distribution $\alpha^1$ (Eq. \ref{eq:alpha}); (iii) rescaled relevance scores $\tilde{\alpha}^1$ drop irrelevant input concepts out (Eq. \ref{eq:drop}); (iv) hidden states $h^1$ (Eq. \ref{eq:forward}) and Boolean-like concepts $\hat{c}^1$ (Eq. \ref{eq:sparse}) are provided as outputs of the entropy-based layer.}
%     \label{fig:awareness2}
% \end{figure}

The relevance of each input concept can be summarized in a first approximation by a measure that depends on the values of the weights connecting such concept to the upper network. In the case of network $f^i$ (i.e. predicting the $i$-th class) and of the $j$-th input concept, we indicate with $W_j^i$ the vector of weights departing from the $j$-th input (see Fig. \ref{fig:awareness2}), and we introduce
%For the network $f^i$ (i.e. predicting the $i$-th class) the relevance of the $j$-th input concept is summarized in first approximation by the vector $W_j^i$ representing the weights connecting the $j$-th concept with the first layer of hidden neurons (see Fig \ref{fig:awareness}):
\begin{equation} \label{eq:gamma}
    \gamma^i_j = ||W^i_j||_1\ .
\end{equation}
The higher $\gamma^i_j$, the higher the relevance of the concept $j$ for the network $f^i$. In the limit case ($\gamma_j^i \rightarrow 0$) the model $f^i$ drops the $j$-th concept out.
% Notice that since the vector $\gamma^i$ is computed for each class, hidden network layers are not shared among network outputs but are independent as shown in Fig \ref{fig:awareness}.
To select only few relevant concepts for each target class, concepts are set up to compete against each other. To this aim, the relative importance of each concept to the $i$-th class is summarized in the categorical distribution 
$\alpha^{i}$, composed of coefficients
$\alpha^i_j \in [0,1]$ (with $\sum_j \alpha_j^i = 1$), modeled by the softmax function:
\begin{equation} \label{eq:alpha}
    \alpha^i_j = \frac{e^{\gamma^i_j/\tau}}{\sum_{l=1}^k e^{\gamma^i_l/\tau}}
\end{equation}
where $\tau \in \mathbb{R}^+$ is a user-defined temperature parameter to tune
% the intrinsic tendency of 
the softmax function. For a given set of $\gamma^i_j$, when using high temperature values ($\tau \rightarrow \infty$) all concepts have nearly the same relevance. For low temperatures values ($\tau \rightarrow 0$), the probability of the most relevant concept tends to $\alpha_j^i\approx 1$, while it becomes $\alpha_k^i\approx 0, \ k \neq j$, for all other concepts. For further details on the impact of $\tau$ on the model predictions and explanations (see Appendix).
As the probability distribution $\alpha^i$ highlights the most relevant concepts, this information is directly fed back to the input, weighting concepts by the estimated importance. To avoid numerical cancellation due to values in $\alpha^i$ close to zero, especially when the input dimensionality is large,
we replace $\alpha^i$ with its normalized instance $\tilde{\alpha}^i$, still  in $[0,1]^k$,
and each input sample % represented by the concept tuple ù
$c \in C$ is modulated by this %(normalized) 
estimated importance, % weighted by the re-normalized vector $\hat{\alpha}^i \in [0,1]^k$:w
\begin{equation} \label{eq:drop}
    \tilde{c}^i = c \odot \tilde{\alpha}^i \qquad\qquad \text{with} \qquad \tilde{\alpha}_j^i = \frac{\alpha_j^i}{\max_u \alpha_u^i},
\end{equation}
where $\odot$ denotes the Hadamard (element-wise) product.
The highest value in $\tilde{\alpha}^i$ is always $1$ (i.e. $\max_j \tilde{\alpha}_j^i = 1$) and it corresponds to the most relevant concept. 
% All the other concepts are weighted by an $\hat{\alpha}_j^i \leq 1$. 
The embeddings $h^i$ are computed as in any linear layer by means of the affine transformation:
\begin{equation} \label{eq:forward}
    h^i = W^i \tilde{c}^i + b^i.
\end{equation}
Whenever $\tilde{\alpha}_j^i \rightarrow 0$, the input $\tilde{c}_j^i \rightarrow 0$.
%In the limit, as $\hat{\alpha}_j^i \rightarrow 0$, the input $\tilde{c}_j \rightarrow 0$. 
This means that the corresponding concept tends to be dropped out and the network $f^i$ will learn to predict the $i$-th class without 
relying on % using 
the $j$-th concept. 

In order to % To 
get logic explanations, the proposed linear layer generates the truth table $\mathcal{T}^i$ formally representing the behaviour of the neural network 
in terms of Boolean-like representations of the input concepts. % for each category employed as classification objective. 
In detail, we indicate with $\bar{c}$ the Boolean interpretation of the input tuple $c \in C$, while $\mu^i \in \{0,1\}^k$ is the binary mask associated to $\tilde{\alpha}^i$.
To encode the inductive human bias towards simple explanations \citep{miller1956magical,cowan2001magical,ma2014changing}, the 
mask % vector
% $\mu^i \in [0,1]^k$ 
$\mu^i$
% is computed and 
is used to generate the 
binary % one-hot 
concept tuple $\hat{c}^i$, 
dropping % masking 
the least relevant concepts out of $c$,
\begin{equation}\label{eq:sparse}
    \hat{c}^i = \xi(\bar{c}, \mu^i)  \quad \text{with} \quad
    \mu^i = \mathbb{I}_{\tilde{\alpha}^i \geq \epsilon} \quad \text{and} \quad \bar{c} = \mathbb{I}_{c \geq \epsilon},
\end{equation}
where $\mathbb{I}_{z \geq \epsilon}$ denotes the indicator function that is $1$ for all the components of vector $z$ being $\geq \epsilon$ and $0$ otherwise (considering the unbiased case, we set $\epsilon=0.5$).
The function $\xi$ returns the vector with the components of $\bar{c}$ that correspond to $1$'s in $\mu^i$ (i.e. it sub-selects the data in $\bar{c}$).
%while $\xi$ sub-selects the concepts for which $\mu^i_j=1$.
As a results, $\hat{c}^i$ belongs to a space $\hat{C}^i$ of $m_i$ Boolean features, with $m_i < k$ due to the effects of the subselection procedure.
%Given a dataset $\mathcal{D}=(\mathcal{C},\mathcal{Y})$, sparse concept representations $\hat{c}^i$ are obtained for each observation from Eq. \ref{eq:sparse} and stacked together in the sparse matrix $\hat{\mathcal{C}}^i$.
%= [\hat{c}^i(1), ...., \hat{c}^i(m)]$. 
%This matrix is concatenated with the Boolean\sm{-interpreted} model predictions $\bar{f}^i = \mathbb{I}_{f^i \geq 0.5}$ to obtain the matrix $\mathcal{T}^i$ corresponding to the sparse truth table used to generate logic explanations (see Sec. \ref{sec:fol}):

The truth table $\mathcal{T}^i$ is a particular way of representing the behaviour of network $f^i$ based on the outcomes of
 processing multiple input samples collected in a generic dataset $\mathcal{C}$.
 As the truth table involves Boolean data, we denote with 
$\hat{\mathcal{C}}^i$ the set with the Boolean-like representations of the samples in $\mathcal{C}$ computed by $\xi$, Eq.~\ref{eq:sparse}.
We also introduce $\bar{f}^i(c)$ as the Boolean-like representation of the network output, $\bar{f}^i(c)=\mathbb{I}_{f^i(c)\geq \epsilon}$.
%old
% From an operational perspective, the contents $\mathbf{T}^i$ of the truth table $\mathcal{T}^i$ are obtained by stacking data of $\hat{\mathcal{C}}^i$ into a 2D matrix $\hat{\mathbf{C}}^i$ (row-wise), and concatenating the result with the column vector $\bar{\mathbf{f}}^i$ whose elements are $\bar{f}^i(c)$, $c\in \mathcal{C}$, that we summarize as
% \begin{equation} \label{eq:truth-table}
%     \mathbf{T}^i = \Big( \hat{\mathbf{C}}^i \ \Big|\Big| \ \bar{\mathbf{f}}^i \Big).
% \end{equation}
The truth table $\mathcal{T}^i$ is obtained by stacking data of $\hat{\mathcal{C}}^i$ into a 2D matrix $\hat{\mathbf{C}}^i$ (row-wise), and concatenating the result with the column vector $\bar{\mathbf{f}}^i$ whose elements are $\bar{f}^i(c)$, $c\in \mathcal{C}$, that we summarize as
\begin{equation} \label{eq:truth-table}
    \mathcal{T}^i = \Big( \hat{\mathbf{C}}^i \ \Big|\Big| \ \bar{\mathbf{f}}^i \Big).
\end{equation}
To be precise, any $\mathcal{T}^i$ is more like an empirical truth table than a classic one corresponding to an $n$-ary boolean function, indeed $\mathcal{T}^i$ can have repeated rows and missing Boolean tuple entries. However, $\mathcal{T}^i$ can be used to generate logic explanations in the same way, as we will explain in Sec. \ref{sec:fol}.
% The purpose of the probability distribution $\alpha$ is threefold: (i) it models the relative importance of concepts enabling the inspection of the most relevant ones for each classification task; (ii) it 
% \begin{equation}
%     \hat{\alpha}^i = \frac{\alpha^i}{\max \alpha^i}
% \end{equation}
% At training epoch $t$, therefore, the weighted concepts $\tilde{C}^i$ will be the input of the neural network $\hat{y}^i = f^i \big( \tilde{C}^i \big)$. 
% \fg{Ma il concept awareness non fa già parte della f?}
% Concept awareness scores close to zero will drop out the least relevant input concepts allowing for simpler logic-based explanations of the neural network's decisions, as explained in Sec. \ref{sec:fol}. 
% More precisely, only input concepts which satisfy the following condition are considered:
% \begin{equation}
%     E_i =  \left\{\langle c_j \rangle  \mid \frac{\alpha^i_j}{\max_j{\alpha^i}} \textgreater 0.5 \right\}, 
% \end{equation}
% where $E_i$ denotes the set of input concepts employed to explain the output $f_i$.
% \subsubsection{Multi-task awareness}


% \begin{figure}[h]
%     \centering
%     \includegraphics[width=0.6\textwidth]{figs/logic_layers.pdf}
%     \caption{Multi-Head awareness consists of several awareness layers running in parallel. GC: CAMBIARE NOME DEL LAYER NELLA FIGURA.}
%     \label{fig:multi_head}
% \end{figure}

% \subsubsection{Multi-Head awareness}


\subsection{Loss Function}
The entropy of the probability distribution $\alpha^i$ (Eq. \ref{eq:alpha}),
\begin{equation}
    \mathcal{H}(\alpha^i) = - \sum_{j=1}^k \alpha^i_j \log \alpha^i_j
    \label{eq:ent}
\end{equation}
is minimized when a single $\alpha^i_j$ is one, thus representing the extreme case in which only one concept matters, while it is maximum when all concepts are equally important. When $\mathcal{H}$ is jointly minimized with the usual loss function for supervised learning $L(f,y)$ (being $y$ the target labels--we used the cross-entropy in our experiments), it allows the model to find a trade off between fitting quality and a parsimonious activation of the concepts, 
allowing each network $f^i$ to predict $i$-th class memberships using few relevant concepts only.
%The minimum of this function corresponds to a one-hot encoded configuration of $\alpha$. When $\alpha_j^i=0$ the $j$-th concept is not taken into consideration when predicting the $i$-th class, as described in Eq. \ref{eq:drop}.
%A cross entropy loss $L(f,y)$ is also employed on the available labels $y$ for standard supervised training.
%\sm{We define with $L(f,y)$ the cross-entropy loss for supervised data, being $y$ the target labels}.
Overall, the loss function to train the network $f$ is defined as,
% The overall loss function to train network $f$ is defined by summing Eq. \ref{eq:ent} with the cross entropy loss $L$ on the available labels $y$ for $f$ used for standard supervised training:
\begin{equation}
    \mathcal{L}(f,y,\alpha_1,\ldots,\alpha_r) = L(f,y) + \lambda \sum_{i=1}^r\mathcal{H}(\alpha^i),
    \label{eq:loss}
\end{equation}
where $\lambda > 0$ is the hyperparameter used to balance the relative importance of low-entropy solutions in the loss function. Higher values of $\lambda$ lead to sparser configuration of $\alpha$, constraining the network to focus on a smaller set of concepts for each classification task (and vice versa), thus encoding the inductive human bias towards simple explanations \citep{miller1956magical,cowan2001magical,ma2014changing}. For further details on the impact of $\lambda$ on the model predictions and explanations (see Appendix).
It may be pointed out that a similar regularization effect could be achieved by simply minimizing the $L_1$ norm over $\gamma^i$. However, as we observed in the Appendix, the $L_1$ loss does not sufficiently penalize the concept scores for those features which are uncorrelated with the predicted category. The Entropy loss, instead, correctly shrink to zero concept scores associated to uncorrelated features while the other remains close to one.

\section{Experiments and Results}

Experiments show how entropy-based networks outperform state-of-the-art white box models such as BRL and decision trees
% \footnote{The height of the tree is limited to obtain rules of comparable lengths. See Appendix \ref{appendix:exp_details}.} 
and interpretable neural models such as $\psi$ networks on challenging classification tasks (Table \ref{tab:model-accuracy}). 
Moreover, the entropy-based regularization and the adoption of a concept-based neural network have minor affects on the classification accuracy of the explainer when compared to
% the proposed architecture matches (and sometimes surpasses) the classification accuracy provided 
a standard black box neural network
% \footnote{%\pb{This is a neural network having the same architecture and hyperparameters of the entropy-based network, with the only exception of the weight hyperparameter $\lambda$ in the loss function (see Eq. \ref{eq:loss}) which is set to $\lambda=0$. This setting makes the network free from any constraint related to explainability.}
% In the case of MIMIC-II and V-Dem, this is a standard neural network with the same hyperparameters of the entropy-based one, but with a linear layer as first layer. In the case of MNIST and CUB, it is the $g$ model directly predicting the final classes $g:X\rightarrow Y$.}
directly working on the input data, and a Random Forest model applied on the concepts.% as shown in Table \ref{tab:model-accuracy}.}
At the same time, the logic explanations provided by entropy-based networks are better than $\psi$ networks and almost as accurate as the rules found by decision trees and BRL, while being far more concise, as demonstrated in Fig.~\ref{fig:multi-objective}. 
More precisely, logic explanations generated by the proposed approach represent non-dominated solutions \citep{marler2004survey} \textit{quantitatively} measured in terms of complexity and classification error of the explanation. %(i.e. $100$ minus the classification accuracy of the explanation).
Furthermore, the time required to train entropy-based networks is only slightly higher with respect to Decision Trees but is lower than $\psi$ Networks and BRL by one to three orders of magnitude (Fig. \ref{fig:time}), making it feasible for explaining also complex tasks. 
% In addition, we observe how the proposed approach consistently outperform $\psi$ networks across all the main metrics (i.e. classification accuracy, explanation accuracy, and fidelity). 
The fidelity (Table~\ref{tab:fidelity})
% \footnote{We did not compute the fidelity of decision trees and BRL as they are trivially rule-based models.} 
of the formulas extracted by the entropy-based network is always higher than $90\%$ with the only exception of MIMIC. This means that almost any prediction made using the logic explanation matches the corresponding prediction made by the model, making the proposed approach very close to a white box model.
%The combination of 
These results empirically shows that our method represents a viable solution for a safe %the lawful 
deployment of \textit{explainable} cutting-edge models.
% The complexity of decision tree formulas is never below 100 terms, making them useless as explanations.
% In terms of fidelity the proposed approach closed the gap from white-box models like decision trees and BRL, whose fidelity is always 100\% \textit{by design}.



\begin{table}[t]
\small
\centering
% \vspace{-3mm}
%\parbox{.49\linewidth}{
\begin{tabular}{lll}
\toprule
{} &         Entropy net  &        $\psi$ net \\
\midrule
\textbf{MIMIC-II     } &  ${\bf 79.11  \pm 2.02}$ &   $51.63 \pm 6.67$ \\
\textbf{V-Dem         }&  ${\bf 90.90 \pm 1.23}$ &  $69.67 \pm 10.43$ \\
\textbf{MNIST}         &  ${\bf 99.63 \pm 0.00}$ & $65.68 \pm 5.05$ \\
\textbf{CUB         }  &  ${\bf 99.86 \pm 0.01}$ &  $77.34 \pm 0.52$ \\
\bottomrule
\end{tabular}
\caption{Out-of-distribution fidelity (\%)}
\label{tab:fidelity}
% \vspace{-1mm}
\end{table}

% \begin{figure}[t]
%     \centering
%     \includegraphics[width=0.49\columnwidth]{figs/MIMIC-II_pareto.pdf}
%     \includegraphics[width=0.49\columnwidth]{figs/V-Dem_pareto.pdf}\\
%     \includegraphics[width=0.49\columnwidth]{figs/MNIST_pareto.pdf}
%     \includegraphics[width=0.49\columnwidth]{figs/CUB_pareto.pdf}
%     % \vskip -2mm    
%     \includegraphics[width=\columnwidth]{figs/legend.pdf}\\
%     \caption{Non-dominated solutions  \citep{marler2004survey} (dotted black line) in terms of average explanation complexity and average explanation test error. The vertical dotted red line marks the maximum explanation complexity laypeople can handle (i.e. complexity $\approx 9$, see  \citep{miller1956magical,cowan2001magical,ma2014changing}). Notice how the explanations provided by the Entropy-based Network are always one of the non-dominated solution.
%     % When humans compare a set of hypotheses outlining the same outcomes, they tend to have an implicit bias towards the simplest ones, making explanations from entropy-based networks the best choice.
%     }
%     % \vskip -1mm
%     \label{fig:multi-objective}
% \end{figure}
\begin{table}[t]
\small
\centering
% \vspace{-3mm}
\begin{tabular}{lllll}
\toprule
{} & Entropy net     &              Tree &               BRL &        $\psi$ net\\
\midrule
\textbf{MIMIC-II     } &  $28.75$ &    ${\bf 40.49}$ &   $30.48$ &    $     27.62$ \\
\textbf{V-Dem         }&  $46.25$ &    $72.00$ &   ${\bf 73.33}$ &    $     38.00$ \\
\textbf{MNIST}         &  ${\bf 100.00}$ &   $41.67$ &  ${\bf 100.00}$ &    $96.00$ \\
\textbf{CUB         }  &  $35.52$ &    $21.47$ &   ${\bf 42.86}$ &    $41.43$ \\
\bottomrule
\end{tabular}
\caption{Consistency (\%)}
\label{tab:consistency}
% \vspace{-2mm}
\end{table}


% \begin{figure}[t]
%     \centering
%     \includegraphics[width=.99\columnwidth]{LaTeX/figs/elapsed_time_plot_elens.pdf}
%     \caption{Time required to train models and to extract the explanations. Our model compares favorably with the competitors, with the exception of Decision Trees. BRL is by one to three order of magnitude slower than our approach.}
%     % Error bars show the 95\% confidence interval of the mean.
%     \label{fig:time}
%     % \vskip -1mm
% \end{figure}


The reason why the proposed approach consistently outperform $\psi$ networks across all the key metrics (i.e. classification accuracy, explanation accuracy, and fidelity) can be explained observing how entropy-based networks are far less constrained than $\psi$ networks, both in the architecture (our approach does not apply weight pruning) and in the loss function (our approach applies a regularization on the distributions $\alpha^i$ and not on all weight matrices). Likewise, the main reason why the proposed approach provides a higher classification accuracy with respect to BRL and decision trees may lie in the smoothness of the decision functions of neural networks which tend to generalize better than rule-based methods, as already observed by Tavares et al. \citep{tavares2020understanding}.
For each dataset, we report in the Appendix a few examples of logic explanations extracted by each method, as well as in Fig. \ref{fig:experiments}. We mention that the proposed approach is the only matching the %logically correct 
ground-truth explanation for the MNIST even/odd experiment, i.e. $\forall x, \mathrm{isOdd(x)} \leftrightarrow \mathrm{isOne(x)} \oplus \mathrm{isThree(x)} \oplus \mathrm{isFive(x)} \oplus \mathrm{isSeven(x)} \oplus \mathrm{isNine(x)}$ and $\forall x, \mathrm{isEven(x)} \leftrightarrow \mathrm{isZero(x)} \oplus \mathrm{isTwo(x)} \oplus \mathrm{isfour(x)} \oplus \mathrm{isSix(x)} \oplus \mathrm{isEight(x)}$, being $\oplus$ the exclusive OR.
In terms of formula consistency, we observe how 
BRL is the most consistent rule extractor, closely followed by the proposed approach (Table \ref{tab:consistency}).

\section{Impact and Significance}
Summary of contributions: a sparse attention mechanism for concept bottlenecks.

Summary of results: a scalable, self-explaining neural approach providing first-order logic explanations for its predictions. The new approach is almost as accurate as black box models (e.g., an equivalent neural model with same number of parameters/layers, random forest). The explanations provided by the new approach are: (i) much more concise than those provided by existing white box rule learners (e.g., decision trees and bayesian rule lists), (ii) as accurate as the ones provided by existing white boxes. It's a way to rigorously assess whether concept-based models learn as intended by checking how they arrived to a prediction.
- the concept-based and the XAI field as novel self-explaining approach which may encourage further advances in concept learning

- other research disciplines as the extraction of FOL rules can be useful to answer research questions in a human interpretable way

- production/society as self-explaining models might become a legal requirement


Impact/significance: self-explaining models may soon become a legal requirement in Europe for the ethical deployment of AI models. So, this work has an impact on:

- the concept-based and the XAI field as novel self-explaining approach which may encourage further advances in concept learning

- other research disciplines as the extraction of FOL rules can be useful to answer research questions in a human interpretable way

- production/society as self-explaining models might become a legal requirement


\section{The Accuracy-vs-Interpretability Trade-Off}
Next steps: the proposed approach provides a good compromise between accuracy and explainability and it's Pareto optimal compared to black and white-boxes but it does not outperforms existing approaches on BOTH explainability and accuracy. How can we solve this trade-off?



\section*{Papers}
\nobibliography*
\begin{itemize}
    \item \bibentry{ciravegna2021logic}
    \item \bibentry{barbiero2021entropy}
    \item \bibentry{jain2022extending}
    \item Federico Siciliano, Pietro Barbiero, ..., Pietro Lio'. Explaining Neural Networks Using a Ruleset Based on Interpretable Concepts. \textit{arXiv preprint arXiv:XXXX.YYYYY},2023
\end{itemize}


%%%%%%%%%%%%%%%%%%%%%%%%%%%%%%%%%%%%%%%%%%%%%%%%%%%%%%%%%%%%%%%%%%%%%%%%%%%%%%%%
%% Beyond the Accuracy-vs-Interpretability Trade-Off:
%%
\chapter{Concept embeddings} \label{chapter:cem}
% \textbf{Research: completed. Status: drafted. Difficulty: low. Priority: low.}

% \textit{In this chapter I will illustrate how concepts can be used to train interpretable models outperforming their non-interpretable equivalents in terms of raw task performance. In particular, I will focus on my contributions in inventing Concept Embedding Models (CEMs)~\citep{zarlenga2022concept} which are now the state of the art of supervised concept-based models. I will describe CEMs architectures and learning paradigms. I will also discuss how CEMs support effective human interactions through learnt concepts highlighting how these interactions can increase both model performances and human trust in the model~\citep{shen2022trust}. I will conclude the chapter demonstrating how CEMs outperform equivalent non-interpretable architectures and state-of-the-art concept-based models on synthetic and real-world datasets.}

% \section{Motivation}
\textbf{Motivation}---In the previous chapter we discussed metrics to evaluate and compare different concept-based models we introduced in Chapter~\ref{chapter:background}. In particular we used these metrics to identify the main weaknesses of state-of-the-art concept-based models. In this chapter we are going to focus on one of the main weaknesses we identified in Concept Bottleneck Models (CBMs): the accuracy VS explainability trade off. Recalling from the previous chapter, this kind of issue is one of the key concerns in state-of-the-art explainable-by-design models as they often struggle to provide a good compromise between the accuracy of their predictions and the quality of their explanations. State-of-the-art CBMs do not escape this doom as they either provide high task performance (accuracy) or high concept alignment (interpretability) when solving challenging problems:
%\begin{figure}[h]
%\includegraphics[width=\textwidth]{}
%\caption{Accuracy VS explainability trade off in Concept Bottleneck Models (CBMs). Traditional CBMs struggle to find a good compromise between task accuracy and explainability. The red star at the top right marks the optimal compromise.}
%\label{fig:cbm_tradeoff}
%\end{figure}

On top of this issue, CBMs also struggle in real-world conditions where concept supervisions are scarce and noisy. In these scenarios, these models struggle in reaching high task accuracy as the set of concepts might not hold enough information to solve the task (see Figure~\ref{fig:cbm_real_world}). This problem is known as ``concept incompleteness'' and might heavily affect the deployment standard CBMs:
%\begin{figure}[h]
%\includegraphics[width=\textwidth]{}
%\caption{Real-world conditions (e.g., concept incompleteness and noisy supervisions) impair Concept Bottleneck Models (CBMs) task accuracy. As concept supervisions available for training are reduced, the task accuracy drops dramatically.}
%\label{fig:cbm_real_world}
%\end{figure}

To overcome this limitation, \citet{mahinpei2021promises} proposed to augment the learning capacity of CBMs by introducing extra neurons at concept level whithout imposing any direct concept supervision on their activations. This solution (a.k.a. ``hybrid CBMs'') allows the model to efficiently solve tasks even in noisy and concept-incomplete settings. However, this performance improvement comes with a cost: test-time concept interventions become ineffective (see Figure~\ref{fig:hybrid_interventions}):
%\begin{figure}[h]
%\includegraphics[width=\textwidth]{}
%\caption{Concept interventions are ineffective in hybrid CBMs.}
%\label{fig:hybrid_interventions}
%\end{figure}

This suggests that hybrid CBMs are prone to the phenomenon known as ``shortcut learning'': task performance is independent from concept activations as it relies mostly on the unsupervised extra neurons. This result demonstrates that hybrid CBMs cannot provide reliable concept-based explanations for task predictions nor they can effectively interact with human experts through the learnt concepts.

\textbf{Solution}---To fill these knowledge gaps, in this chapter we will present Concept Embedding Models (CEMs,~\citep{zarlenga2022concept}), a novel class of CBMs aiming at:
\begin{itemize}
	\item breaking the accuracy VS explainability trade off in CBMs;
	\item scaling CBMs to real-world conditions where concept supervisions are scarce and noisy;
	\item supporting effective and simple human interventions through the learnt concepts.
\end{itemize}
The \textbf{key innovation} of CEMs is a fully supervised high-dimensional concept representation. This high-dimensional representation increases the capacity of CEMs at concept level. The increased model capacity allows to encode more information in each concept beyond the probability of a concept being active/inactive, including contextual nuances which CEMs can use to have a deeper understanding of each concept and to solve tasks more efficiently.

We will first present CEM's architecture~\ref{sec:cem} and then we will demonstrate how CEMs fill the key CBMs knowledge gaps we discussed with a set of experiments of increasing complexity~\ref{sec:cem}.

%\paragraph{Research questions---}
%To evaluate these models and compared them to the current SOTA we use:
%
%- performance: model accuracy
%
%- interpretability/trust: concept alignment score, logic explanation accuracy/complexity, test-time concept interventions


%
%Research questions: how would this approach compare with existing approaches? is it interpretable as white boxes in terms of rule complexity? is it accurate as black boxes? what's the accuracy-vs-explainability trade-off of the approach? is there any advantage of using this approach over existing methods?
%
%\section{The Information Bottleneck in Concept Learning}

\section{Concept Embedding Models} \label{sec:cem}
Concept Embedding Models are a family of Concept Bottleneck Models with fully supervised high-dimensional concept representations. In particular, for each concept CEMs learn a mixture of two embeddings with explicit semantics representing the concept's activity. Such design allows CEMs to construct evidence both in favour of and against a concept being active, and supports simple concept interventions as one can switch between the two embedding states at intervention time.
%
% \begin{figure}[t]
%     \centering
%     \includegraphics[width=\textwidth]{fig/mixcem_simple_extended.png}
%     \caption{\textbf{Mixture of Concept Embeddings Model}: from an intermediate latent code $\mathbf{h}$, we learn two embeddings per concept, one for when it is active (i.e., $\hat{\textbf{c}}^+_i$), and another when it is inactive (i.e., $\hat{\textbf{c}}^-_i$). Each concept embedding (shown in this example as a vector with $m=2$ activations) is then aligned to its corresponding ground truth concept through the scoring function $s(\cdot)$, which learns to assign activation probabilities $\hat{p}_i$ for each concept. These probabilities are used to output an embedding for each concept via a weighted mixture of each concept's positive and negative embedding.
%     }
%     \label{fig:split_emb_architecture}
% \end{figure}
%
More specifically, we represent concept $c_i$ with two embeddings $\hat{\textbf{c}}^+_i, \hat{\textbf{c}}^-_i \in \mathbb{R}^m$, each with a specific semantics: $\hat{\textbf{c}}^+_i$ represents its active state (concept is \texttt{true}) while $\hat{\textbf{c}}^-_i$ represents its inactive state (concept is \texttt{false}). To this aim, a DNN $\psi(\mathbf{x})$ learns a latent representation $\mathbf{h} \in \mathbb{R}^{n_\text{hidden}}$ which is the input to any CEM. 
% In practice this could be any differentiable encoder architecture designed to learn a meaningful concise representation of the input or it could be an identity function.
MixCEM then feeds $\mathbf{h}$ into two concept-specific fully connected layers, which learn two concept embeddings in $\mathbb{R}^m$, namely $\hat{\mathbf{c}}^+_i = \phi^+_i(\mathbf{h}) = a(W^+_i\mathbf{h} + \mathbf{b}^+_i$) and $\hat{\mathbf{c}}^-_i = \phi^-_i(\mathbf{h}) = a(W^-_i\mathbf{h} + \mathbf{b}^-_i)$.\footnote{In practice, we use a leaky-ReLU for the activation $a(\cdot)$} Notice that while more complicated architectures/models can be used to parameterise our concept embedding generators $\phi^+_i(\mathbf{h})$ and $\phi^-_i(\mathbf{h})$, we opted for a simple one-layer neural network to constrain parameter growth in models with large bottlenecks.

Our architecture encourages embeddings $\hat{\mathbf{c}}^+_i$ and $\hat{\mathbf{c}}^-_i$ to be aligned with ground-truth concept $c_i$ via a learnable and differentiable scoring function $s: \mathbb{R}^{2 m} \rightarrow [0, 1]$, trained to predict the probability $\hat{p}_i$ of concept $c_i$ being active from the embeddings' joint space, i.e., $\hat{p}_i \triangleq s([\hat{\mathbf{c}}^+_i, \hat{\mathbf{c}}^-_i]^T) =  \sigma\big(W_s[\hat{\mathbf{c}}^+_i, \hat{\mathbf{c}}^-_i]^T + \mathbf{b}_s\big)$. We constrain parameters $W_s$ and $\mathbf{b}_s$ to be shared across all concepts for parameter efficiency.
% and to be consistent with existing CBMs. 
We construct the final concept embedding $\hat{\mathbf{c}}_i$ for $c_i$ as a weighted mixture of $\hat{\mathbf{c}}^+_i$ and $\hat{\mathbf{c}}^-_i$ as:
\[
\hat{\mathbf{c}}_i \triangleq \big(\hat{p}_i \hat{\mathbf{c}}^+_i + (1 - \hat{p}_i) \hat{\mathbf{c}}^-_i \big)
\]
Intuitively, this serves a two-fold purpose: (i) it forces the model to depend only on $\hat{\mathbf{c}}^+_i$ when the $i$-th concept is active, i.e., $c_i = 1$ (and only on $\hat{\mathbf{c}}^-_i$ when inactive), leading to two different semantically meaningful latent spaces, and (ii) it enables a clear intervention strategy where one switches the embedding states when correcting a mispredicted concept, as discussed below.

Finally, MixCEM concatenates all $k$ mixed concept embeddings, resulting in a bottleneck $g(\mathbf{x}) = \hat{\textbf{c}}$ with $k\cdot m$ units (see end of Figure~\ref{fig:split_emb_architecture}). This is passed to the label predictor $f$ to obtain a downstream task label. In practice, following~\citet{koh2020concept}, we use an interpretable label predictor $f$ parameterised by a simple linear layer, though more complex functions could be explored too. Notice that as in vanilla CBMs, MixCEM provides a concept-based explanation for the output of $f$ through its concept probability vector $\hat{\mathbf{p}} \triangleq [\hat{p}_1, \cdots, \hat{p}_k ]$, indicating the predicted concept activity. This architecture can be trained in an end-to-end fashion by \textit{jointly} minimising via stochastic gradient descent a weighted sum of the cross entropy loss on both task prediction and concept predictions:
% To train our architecture, we aim to produce both accurate downstream predictions and meaningful concept-based explanations $\hat{\mathbf{p}}$ of those predictions. For this, we minimize the following loss via stochastic gradient descent:
\begin{align}
    % \mathcal{L} := \mathbb{E}_{(\mathbf{x}, \mathbf{y}, \mathbf{c})}\Big[ \mathcal{L}_\text{task}\Big(\mathbf{y}, f\big(g(\mathbf{x}) \; \big| \big| \; s(g(\mathbf{x}))\big)\Big) + \alpha \mathcal{L}_\text{CrossEntr}\Big(\mathbf{c}, s\big(g(\mathbf{x})\big)\Big) \Big]
    \mathcal{L} \triangleq \mathbb{E}_{(\mathbf{x}, y, \mathbf{c})}\Big[ \mathcal{L}_\text{task}\Big(y, f\big(g(\mathbf{x})\big)\Big) + \alpha \mathcal{L}_\text{CrossEntr}\Big(\mathbf{c}, \hat{\mathbf{p}}(\mathbf{x})\Big) \Big]
\end{align}
where hyperparameter $\alpha \in \mathbb{R}^+$ controls how much we value concept accuracy w.r.t. downstream task accuracy. 


\section{Beyond the Accuracy-vs-Interpretability Trade-Off}
\subsection{Task Accuracy}

\paragraph{MixCEM improves generalisation accuracy (y-axis of Figure \ref{fig:accuracy}).}
Our evaluation shows that embedding-based CBMs (i.e., Hybrid-CBM and MixCEM) can achieve even better downstream accuracy than DNNs without concepts, and can easily outperform Boolean and Fuzzy CBMs by a large margin (up to $+45\%$ on Dot). This effect is emphasised when the downstream task is not a linear function of the concepts (e.g., XOR and Trigonometry) or when concept annotations are incomplete (e.g., Dot and CelebA). At the same time, we observe that all models achieve a similar high mean concept accuracy for all datasets (see Appendix~\ref{sec:taks_and_concept_perf}). This suggests that, as hypothesised, the trade-off between concept accuracy and task performance in concept-incomplete tasks is significantly alleviated by the introduction of concept embeddings in a CBM's bottleneck. Finally, notice that CelebA showcases how including concept supervision during training (as in MixCEM) can lead to an even higher task accuracy than the one obtained by the no-concept model ($+5\%$). This result further suggests that concept embedding representations enable high levels of interpretability without sacrificing performance.
% These results show that concept embedding models make use of their extra capacity to encode important information needed for the downstream task which cannot be effectively encoded in a scalar/binary vector; all without the need of sacrificing the accuracy of their explanations.
% On the XOR dataset all concept representations lead to similar task accuracy (close to $100\%$). On the Trigonometric dataset, the task accuracy of Boolean models drops to about $80\%$. On the Dot dataset, only the task accuracy of concept embedding models is above $90\%$, outperforming Boolean- and Fuzzy-based models by $\sim 20\%$ accuracy. On CUB Isolated Concept Embeddings outperform scalar concepts, but the differences are less pronounced as the bottleneck is large. Notably, in CelebA the proposed concept embedding models outperform even standard end-to-end models by $\sim 10\%$ accuracy.


% \begin{figure}[t]
%     \centering
%     \includegraphics[width=\textwidth]{fig/tradeoff.pdf}
%     \caption{Accuracy-vs-interpretability trade-off in terms of \textbf{task accuracy} and \textbf{concept alignment score} for different concept bottleneck models. In CelebA, our most constrained task, we show the top-1 accuracy for consistency with other datasets.
%     % As the classification problem gets harder, concept embedding models outperform scalar-based concept methods. Notably, in CelebA our proposed approach outperforms even end-to-end models by a significant margin.
%     For these results, and those that follow, we compute all metrics on test sets across $5$ seeds and report their mean and $95\%$ confidence intervals.}
%     \label{fig:accuracy}
% \end{figure}

% \begin{figure}[!t]
%     \centering
%     %  \centering
%     %  \begin{subfigure}[b]{1\textwidth}
%     %      \centering
%     %      \includegraphics[width=0.4\textwidth]{fig/inference.pdf}
%     %      \caption{Task inference from concepts, i.e. $\mathbb{P}(Y|\hat{C})$. \fg{$R^e$}}
%     %      \label{fig:cbm_inference_scheme}
%     %  \end{subfigure}\\
%     %  \vspace{2mm}
%     %  \centering
%     %  \begin{subfigure}[b]{1\textwidth}
%     %      \centering
%      \includegraphics[width=\textwidth]{fig/repr_accuracy_test.pdf}
%      \caption{Test accuracy of a random forest trained on different components of the concept representation. On the XOR dataset, all concept representations and components lead to similar task accuracy. On the Trigonometric dataset, the task accuracy of the model based on Boolean concepts drops significantly. On the Dot dataset, only the model based on concept embeddings context has high task accuracy. The task accuracy based on concept semantics is similar for all models and for all datasets.}
%     %      \label{fig:cbm_inference_results}
%     %  \end{subfigure}
%     % \caption{Inference.}
%     \label{fig:cbm_inference_results}
% \end{figure}


\paragraph{MixCEM overcomes the information bottleneck (Figure~\ref{fig:info_plane}).}
The Information Plane method indicates, as hypothesised, that embedding-based CBMs (i.e., Hybrid-CBM and MixCEM) do not compress input data information, with $I(X, C)$ monotonically increasing during training epochs. On the other hand, Boolean and Fuzzy CBMs, as well as vanilla end-to-end models, tend to ``forget''~\citep{shwartz2017opening} input data information in their attempt to balance competing objective functions. Such a result constitutes a plausible explanation as to why embedding-based representations are able to maintain both high task accuracy and mean concept accuracy compared to CBMs with scalar concept representations. In fact, the extra capacity allows CBMs to maximise concept accuracy without over-constraining concept representations, thus allowing useful input information to pass by. In MixCEMs all input information flows through concepts, as they supervise the whole concept embedding. In contrast with Hybrid models, this makes the downstream tasks completely dependent on concepts, which explains the higher concept alignment scores obtained by MixCEM (see below).
% The relationship between the quality of concept representations w.r.t. the input distribution remains widely unexplored. In this work we propose investigating this relationship using the notion of the Information Plane~\cite{tishby2000information}.
% : we would be (probably) the first at pinpointing a clear relationship.
% Specifically, we conjecture that employing embeddings as concept representations circumvents the information bottleneck by preserving more information from the input distribution as part of their high-dimensional activations. Such a result would constitute a pausible explanation as to why embedding-based representations are able to maintain both high task accuracy and mean concept accuracy compared to CBMs with scalar concept-representations. If true, we believe such effect should be captured by the Information Plane in the form of a positively correlated evolution of $I(X, C)$, the MI between inputs $X$ and learnt concept representations $C$,  and $I(C, Y)$, the MI between learnt concept representations $C$ and task labels $Y$. In contrast, we anticipate that scalar-based concept representations (e.g., Fuzzy and Bool CBMs), as well as end-to-end models, will be forced compress the information from the input data at concept level, leading to a compromise between the $I(X, C)$ and $I(C, Y)$. 
% We empirically show this exact phenomenon in Figure~\ref{fig:info_plane}, where we plot the evolution of the $I(X, C)$ vs $I(C, Y)$ evaluated on the test set of each task and model during training. Notice that our results indicate that CBMs using embedding representations do not compress input data information (i.e., $I(X, C)$ remains monotonically increasing during training) while CBMs using scalar-based representations, as well as vanilla end-to-end models, tend to ``forget'' input data information in their attempt to balance competing objective functions (e.g., interpretability and downstream performance). \todo{THIS NEEDS TO BE CONCLUDED. What is the main takeaway from this experiment and what is the message we want the reader to take from it?}.



% \begin{figure}[!ht]
%     \centering
%     \includegraphics[width=\textwidth]{fig/mutual_info_xcy.pdf}
%     % \hspace{1.cm}\textsc{XOR}\hspace{4.2cm}\textsc{DOT}\hspace{3.5cm}\textsc{Trigonometry}\\
%     % \vspace{-.cm}
%     % \includegraphics[width=0.2\textwidth]{fig/xor_plot_I(X,T)vsI(T,Y).pdf}
%     % \includegraphics[width=0.2\textwidth]{fig/dot_plot_I(X,T)vsI(T,Y).png}
%     % \includegraphics[width=0.2\textwidth]{fig/trig_plot_I(X,T)vsI(T,Y).png}\\
%     % \includegraphics[trim=0 0 50 0, clip, width=0.32\textwidth]{fig/xor_plot_I(X,T)vsI(T,C).png}
%     % \includegraphics[trim=0 0 50 0, clip, width=0.32\textwidth]{fig/dot_plot_I(X,T)vsI(T,C).png}
%     % \includegraphics[trim=0 0 50 0, clip, width=0.32\textwidth]{fig/trig_plot_I(X,T)vsI(T,C).png}\\
%     % \includegraphics[trim=22 235 0 92, clip, width=0.65\textwidth]{fig/legend1.png}    
%     % \includegraphics[trim=115 235 102 90, clip, width=0.33\textwidth]{fig/legend2.png}
%     \caption{Mutual Information (MI) of concept representations ($\hat{C}$) w.r.t. input distribution ($X$) and ground truth labels ($Y$) during training.
%     % MI between concept representations and task labels ($I(T;Y))$) against MI between concept representations and input distributions ($I(X;T))$) (top row). MI between concept representations and concept labels ($I(T;C))$) against MI between concept representations and input distributions ($I(X;T))$) (bottom row).
%     Each point is produced by averaging over $5$ runs. The size of the points is proportional to the training epoch.%\todo{Add legend and results for CUB and CelebA.}
%     }
%     \label{fig:info_plane}
% \end{figure}

\subsection{Interpretability}

% \giu{The headings of the following two paragraphs overlaps a bit in message. Make the distinction a bit sharper. Maybe also switch their order?  }

\paragraph{MixCEM learns more interpretable concept representations (x-axis of Figure~\ref{fig:accuracy}).}
Using the proposed CAS metric, we show that concept representations learnt by MixCEMs have alignment scores competitive or even better (e.g., on CelebA) than the ones of Boolean and Fuzzy CBMs. The alignment score also shows, as hypothesised, that hybrid concept embeddings are the least faithful representations---with alignment scores up to $25\%$ lower than MixCEM in the Dot dataset. This is due to their unsupervised activations containing information which may not be necessarily relevant to a given concept. This result  is a further evidence for why we expect interventions to be ineffective in Hybrid models (as we show shortly).

% \begin{figure}[!ht]
%     \centering
%     \includegraphics[width=\textwidth]{fig/all_concept_auc.pdf}
%     \caption{\textbf{Alignment of learnt concepts w.r.t. their ground truth labels.} As the classification problem gets harder, the proposed approach outperforms hybrid representations by a large margin. Alignment scores of hybrid concepts are close to random chance. Notably, in CUB$^*$ concepts learnt by the proposed approach are even more aligned with the corresponding labels compared with Boolean or Fuzzy concepts. Concept alignment is computed on test sets over $5$ runs. Results are reported with the mean and its $95\%$ confidence interval.}
%     \label{fig:alignment}
% \end{figure}



\paragraph{MixCEM captures meaningful concept semantics (Figure \ref{fig:xai}).}
% \todo{This entire subsection needs to be updated (MATEO). Please do not review yet.... Story to be told here: (1) our concept embeddings can be useful as representations for downstream tasks that depend only on a few concepts (2) why? we can clearly see that their latent space is very separable compared to that of Hybrid by maintaining a clear separation between activated and inactivated concepts (3) this is further corroborated by looking at nearest neighbors of a given concept.}
Our concept alignment results hint at the possibility that concept embeddings learnt by MixCEM may be able to offer more than simple concept prediction. In fact, we hypothesise that their seemingly high alignment may lead to these embeddings forming more interpretable representations than Hybrid embeddings, which can lead to more useful representations for external downstream tasks. To explore this, we train a Hybrid-CBM and a MixCEM using a variation of CUB with only 25\% of its concept annotations randomly selected before training, resulting in a bottleneck with 28 concepts. Once these models have been trained to convergence,
we use their learnt bottleneck representations to predict the remaining 75\% of the concept annotations in CUB using a simple logistic linear model. The model trained using the Hybrid bottleneck notably underperfoms when compared to the model trained using the MixCEM bottleneck (Hybrid-trained model has a mean concept accuracy of 91.83\% $\pm$ 0.51\% while the MixCEM-trained model's concept accuracy is 94.33\% $\pm$ 0.88\%). This corroborates our CAS results by suggesting that the bottlenecks learnt by MixCEMs are considerably more powerful as interpretable representations and can be used in separate downstream tasks.

We can further explore this phenomena qualitatively by visualising the embeddings learnt for a single concept using its 2-dimensional t-SNE~\citep{van2008visualizing} plot.
% and (ii) by looking at samples whose concept embeddings $\{\hat{\mathbf{c}}_i^{(1)}, \hat{\mathbf{c}}_i^{(2)}, \cdots \}$ for concept $c_i$ are closest to the embedding $\hat{\mathbf{c}}_i^{(\text{test})}$ of a test point $\mathbf{x}^{(\text{test})}$.
As shown in colour in Figure~\ref{fig:mixcem_tsne}, we can see that the embedding space learnt for a concept $\hat{\mathbf{c}}_i$ (we show here the concept ``has white wings'') forms two clear clusters of samples, one for points in which the concept is active and one for points in which the concept is inactive. When performing a similar analysis for the same concept in the Hybrid CBM (Figure~\ref{fig:hybrid_tsne}), where we use the entire extra capacity as the concept's representation, we see that this latent space is not as clearly separable as that in MixCEM's embeddings, suggesting this latent space is unable to capture concept-semantics as clearly as MixCEM's latent space. Notice that MixCEM's t-SNE seems to also show smaller subclusters within the activated and inactivated clusters. As Figure~\ref{fig:mixcem_nn} shows, by looking at the nearest Euclidean neighbours in concept's $c_i$ embedding's space, we see that MixCEM concepts do not only clearly capture a concept's activation, but they exhibit high class-wise coherence by mapping same-type birds close to each other (explaining the observed subclusters). These results, and similar results shown in Appendix), strongly suggest that MixCEM is learning useful and interpretable high-dimensional concept representations.

% The proposed approach makes concepts' latent spaces isolated. For each concept the approach generates clusters strongly correlated with concept labels (top row). As a result, each super-cluster has a straightforward interpretation: concept on / concept off---as in Boolean or fuzzy representations. Interpretable sub-clusters correlated to both concepts and tasks emerge in latent spaces (see Figure \ref{fig:tsne_tasks} and \ref{fig:bird_clusters}, top): samples' nearest neighbors are coherent and share common features (see Figure \ref{fig:bird_nearest}, top). On the contrary, in hybrid concept representations samples do not cluster according to concept labels (see Figure \ref{fig:xai}, bottom row) as the extra neurons are not supervised and contain strongly entangled information. In hybrid concepts, small clusters are less coherent (see Figure \ref{fig:tsne_tasks} and \ref{fig:bird_clusters}, bottom).

% \begin{figure}[!ht]
%     \centering
%     \begin{subfigure}[b]{0.3\textwidth}
%         \centering
%         \includegraphics[width=\textwidth]{fig/mixcem_tsne.png}
%         \subcaption{
%         }
%         \label{fig:mixcem_tsne}
%     \end{subfigure}
%     \begin{subfigure}[b]{0.3\textwidth}
%         \centering
%         \includegraphics[width=\textwidth]{fig/hybrid_tsne.png}
%         \subcaption{
%         }
%         \label{fig:hybrid_tsne}
%     \end{subfigure}
%     \begin{subfigure}[b]{0.3\textwidth}
%         \centering
%         \includegraphics[width=\textwidth]{fig/mixcem_nn.png}
%         \subcaption{
%         }
%         \label{fig:mixcem_nn}
%     \end{subfigure}
    
%     % \begin{subfigure}[b]{0.23\textwidth}
%     %      \centering
%     %      \includegraphics[width=\textwidth]{fig/xai/tsne_task_splitemb_perching.png}
%     %     %  \caption{Low dimensional representation of the neural symbolic concept embedding "has wing color grey".}
%     %     %  \label{fig:tsne_emb}
%     %  \end{subfigure}\quad
%     % \begin{subfigure}[b]{0.23\textwidth}
%     %      \centering
%     %      \includegraphics[width=\textwidth]{fig/xai/tsne_concept_splitemb_perching.png}
%     %     %  \caption{Low dimensional representation of the hybrid concept embedding "has wing color grey".}
%     %     %  \label{fig:tsne_fuzzyplus}
%     %  \end{subfigure}\quad
%     % \begin{subfigure}[b]{0.23\textwidth}
%     %      \centering
%     %      \includegraphics[width=\textwidth]{fig/xai/cluster_splitemb_perching.png}
%     %     %  \caption{Rows are clusters found from the concept embedding of "multi-colored chest".}
%     %     %  \label{fig:bird_clusters}
%     %  \end{subfigure}\quad
%     % \begin{subfigure}[b]{0.23\textwidth}
%     %      \centering
%     %      \includegraphics[width=\textwidth]{fig/xai/nearest_splitemb_perching.png}
%     %     %  \caption{Rows are clusters found from the concept embedding of "multi-colored chest".}
%     %     %  \label{fig:bird_clusters}
%     %  \end{subfigure}\\
%     % \begin{subfigure}[b]{0.23\textwidth}
%     %      \centering
%     %      \includegraphics[width=\textwidth]{fig/xai/tsne_task_fuzzyp_perching.png}
%     %      \caption{TSNE of the concept embedding "has wing color grey". Colors are tasks (bird species). Top: MixCEM. Bottom: hybrid.}
%     %      \label{fig:tsne_tasks}
%     %  \end{subfigure}\quad
%     % \begin{subfigure}[b]{0.23\textwidth}
%     %      \centering
%     %      \includegraphics[width=\textwidth]{fig/xai/tsne_concept_fuzzyp_perching.png}
%     %      \caption{TSNE of the concept embedding "has wing color grey". Red/blue colors show if "has wing color grey" is true/false. Top: MixCEM. Bottom: hybrid.}
%     %      \label{fig:tsne_concepts}
%     %  \end{subfigure}\quad
%     % \begin{subfigure}[b]{0.23\textwidth}
%     %      \centering
%     %      \includegraphics[width=\textwidth]{fig/xai/cluster_fuzzyp_perching.png}
%     %      \caption{Rows are clusters found from the concept embedding of "multi-colored chest". Top: MixCEM. Bottom: hybrid.}
%     %      \label{fig:bird_clusters}
%     %  \end{subfigure}\quad
%     % \begin{subfigure}[b]{0.23\textwidth}
%     %      \centering
%     %      \includegraphics[width=\textwidth]{fig/xai/nearest_fuzzyp_perching.png}
%     %      \caption{Sample's nearest neighbors in the embedding space of the concept "multi-colored chest". Top: MixCEM. Bottom: hybrid.}
%     %      \label{fig:bird_nearest}
%     %  \end{subfigure}\\
%     \caption{Qualitative results: (a and b) t-SNE visualisations of ``has white wings'' concept embedding learnt in CUB with sample points coloured red if the concept is active in that sample, (c) top-5 test neighbours of MixCEM embedding for the concept ``has white wings'' across 5 random test samples.}
%     \label{fig:xai}
% \end{figure}



\section{Interacting with High-Dimensional Concepts}

\subsection{Intervening with Concept Embeddings}
As in vanilla CBMs, MixCEMs support test-time concept interventions. To intervene on concept $c_i$, one can update $\hat{\mathbf{c}}_i$ by swapping the output concept embedding for the one semantically aligned with the concept ground truth label. For instance, if for some sample $\mathbf{x}$ and concept $c_i$ a MixCEM predicted $\hat{p}_i = 0.1$ while a human expert knows that concept $c_i$ is active ($c_i=1$), they can perform the intervention $\hat{p}_i := 1$. This operation updates MixCEM's bottleneck by setting $\hat{\mathbf{c}}_i$ to $\hat{\mathbf{c}}^+_i$ rather than $\big(0.1 \hat{\mathbf{c}}^+_i + 0.9 \hat{\mathbf{c}}^-_i\big)$. Such an update allows the downstream label predictor to act on information related to the corrected concept.

In addition, we introduce \textit{RandInt}, a regularisation strategy exposing MixCEMs to concept interventions during training to improve the effectiveness of such actions at test-time. RandInt randomly performs independent concept interventions during training with probability $p_\text{int}$ (i.e., $\hat{p}_i$ is set to $\hat{p}_i := c_i$ for concept $c_i$ with probability $p_\text{int}$). In other words, for all concepts $c_i$, their embeddings during training are computed as:
\[
    \hat{\mathbf{c}}_i = \begin{cases}
        \big(c_i \hat{\mathbf{c}}^+_i + (1 - c_i) \hat{\mathbf{c}}^-_i\big) & \text{with probability } p_\text{int} \\
        \big(\hat{p}_i \hat{\mathbf{c}}^+_i + (1 - \hat{p}_i) \hat{\mathbf{c}}^-_i\big) & \text{with probability } (1 - p_\text{int})
    \end{cases}
\]

while at test-time we always use the predicted probabilities for performing the mixing. During backpropagation, this strategy forces feedback from the downstream task to update only the correct concept embedding (e.g., $\hat{\mathbf{c}}^+_i$ if $c_i = 1$) while feedback from concept predictions can update both $\hat{\mathbf{c}}^+_i$ and $\hat{\mathbf{c}}^-_i$. Under this view, RandInt can be thought of as learning an average over an exponentially large family of MixCEM models (similarly to dropout~\citep{srivastava2014dropout}) where some of the concept representations are trained using only feedback from their concept label while others receive training feedback from both their concept and task labels. In the extreme case when the embedding size is $m = 1$ and we only have one concept (i.e., $k = 1$), this process can be seen as randomly alternating between learning a Joint-CBM and a Sequential-CBM during training, with $p_\text{int}$ controlling how often we switch between joint training and sequential training.


\subsection{Interventions}

\paragraph{MixCEM supports effective concept interventions and is more robust to incorrect interventions (Figure~\ref{fig:interventions}).} When describing our MixCEM architecture, we argued in favour of using a mixture of two semantic embeddings for each concept as this would permit test-time interventions which can meaningfully affect entire concept embeddings. In Figure~\ref{fig:interventions} left and center-left, we observe, as hypothesised, that using a mixture of embeddings allows MixCEMs to be highly responsive to random concept interventions in their bottlenecks. Notice that as predicted, although all models have a similar concept accuracy, we observe that Hybrid CBMs, while highly accurate without interventions, quickly fall short against even scalar-based CBMs once several concepts are intervened in their bottlenecks. In fact, we observe that interventions in Hybrid CBM bottlenecks have little effect on their predictive accuracy, something that did not change if logit concept probabilities were used instead of sigmoidal probabilities. More interestingly, however, we see in Figure~\ref{fig:interventions} center-right and right that when we perform intentionally incorrect interventions (where a concept is set to the wrong value), MixCEM's performance hit is not as sharp as that of CBMs with scalar representations. We believe this is a consequence of MixCEM's ``incorrect'' embeddings still carrying important task-specific information which can then be used by the label predictor to produce more accurate task labels. Finally, by comparing the effect of interventions in both MixCEMs and MixCEMs trained without RandInt, we observe that RandInt in fact leads to a model that is not just significantly more receptive to interventions, but is also able to outperform even scalar-based CBMs when large portions of their bottleneck are artificially set by experts. This suggests that our proposed architecture can not only be trusted in terms of its downstream predictions and concept explanations, as seen above, but it can also be a highly effective model when used along with experts that can correct mistakes in their concept predictions.

% \begin{figure}[!h]
%     \centering
%     \includegraphics[width=1.\textwidth]{fig/interventions_complete_with_adversarial.png}
%     \caption{Effects of performing positive random concept interventions (left and center left) and intentionally incorrect interventions (center right and right) for different concept representations in CUB and CelebA. As in~\citep{koh2020concept}, when intervening in CUB we intervene using groups of concepts which are mutually exclusive.}
%     \label{fig:interventions}
% \end{figure}

\section{Robustness and Cost Effectiveness}
Overall the results of our experiments demonstrate how CEMs can:
\begin{itemize}
	\item break the accuracy VS explainability trade off in CBMs;
	\item scale CBMs to real-world conditions where concept supervisions are scarce and noisy;
	\item support effective and simple human interventions through the learnt concepts.
\end{itemize}
In fact, CEMs' task accuracy is higher than equivalent black boxes' and comparable with hybrid CBMs. At the same time, CEMs' concept alignment is as good as in vanilla CBMs and much higher than in hybrid or fuzzy CBMs for challenging tasks. This can be explained thanks to the high-dimensional concept representation in CEMs which allows more information to flow through concepts, thus breaking the information bottleneck at concept level. As all neurons at concept level are supervised, all the information flowing through task depends on concept neurons, which makes tasks fully dependent on concepts, making CEMs able to support efficient concept interventions which increase human trust as opposed to hybrid CBMs.



Impact/significance: self-explainability is an ethical and (soon) legal requirement for the deployment of AI-based technologies. However, it's often coming at the cost of reducing models' accuracy. This work solves this trade-off for concept-based models, so it has an impact on:

- the concept-based and the XAI field as novel self-explaining approach which may encourage further advances in concept learning

- other research disciplines as CEMs keep working well in real-world conditions where (expensive!) concept-supervisions are scarce + users can trust predictions (model is accurate) and can interact with the model to verify the "causal" relationship between learnt tasks and concepts (or just improve model performance through interventions)

- production/society as self-explaining models might become a legal requirement and high-performance is a must

\section{A Remark on Concept Annotations}
Even though CEM is efficient in real-world conditions where concept supervisions are scarce, it still requires some supervisions at concept level. Such supervisions might be expensive and in some cases (e.g. biology) concepts might be unknown a priori which makes it impossible to train supervised concept bottleneck models. How can we train concept bottleneck models without supervised concepts?


\section*{Papers}
\nobibliography*
\begin{itemize}
    \item \bibentry{zarlenga2022concept}
\end{itemize}


%%%%%%%%%%%%%%%%%%%%%%%%%%%%%%%%%%%%%%%%%%%%%%%%%%%%%%%%%%%%%%%%%%%%%%%%%%%%%%%%
%% Interpretable Neural Symbolic Concept Reasoning:
%%
\chapter{Interpretable Concept Reasoning}
\label{chapter:DCR}

%\begin{itemize}
%	\item concept embeddings allow cbms to attain sota performance while keeping the semantics of concepts intact
%	\item however, existing interpretable concept-based models are not designed for concept embeddings
%	\item as a result, cbms applied on concept embeddings may generate explanations that are no longer semantically meaningful
%	\item as a consequence standard cbms on concept embeddings may not be trustworthy as much as cbms on concept truth values
%	\item to overcome this limitation we propose dcr, the first interpretable cbm using concept embeddings
%	\item key innovation: use neural networks to generate logic rules from concept embeddings and execute the rules using concept truth values
%	\item this way neural networks can exploit the whole information in the embeddings to build compound logic expressions while logic inference is used to make predictions on downstream tasks based on the rules and thus being totally interpretable
%\end{itemize}

\textbf{Motivation---} As we illustrated in the previous chapter, concept embeddings are a powerful concept representation as they allow concept-based models to attain state-of-the-art task performance while keeping intact the semantics of concepts. In fact, high-dimensional embeddings allow concept encoders to incorporate additional information coming from the input space which might be specific to each sample. Concept-based models can then leverage this extra information to make task predictions more accurate by considering instance-specific conditions. However, existing concept-based models are not designed to deal efficiently with concept embeddings. In fact, existing concept-based models always assume that all their input features are semantically meaningful. This way these models can leverage concept semantics to generate meaningful explanations. However, concept embeddings break this assumption as the individual features of the embedding do not have an explicit semantics. For this reason, even the most interpretable concept-based models fail to provide meaningful explanations when applied on concept embeddings.

% the only semantically explicit information is encoded in concept truth-values.

\textbf{Solution---} To fill this gap, in this chapter we introduce the Deep Concept Reasoner (DCR) the first interpretable concept-based model using concept embeddings. In particular DCR represents the first differentiable concept-based model attaining state-of-the-art performance in solving complex tasks, thanks to concept embeddings, while providing human-understandable and formal explanations for its decisions, thus representing a concrete step towards efficient and trustworthy AI systems.

The \textbf{key innovation} of DCR is the use of neural networks to generate interpretable rules. In particular, DCR generates the structure of logic rules whose terms are concept literals. To build these rules, DCR uses concept embeddings which allow DCR to customize logic expressions for different input samples as embeddings may hold instance-specific contextual information. Having generated the logic rule, DCR then executes the logic expression evaluating concept literals on the semantically meaningful concept truth degrees.

In this chapter we will first present the syntax, the semantics and the architecture of the Deep Concept Reasoner (Section~\ref{sec:dcr-method}). We will then show how DCR outperforms existing interpretable models while provide simple logic explanations (Section~\ref{sec:dcr-exp}). Finally we will discuss how DCR and concept embeddings represent a significant innovation in the context of explainable and neural-symbolic AI (Section~\ref{sec:dcr-disc}).

\section{Deep Concept Reasoning}
\label{sec:dcr-method}
% \subsection{Goal: Interpretable Model with Top Performances}
Here we describe the ``Deep Concept Reasoner'' (DCR, Figure~\ref{fig:dcr-method}), the first interpretable concept-based model based on concept embeddings.  Similarly to existing models based on concept embeddings, DCR exploits high dimensional representations of the concepts. However, in DCR, such representations are only used to compute a logic rule. The final prediction is then obtained by evaluating such rule on the concepts truth values and not on their embeddings, thus maintaining a clear semantics and providing a totally interpretable decision. 
%DCR learns logic rules by composing concept literals into a logical formula identifying relevant literals using concept embeddings. 
Being differentiable, DCR is trainable as an independent module on concept databases, but it can also be trained end-to-end with differentiable concept encoders.
% , relying its class prediction to its concept activations, and uses 
% symbolically 
% the rule itself to provide the final classification. 
%On the other hand, we enable our method to be highly expressive, therefore able to achieve a competitive predictive performance, by allowing our model to build such rules using the rich information contained in the concept embeddings~\cite{zarlenga2022concept}.
% However, our model differently from common fuzzy rule-based systems \cite{magdalena2015fuzzy}.
% , to achieve competitive predictive performance and inspired by \cite{zarlenga2022concept}, our model builds the rules on the richer embedding representations of concepts.
% In the next sections, we show how the entire process takes place. 
In the following section, we describe \mbox{(1) the} syntax of the rules we aim to learn (Section~\ref{sec:rulepred}), \mbox{(2) how} to (neurally) generate and execute learnt rules to predict task labels (Section~\ref{sec:ruleexec}), (\mbox{3) how} DCR learns simple rules in specific t-norm semantics (Section~\ref{sec:ruleexec}), and \mbox{(4) how} we can generate global and counterfactual explanations with DCR (Section~\ref{sec:ruleadv}). We provide Figure~\ref{fig:dcr-method} as a reference to graphically follow the discussion.
% \todo{add section on semantics}
% we show how concept embeddings can be exploited to obtain a symbolic fuzzy rule. 
% Lastly in Section \ref{sec:ruleadv} we consider more advanced topics like, obtaining global explanations, concept interventions and counterfactual rules.
% \todo{Maybe 3.3 $->$ New Section?}


\begin{figure*}[!t]
    \centering
    % \begin{subfigure}[b]{1.\textwidth}
    %     \centering
    %     \includegraphics[clip, trim=0.4cm 0cm 7cm 0cm, width=0.2\textwidth]{figs/dcr_method_simple.pdf}
    %     \caption{$y=x$}
    %     \label{fig:y equals x}
    % \end{subfigure}
    % \begin{subfigure}[b]{0.6\textwidth}
    %     \centering
    %     \includegraphics[clip, trim=0.4cm 0cm 7cm 0cm, width=0.62\textwidth]{figs/dcr_method_horizontal.pdf}
    %     \caption{$y=x$}
    %     \label{fig:y equals x}
    % \end{subfigure}
%    \includegraphics[clip, trim=0.4cm 0cm 7cm 0cm, width=0.22\textwidth]{figs/dcr_method_simple.pdf}
%    \qquad
%    \includegraphics[clip, trim=0cm 0cm 2.9cm 0cm, width=0.62\textwidth]{figs/dcr_method_horizontal.pdf}
    \caption{(left) Deep Concept Reasoner (DCR) generates fuzzy logic rules using neural models on concept embeddings. Then DCR executes the rule using the concept truth degrees to evaluate the rule symbolically. (right) Schema of DCR modules: first neural models $\phi$ and $\psi$ generate the rule, and then the rule is executed symbolically.}
    \label{fig:dcr-method}
\end{figure*}


\subsection{Rule Syntax}
\label{sec:rulepred}
% \giu{Two ideas for description: this section is about "predicting the rule". So in principle, it could have already included the $\phi$ and $\psi$ functions. They are just the implementation of the two filters. The idea is that, at the end of this section, the user can "read" the rule from the output of the nets.  The next section is about the rule execution: given the "rule" (i.e. output of $\phi$ and $\psi$) and given the concepts truth degrees (the model) $\hat{c}$, we can execute / interpret / evaluate the rule on this model.}
% Deep Concept Reasoners are designed such that the positive class predictions for each sample are obtained as output of a specific fuzzy rule. 
To understand the rationale behind DCR's design, we begin with an illustrative toy example:
\begin{example}
\label{ex:banana}
Consider the problem of defining the fruit ``banana'' given the vocabulary of concepts ``soft'', ``round'', and ``yellow''. A simple definition can be $y_{\textit{banana}} \Leftrightarrow \neg c_{\textit{round}} \land c_{\textit{yellow}}$. From this rule we can deduce that (i) being ``soft'' is irrelevant for being a ``banana'' (indeed bananas can be both soft or hard), and (ii) being both ``\underline{not} round" and ``yellow'' is relevant to being a ``banana''.
% the following example where a concept-based model is used to predict the type of fruit in an input image, e.g. a ``banana'',  having as available concepts, e.g. ``soft'', ``round'' and ``yellow''. 
% Namely, we are interested in computing a rule that would classify a given sample as ``banana'', given the set of available concepts as a vocabulary to use to refer to the predicted class. For instance we can expect:
% We would like to obtain $\psi_{banana}(\mathbf{c}_{yellow}) = \psi_{banana}(\mathbf{c}_{round}) =  1$, as being ``round'' and ``yellow'' are relevant concept for predicting ``banana''.
% \begin{equation*}
% y_{banana} \leftrightarrow \neg c_{round} \land c_{yellow}
% \end{equation*}
\end{example}
As in this example, DCR rules can express whether a concept is \textbf{\emph{relevant}} or not (e.g., ``soft''), and whether a concept plays a positive (e.g., ``yellow'') or negative (e.g., ``\underline{not} round'') \textbf{\emph{role}}. 
% These rules combine the activations of concepts $\mathbf{\hat{c}}$ to get class predictions in $\mathbf{\hat{y}}$ in this way: each rule can express a positive or negative role for the concepts, and they can filter out the concepts that are not ``relevant" for the class prediction.
To formalize this description of rule syntax, we let $l_{ji}$ denote the literal of concept $c_i$ (i.e., $\hat{c}_i$ or $\neg \hat{c}_i$) representing the \emph{role} of the concept $i$ for the $j$-th class. Similarly, we let $r_{ji} \in \{0,1\}$ representing whether $\hat{c}_i$ is \emph{relevant} for predicting the class $y_j$.
% DCR rules have the following syntax:
For each sample $\mathbf{x}$ and predicted class $\hat{y}_j$, DCR learns a rule with the following syntax\footnote{Here and in all equations we omit the explicit dependence on $\mathbf{x}$ for simplicity, i.e., we write $\hat{y}_j$ for $\hat{y}_j(\mathbf{x})$.}:
\begin{equation} \label{eq:target-rule}
    \hat{y}_j \Leftrightarrow \bigwedge_{i:\; r_{j i} = 1}{l_{ji}}
\end{equation}
% where: , and $r_{ji}$ can be thought of as a flag denoting whether $\hat{c}_i$ is relevant for predicting the class $\hat{y}_j$. 
Such a rule defines a logical statement for why a given sample is predicted to have label $\hat{y}_j$ using a conjunction of relevant concept literals (i.e., $\hat{c}_i$ or $\neg \hat{c}_i$).%, which can potentially be negated.


% Notice that, given a class $y_j$, the corresponding rule is completely determined if, for all concepts $c_i$, we know two indicators: (1) whether the concept is negated or not in the literal $l_{ji}$ and (2) the concept relevance flag $r_{ji}$.

% In particular, we take $y_j$ as the conjunction of only the effectively relevant concepts $\hat{c}_i$ (or their negation) for the prediction. 
%For the sake of simplicity and without loss of generality, in the following sections each rule refers to a fixed sample $\mathbf{x}$, therefore allowing us to omit writing the explicit dependence on the $\mathbf{x}$ for all other variables. 
% Without loss of generality \todo{say why: it's a DNF!}, we can structure rules as a conjunction of concept literals (or truth degrees) $\hat{c}_i$:
% \begin{equation}
%     \hat{y}_j \leftarrow \bigwedge_i \hat{c}_i
% \end{equation}
% For this sample we aim to learn a rule such as $\hat{y}_{\text{banana}} \leftrightarrow \neg \hat{c}_{\text{round}} \wedge \hat{c}_{\text{yellow}}$. As we can notice, not all of the concepts are necessary to explain the prediction,  
% Notice the two key elements of this rule: The concept ``round'' is negated and the concept ``soft'' is missing. More generally, we aim to: \\


% At the same time we expect $\psi_{banana}(\mathbf{c}_{soft}) = 0$, as bananas can be both soft or hard. Once we know which concepts are relevant, we want to understand whether they contribute positively or negatively to the class ``banana''. 
% Therefore, we expect $\phi_{banana}(\mathbf{c}_{round}) = 0$ as bananas are not round objects, while we would expect $\phi_{banana}(\mathbf{c}_{yellow}) = 1$, as bananas are yellow. Given the values of the $\phi(\cdot)$ and $\psi(\cdot)$ functions, we can uniquely identify the corresponding logical rule:







% \textbf{(Aim 1) Select relevant concepts} e.g., being ``soft'' might be not relevant for the class ``banana'', as bananas can be both soft or hard, while being ``round'' and ``yellow'' could be relevant.
% To this end we use a parameter $r_{ji} \in \{0,1\}$ corresponding to the importance of concept $i$ for class $j$:
% \begin{equation} \label{eq:target-irrelevance}
%     r_{ji} = 
%     \begin{cases}
%         1 & \text{ if } \hat{c}_i \text{ is ``relevant'' for $\hat{y}_j$} \\
%         0 & \text{ otherwise}
%     \end{cases}
% \end{equation}
% In general, we expect $\mathbf{r}_j=(r_{j1},\ldots,r_{jk})$ to have a different degree of sparsity depending on the class and on the data set. In some cases we may expect several concepts to be relevant (more $r_{ji}=1$), while in others just few of them (more $r_{ji}=0$). In the practice, we take $r_{ji}\in[0,1]$, as defined in Equation (\ref{eq:rel}).
% %  As a result on average our expectation of a concept being relevant is around $0.5$:
% % \begin{equation} \label{eq:target-expect}
% %     \mathbb{E}[r_{ji}=1] = 0.5
% % \end{equation}

% Apart from selecting the relevant concepts for a class, we are also interested in expressing their sign, i.e. if they have a positive or negative role for the prediction. 
% \\
% \textbf{(Aim 2) Learn concept literals} e.g., \textit{not being} ``round'' might be relevant for a ``banana'' as much as \textit{being} ``yellow''. To this end, we denote by $l_{i} \in [0,1]$ the  literal corresponding to concept $i$:
% \begin{equation} \label{eq:target-literal}
%     l_{i} = 
%     \begin{cases}
%         \hat{c}_i & \text{ if the concept $i$ is ``true'' (e.g., } \hat{c}_i \geq 0.5 \text{)} \\
%         \neg\hat{c}_i & \text{otherwise.} 
%         % \text{don't care} & \text{ otherwise}
%     \end{cases}
% \end{equation}

% Abusing notation, we assume each $l_{i}$ to denote both the propositional variable $\hat{c}_i$ or its negation $\neg\hat{c}_i$, and the corresponding truth-value $\hat{c}_i$ or $1-\hat{c}_i$, respectively.
% % \todo{FRA: literals do not really depend on classes.. remove j?}
% \giu{I think fuzzy-logic and logic need to be a preliminary/background, or I will skip such details later. Not many people will know what a truth-value is, and there will be a lot of confusion with this negation here if not explained before.}
% \\
% \textbf{(Aim 3) Identify classes with the conjunction of relevant concept literals} using the learnt parameters $r_{ji}$ and $l_{i}$:
% \begin{equation} \label{eq:target-rule}
%     \hat{y}_j \leftrightarrow \bigwedge_{i:\; r_{j i} = 1}{l_{i}}
% \end{equation}
% which reads as: (case a) if a concept is relevant (i.e. $r_{j i} = 1$), then it will appear with its sign in the conjunction; (case b) otherwise the concept is irrelevant and we can filter it out from the expression. As a result, in the example our  model can learn the rule $\hat{y}_{\text{banana}} \leftrightarrow \neg \hat{c}_{\text{round}} \wedge \hat{c}_{\text{yellow}}$ by learning $\mathbf{l} = [0.5,0.9,0.9]$ and $\mathbf{r}_{\text{banana}} = [0,1,1]$. The next section describes how we design a neural-symbolic architecture learning these rules.
% \todo{Stress more the difference between the rule and the way
% $f$ is implemented?}


\subsection{Rule Generation and Execution}
\label{sec:ruleexec}
Having defined the syntax of DCR rules, we describe how to \textit{generate} and \textit{execute} these rules in a differentiable way. To generate a rule we use two neural modules $\phi_j$ and $\psi_j$ which determine the role and relevance of each concept, respectively. Then, we execute each rule using the concepts' truth degrees of a given sample.
% In the previous section, we have seen how we can uniquely obtain a fuzzy rule predicting a class $\hat{y}_j$ for a certain sample $\mathbf{x}$. Here we show how we can execute this rule in a fully differentiable fashion, by exploiting two different neural  architectures $\phi_j$ and $\psi_j$ to determine the sign of the concept literals and which are the relevant ones, respectively. 
We split this process into three steps: (i) learning each concept's roles, (ii) learning each concept's relevance, and (iii) predicting the task using the relevant concepts.

% by reading the output of the two functions $\phi_j(\mathbf{c}_i)$ and  $\psi_j(\mathbf{c}_i)$, $\forall i$. In this section, we investigate how we can execute this rule symbolically on a given set of concept predictions $\{\hat{c}_i\}$. Notice that, at this stage, only the concept predictions $\{\hat{c}_i\}$ and not their embeddings $\{\hat{\textbf{c}}_i\}$ will be given as input to the rule execution process. This is a fundamental property to ensure full interpretability of the prediction.

% We compute these two indicators using two neural networks for each class $\hat{y}_j$.


% \paragraph{Conditional negation.} 
\paragraph{Concept role}
% \giu{Input/Output domains}
\underline{Generation:} To determine the \emph{role} (positive/negative) of a concept, we use a feed-forward neural network $\phi_j: \mathbb{R}^m \rightarrow [0,1]$, with $m$ being the dimension of each concept embedding.
% In order to self-arrange the sign of the concept activation provided as input to the function $f$ to predict a class $j$, we consider a feed-forward neural network $\phi_j: \mathbb{R}^m \rightarrow [0,1]$. 
The neural model $\phi_j$ takes as input a concept embedding $\hat{\mathbf{c}}_i \in \mathbb{R}^m$ and returns a soft indicator representing the role of the concept in the formula, that is, whether in literal $l_{ji}$ the concept should appear negated (e.g., $\phi_{\textit{banana}}(\hat{\mathbf{c}}_{\textit{round}}) = 0$)  or not (e.g., $\phi_{\textit{banana}}(\hat{\mathbf{c}}_{\textit{yellow}}) = 1$).
% of Example \ref{ex:banana}. 
% The conditional negation function $n(\hat{c}_i, \phi_j(\mathbf{c}_i)$  negates the concepts $\hat{c}_i$ based on the soft indicator computed by $\phi_j(\mathbf{c}_i)$.
\underline{Execution:} When we execute the rule, we need to compute the actual truth degree of a literal $l_{ji}$ given its role $\phi(\hat{\mathbf{c}}_i)$. We define this truth degree  $\ell_{ji} \in [0,1]$. In particular, we want to (i) forward the same truth degree of the concept, i.e.  $\ell_{ji} = \hat{c}_i$,  when $\phi(\hat{\mathbf{c}}_i)=1$,  and (ii) negate it, i.e. $\ell_{ji} = \neg \hat{c}_i$,  when $\phi(\hat{\mathbf{c}}_i)=0$.
% we evaluate the indicated \emph{role} $\phi_{j}(\mathbf{\hat{c}}_i)$ on the corresponding concept truth-value $\hat{c}_i$. 
% To this end, 
% we need to produce an object $\ell_{ji} \in [0,1]$ which models the concept literal $l_{ji}$, that is, $\ell_{ji} = \hat{c}_i$ if $\hat{c}_i=1$ and $\ell_{ji} = \neg \hat{c}_i$ if $\hat{c}_i=0$.
% Note that we only care when $\ell_{ji} = 1$, since the rule in Equation~\ref{eq:target-rule} predicts $\hat{y}_j=1$ \emph{if and only if} all the (relevant) literals $l_{ji}= 1$. We can then safely ignore when $\ell_{ji}= 0$ as the rule will not be in use. 
This behaviour can be generalized by a fuzzy equality $\Leftrightarrow $ when both $\phi_j$ and $\hat{c}$ are fuzzy values, i.e.: 
% \footnote{In the truth table, the significant rows are marked in green, i.e., where $\ell_{ji}=1$.}:
\begin{equation}
    \label{eq:iff}
     \ell_{ji} = (\phi_j(\hat{\mathbf{c}}_i) \Leftrightarrow \hat{c}_{i})
\end{equation}
% {\small
% \begin{center}
% \begin{tabular}{cc|c}
%      $\hat{c}_i$ & $\phi_j(\hat{\mathbf{c}})$ & $\ell_{ji}$ \\
%      \hline
%      \rowcolor{green!30}0 & 0 & 1 \\
%      \rowcolor{gray!30}0 & 1 & 0 \\
%      \rowcolor{gray!30}1 & 0 & 0 \\
%      \rowcolor{green!30}1 & 1 & 1 \\
% \end{tabular}
% \end{center}
% }
\begin{example}
    For a given object consider $\hat{c}_{\textit{round}}=0$ and $\phi_{\textit{banana}}(\hat{\mathbf{c}}_{\textit{round}})=0$. Then we get $\ell_{\textit{banana},\textit{round}}=(\phi_{\textit{banana}}(\hat{\mathbf{c}}_{\textit{round}}) \Leftrightarrow \hat{c}_{\textit{round}})=\neg \hat{c}_{\textit{round}}=1$. If instead we had $\phi_{\textit{banana}}(\hat{\mathbf{c}}_{\textit{round}}) = 1$, then $\ell_{\textit{banana},\textit{round}}=(\phi_{\textit{banana}}(\hat{\mathbf{c}}_{\textit{round}}) \Leftrightarrow \hat{c}_{\textit{round}})=0$.
\end{example}
% We can easily verify that when $\ell_{ji} \approx 1$ this equation approximates the concept literal $l_{ji}$ i.e., $\ell_{ji} \approx \hat{c}_i$ if $\hat{c}_i=1$ and $\ell_{ji} \approx \neg \hat{c}_i$ if $\hat{c}_i=0$:
% {\small
% \begin{center}
% \begin{tabular}{cc|c}
%      $\hat{c}_i$ & $\phi_j(\hat{\mathbf{c}})$ & $\ell_{ji}$ \\
%      \hline
%      \rowcolor{gray!50}0 & 0 & 1 \\
%      0 & 1 & 0 \\
%      1 & 0 & 0 \\
%      \rowcolor{gray!50}1 & 1 & 1 \\
% \end{tabular}
% \end{center}
% } 
% However, as we will show in our examples, this network tends to have very crisp values (i.e. either $0$ or $1$) at the end of the learning.  


% we will have $\ell_{ji}$ tending to the actual literal $l_{ji}$, i.e. $\ell_{ji}\approx \hat{c}_{i}$ if $\hat{c}_{i}\approx 1$ and $\ell_{ji}\approx \neg\hat{c}_{i}$ otherwise.

% In practice, we notice that when $\phi_j(\hat{\mathbf{c}}_i) \approx \hat{c}_{i}$ 

% and we will have $\ell_{ji}$ tending to the actual literal $l_{ji}$, i.e. $\ell_{ji}\approx \hat{c}_{i}$ if $\hat{c}_{i}\approx 1$ and $\ell_{ji}\approx \neg\hat{c}_{i}$ otherwise.

% To this end, we generate the object $\ell_{ji}\in[0,1]$ to.

% (e.g., $\phi_{banana}(\hat{\mathbf{c}}_{round}) = 0$ matches the concept truth-value $\hat{c}_{round} = 0$).
% by enforcing the correspondence between the indicator $\phi_{j}(\mathbf{\hat{c}}_i)$ and the truth-value $\hat{c}_i$, e.g. by using a logical equivalence that can be calculated according to the chosen fuzzy semantics:
% \begin{equation}
%     \label{eq:iff}
%      \ell_{ji} = \phi_j(\hat{\mathbf{c}}_i) \leftrightarrow \hat{c}_{i}
% \end{equation}
% \giu{This next two sentences are hard to follow (I understand them but readers may not).}The value $\ell_{ji}\in[0,1]$ indicates if $\phi_j$ recognizes as correct for the prediction $\hat{y}_j$ the truth-value $\hat{c}_i$ (i.e. $\ell_{ji}\approx 1$) or if it should be negated (i.e. $\ell_{ji}\approx 0$).


% As we will see from Equation \ref{eq:ruletarget}, having $l_{ji}\approx 1$ is fundamental to get a positive prediction $\hat{y}_j$, and the only other option is that the concept $i$ is irrelevant for the class $j$. Hence, the only interesting case is assuming $\ell_{ji}\approx 1$. As a rule of thumb, $\phi_j(\hat{\mathbf{c}}_i) \approx \hat{c}_{i}$ and we will have $\ell_{ji}$ tending to the actual literal $l_{ji}$, i.e. $\ell_{ji}\approx \hat{c}_{i}$ if $\hat{c}_{i}\approx 1$ and $\ell_{ji}\approx \neg\hat{c}_{i}$ otherwise. 
% The reader can easily verify in the table below that a conditional negation behave 
% \footnote{The negation is applied when $\phi_j(\mathbf{\hat{c}}) = 0$)}
% exactly like an if-and-only-if for crisp values:
% \begin{center}
% \begin{tabular}{cc|c}
%      $\hat{c}$ & $\phi_j(\hat{\mathbf{c}})$ & $\ell_{ji}$ \\
%      \hline
%      0 & 0 & 1 \\
%      0 & 1 & 0 \\
%      1 & 0 & 0 \\
%      1 & 1 & 1 \\
% \end{tabular}
% \end{center}
% Notice that,  also intermediate values are allowed, which is fundamental to learn such network. 
% However, as we will show in our examples, this network tends to have very crisp values (i.e. either $0$ or $1$) at the end of the learning.  


% \paragraph{Filtering}
\paragraph{Concept relevance.}
\underline{Generation:} To determine the \emph{relevance} of a concept $\mathbf{\hat{c}}_i$, we use another feed-forward neural network $\psi_j: \mathbb{R}^m \rightarrow [0,1]$. The model $\psi_j$ takes as input a concept embedding $\hat{\mathbf{c}}_i \in \mathbb{R}^m$ and returns a soft indicator representing the likelihood of a concept being relevant for the formula (e.g., $\psi_{\textit{banana}}(\hat{\mathbf{c}}_{\textit{soft}}) = 1$) or not (e.g., $\psi_{\textit{banana}}(\hat{\mathbf{c}}_{\textit{yellow}}) = 0$).
\underline{Execution:} When we execute the rule, we need to compute the truth degree of a literal given its relevance $r_{ji}$. We define the truth degree of a relevant literal as $\ell^r_{ji} \in [0,1]$, where $r$ stands for ``relevant''. In particular, we want to  \mbox{(i) filter} irrelevant concepts when $\psi_j(\hat{\mathbf{c}}_i) = 0$ by setting $\ell^r_{ji}=1$, and  \mbox{(ii) retain} relevant literals when $\psi_j(\hat{\mathbf{c}}_i) = 1$ by setting $\ell^r_{ji}= \ell_{ji}$. This behaviour can be generalized to fuzzy values of $\psi_j$ as follows: 
% This is simply $\ell^r_{ji} = \ell_{ji}$ if $r_{ji} =1$ and $\ell^r_{ji} = 1$  if $r_{ji} = 0$
%produce an object $\ell^\pi_{ji} \in [0,1]$ representing a relevant literal, that is, $\ell^\pi_{ji}\approx \ell_{ji}$ if the concept $i$ is relevant for class $j$. 
%Note that setting $\ell^\pi_{ji}=1$ makes the literal $\ell_{ji}$ irrelevant since ``$1$'' is neutral w.r.t.\ the conjunction in Equation~\ref{eq:ruletarget}. 

% We can then \mbox{(i) filter} irrelevant concepts out by setting the corresponding $\ell^\pi_{ji}=1$ when $\psi_j(\hat{\mathbf{c}}_i) = 0$, or \mbox{(ii) retain} relevant literals setting $\ell^\pi_{ji}= \ell_{ji}$ when $\psi_j(\hat{\mathbf{c}}_i) = 1$. For this reason we model $\ell^\pi_{ji}$ as follows:
% In particular  we need $\ell^\pi_{ji} \approx \ell_{ji}$ if the concept is relevant and $\ell_{ji} \approx \neg \hat{c}_i$ if $\hat{c}_i=0$.
% To establish the relevance of a concept, we use another feed-forward neural network $\psi_j: \mathbb{R}^m \rightarrow [0,1]$ predicting the likelihood of a concept being relevant, i.e. $r_{ji}$. This network takes as input any concept embedding $\hat{\mathbf{c}}_i \in \mathbb{R}^m$ and returns for each class $j \in [1,\dots,o]$ a degree of relevance for that concept (e.g. $\psi_{banana}(\hat{\mathbf{c}}_{soft}) = 0$ and $\psi_{banana}(\hat{\mathbf{c}}_{yellow}) = 1$).
% Therefore, given as input a literal $\ell_{ji}$, we define a filtered literal $\ell^\pi_{ji}$ to be as close as possible to $\ell_{ji}$ when $\psi_j(\hat{\mathbf{c}}_i) \approx 0$ (i.e. no filter), and to output a value close to $1$ when $\psi_j(\hat{\mathbf{c}}_i) \approx 0$ (i.e. filter).
% We implement the filtering function as a fuzzy material implication:
\begin{equation}~\label{eq:filter}
    \ell^r_{ji} = (\psi_j(\hat{\mathbf{c}}_i)\Rightarrow \ell_{ji})= (\neg \psi_j(\hat{\mathbf{c}}_i) \vee \ell_{ji})
\end{equation}

Note that setting $\ell^r_{ji}=1$ makes the literal $l_{ji}$ irrelevant since ``$1$'' is neutral w.r.t.\ the conjunction in Equation~\ref{eq:ruletarget}. 

% We invite the reader to verify that Equation~\ref{eq:iff} and~\ref{eq:filter} model the desired behavior (gray represents useless configurations and red conditions making a rule inactive):
% {\small
% \begin{center}
% % \begin{tabular}{cc|c}
% %      $\hat{c}_i$ & $\phi_j(\hat{\mathbf{c}})$ & $\ell_{ji}$ \\
% %      \hline
% %      \rowcolor{green!30}0 & 0 & 1 \\
% %      \rowcolor{gray!30}0 & 1 & 0 \\
% %      \rowcolor{gray!30}1 & 0 & 0 \\
% %      \rowcolor{green!30}1 & 1 & 1 \\
% % \end{tabular}
% % \qquad
% \begin{tabular}{cc|c}
%      $\psi_j(\hat{\mathbf{c}}_i)$ & $\ell_{ji}$ & $\ell^\pi_{ji}$ \\
%      \hline
%      \rowcolor{gray!30}0 & 0 & 1 \\
%      \rowcolor{gray!30}0 & 1 & 1 \\
%      \rowcolor{red!30}1 & 0 & 0 \\
%      \rowcolor{green!30}1 & 1 & 1 \\
% \end{tabular}
% \end{center}
% }

\begin{example}
    For a given object of type ``banana'', let the concept ``soft'' be irrelevant, that is $\psi_{\textit{banana}}(\hat{\mathbf{c}}_{\textit{soft}}) = 0$. Then we get $\ell^r_{\textit{banana},\textit{soft}}=(\psi_{\textit{banana}}(\hat{\mathbf{c}}_{\textit{soft}}) \Rightarrow \ell_{\textit{banana},\textit{soft}})=1$, independently from the content of $\hat{c}_{\textit{soft}}$ or $\ell_{\textit{banana},\textit{soft}}$. Conversely, let the concept ``yellow'' by relevant, that is $\psi_{\textit{banana}}(\hat{\mathbf{c}}_{\textit{yellow}}) = 1$, and let its concept literal be $\ell_{\textit{banana},\textit{yellow}}=\hat{\mathbf{c}}_{\textit{yellow}}=1$. As a result, we get
    $\ell^r_{\textit{banana},\textit{yellow}}=(\psi_{\textit{banana}}(\hat{\mathbf{c}}_{\textit{yellow}}) \Rightarrow \ell_{\textit{banana},\textit{yellow}})=1$.
    % Let instead the neural model provide the output $\phi_{banana}(\hat{\mathbf{c}}_{round}) = 1$, then the literal $\ell_{banana,round}=(\phi_{banana}(\hat{\mathbf{c}}_{round}) \leftrightarrow \hat{c}_{round})=0$.
\end{example}

% The default value for each $\psi_j$ is set to 1 because, when a literal is filtered, it becomes neutral w.r.t.\ the conjunction which will be applied immediately after to predict the class.


\paragraph{Task prediction}
Finally, we conjoin the relevant literals $\ell^r_{ji}$ to obtain the task prediction $\hat{y}_j$:
\begin{equation}
\label{eq:ruletarget}
    \hat{y}_j = \bigwedge_{i=1}^k \ell^r_{ji} 
\end{equation}
\begin{example}
    For a given object of type ``banana'', consider the following truth degrees for the concepts:  $\hat{c}_{soft}= 1, \hat{c}_{round} = 0,\hat{c}_{yellow} = 1$. Consider also the following values for the role and relevance for the class ``banana'': $\phi_{\textit{banana}}(\hat{\mathbf{c}}_{i})=[0,0,1]$ and $\psi_{\textit{banana}}(\hat{\mathbf{c}}_{i})=[0,1,1]$ for $i \in \{\textit{soft}, \textit{round}, \textit{yellow}\}$. Then, we obtain the final prediction for class $banana$ as: 
    \[
    \begin{array}{r}
    \hat{y}_{\textit{banana}} = \bigwedge_{i=1}^3\left(\neg \psi_{\textit{banana}}(\hat{\mathbf{c}}_i) \vee (\phi_{\textit{banana}}(\hat{\mathbf{c}}_i) \Leftrightarrow \hat{c}_{i})\right) =    \\
    =(1\vee(0\Leftrightarrow 1))\wedge (0\vee(0\Leftrightarrow 0))\wedge (0\vee(1\Leftrightarrow 1))=
    \\
    =(1\vee 0)\wedge (0\vee 1)\wedge (0\vee 1)=1 \wedge 1 \wedge 1 = 1
    \end{array}
    \]
\end{example}
We remark that the models $\phi_j$ and $\psi_j$: (a) generate fuzzy logic rules using concept embeddings which might hold more information than just concept truth degrees, and (b) do not depend on the number of input concepts which makes them applicable---without retraining---in testing environments where the set of concepts available differs from the set of concepts used during training.
We also remark that the whole process is differentiable as the neural models $\phi_j$ and $\psi_j$ are differentiable as well as the fuzzy logic operations as we will see in the next section.
% concepts $\hat{c}_i$ may represent fuzzy truth-degrees in $[0,1]$.

% \subsubsection{Rule Implementation}
\subsection{Rule Parsimony and Fuzzy Semantics}
\paragraph{Rule parsimony} 
Simple explanations and logic rules are easier to interpret for humans~\cite{miller1956magical,rudin2019stop}. We can encode this behaviour within the DCR architecture by enforcing a certain degree of competition among concepts to make only relevant concepts survive. To this end, we design a special activation function for the neural network $\psi_j$ rescaling the output of a log-softmax activation:
\begin{align}
    \gamma_{ji} &= \log \Bigg( \frac{\exp(\text{MLP}_j(\hat{\mathbf{c}}_i))}{\sum_{i^\prime=1}^k \exp(\text{MLP}_j(\hat{\mathbf{c}}_{i^\prime}))} \Bigg) 
    % \ ,\ i=[1,\dots,k] 
    \label{eq:comp}\\
    r_{ji} &= \psi_j(\hat{\mathbf{c}}_i) = \sigma \Bigg(\gamma_{ji} - \frac{1}{k} \sum_{i^\prime=1}^k \gamma_{ji^\prime} \Bigg)\label{eq:rel}
\end{align}
% The log-softmax function which enforces a soft competition for concept survival for each class $j$. Rescaling the log-softmax scores $\gamma_{ji}$ to have zero-mean. 
This way, if the scores $\gamma_{ji}$ are uniformly distributed, then we expect the network $\psi_j$ to select half of the concepts. 
% that is, $\mathbb{E}[r_{ji} \geq 0.5 | \mathcal{U}(\gamma_{ji})] = 0.5$. 
% This way we obtain a behaviour which models Equations~\ref{eq:target-irrelevance} and~\ref{eq:target-expect} as it will be hard for the model to set all values to $1$ (all concepts are relevant) or $0$ (all concepts are irrelevant). 
We can also parametrise this function by introducing a parameter $\tau \in [-\infty, \infty]$ that allows a user to bias the default behaviour of the activation function: 
% \begin{equation}
% \label{eq:relfin}
$r_{ji} = \sigma (\gamma_{ji} - \frac{\tau}{k} \sum_{i^\prime=1}^k \gamma_{ji^\prime} )$.
% \end{equation}
A user can increase $\tau$ to get more relevance scores closer to $1$ (more complex rules) or decrease it to get more relevance scores closer to $0$ (simpler rules).











%We design our architecture interconnecting both neural and symbolic functions to obtain for each sample an interpretable logic rule such as the one described in the previous paragraph. The architecture is composed of: a concept relevance learner to model Equation~\ref{eq:target-irrelevance}, a concept literal learner to model Equation~\ref{eq:target-literal}  and a logic aggregator to model Equation~\ref{eq:target-rule}. We discuss how to make gradient flow through logic operations in Section~\ref{sec:semirings}. \giu{I would not anticipate the gradient issue here (confusing); and I am not sure we really want to fix a specific semiring in the paper. I really consider this an implementation issue akin to an hyperparam.}

%\paragraph{Learning the relevant concept literal}
%\todo{to say that g can be learnable or given.. and so the $\hat{c}_i$}
%\giu{consider removing the word "relevant" here. We focus on learning the f that means learning $\phi_j,\psi_j$ etc}
%The first part of our architecture aims at learning relevant concept literals. To this goal we use a two-steps approach. (\textbf{step 1}) First, we use  a feed-forward neural network $\phi_j: \mathbb{R}^k \rightarrow [0,1]$ that takes as input a concept embedding $\hat{\mathbf{c}}_i \in \mathbb{R}^k$ and returns,whether the concept should be negated (i.e. $\phi_j(\hat{\mathbf{c}}_i) = 0$)  or not (i.e. $\phi_j(\hat{\mathbf{c}}_i) = 1$) for the class $y_j$..  
%This network takes as input  and returns for each class $j \in [1,\dots,l]$ a guess of the relevant concept literal. 
% In our experiments, we implement the model $\phi_j$ as a multi-layer perceptron with leaky-ReLU activations in the hidden layers and a sigmoid activation on the output layer i.e., $\phi_j(\hat{\mathbf{c}}_i) := \sigma(\text{MLP}_j(\hat{\mathbf{c}}_i))$.
% (\textbf{step 2}) We then model Equation~\ref{eq:target-literal} by comparing the guessed concept literal $\phi(\hat{\mathbf{c}}_i)$ with the concept truth degree $\hat{c}_{i}$:
% \todo{we call this an interpretation of the literal}
% \begin{equation} \label{eq:iff}
%     l_{ji} = (\phi_j(\hat{\mathbf{c}}_i) \iff \hat{c}_{i})
% \end{equation}
% \todo{F: This Eq should be used to align the concept-embeddings to labels or concept-predictions? Who is $\hat{c}_i$ not bold?}
% This way we obtain a behaviour which models Equation~\ref{eq:target-literal}: (case 1) $l_{ji} \rightarrow \hat{c}_i$ when the guessed literal $\phi_j(\hat{\mathbf{c}}_i)$ is correct and its predicted logic state $\hat{c}_{i}$ is ``truth''; (case 2) $l_{ji} \rightarrow \neg \hat{c}_i$ when the guessed literal $\phi_j(\hat{\mathbf{c}}_i)$ is correct and its predicted logic state $\hat{c}_{i}$ is ``false''; (case 3) $l_{ji} \rightarrow 0$ when the guessed literal is incorrect i.e., it does not match the predicted logic state.


% \paragraph{Identifying relevant concepts}
% The second part of our architecture aims at identifying relevant and irrelevant concepts. To this end we make a feed-forward neural network $\psi_j: \mathbb{R}^k \rightarrow [0,1]$ predict the likelihood of a concept being relevant: This network takes as input any concept embedding $\hat{\mathbf{c}}_i \in \mathbb{R}^k$ and returns for each class $j \in [1,\dots,l]$ a probability of a concept being relevant. In our experiments, we implement the model $\psi_j$ as a multi-layer perceptron with leaky-ReLU activations in the hidden layers. 
% \paragraph{Conjoin relevant concepts}
% The final part of our architecture aims at generating a logic rule to make a prediction for a given class using relevant concepts only. To this end, we remove unimportant concepts and then conjoin the literal $l_{ji}$ of the others. One way to remove unimportant concepts is to make their concept literal $l_{ji}$ irrelevant. We can now recall that the structure of our logic rule is a conjunction of terms. As the neutral element of the conjunction is $1$, we can simply set to $1$ all unimportant concepts thus making their presence irrelevant in the conjunction. We can model this behaviour by disjoining concept literals and \textit{irrelevant} scores. We can now safely conjoin all concepts as unimportant ones are now irrelevant to the prediction: 
% \begin{equation} \label{eq:rule}
%     \hat{y}_j = \bigwedge_i (l_{ji} \vee \neg r_{ji})
% \end{equation}
% This way we obtain a behaviour which models Equation~\ref{eq:target-rule}: (case 1) if a concept is relevant then $r_{ji} \rightarrow 1$ which makes $\neg r_{ji} \rightarrow 0$ which in turn makes the disjunction $(l_{ji} \vee \neg r_{ji}) \rightarrow l_{ji}$; (case 2) if a concept is irrelevant then $r_{ji} \rightarrow 0$ which makes $\neg r_{ji} \rightarrow 1$ which in turn makes the disjunction $(l_{ji} \vee \neg r_{ji}) \rightarrow 1$ making the concept literal irrelevant in the final conjunction.


\paragraph{Fuzzy semantics}
% \paragraph{Neural-Symbolic Semirings}
\label{sec:semirings}
To create a semantically valid model, we enforce the same semantic structure in all logic and neural operations. 
Moreover, to train our model end-to-end, we need this semantics to be differentiable in all its operations, including logic functions. \citet{marra2020lyrics} describe a set of possible t-norm fuzzy logics which can serve the purpose. In our experiments, we use the G\"odel t-norm. With this semantics, we can rewrite Equation~\ref{eq:iff} as:
\[
\begin{array}{ll}
    \ell_{ji} & =\phi_j(\hat{\mathbf{c}}_i)\Leftrightarrow \hat{c}_i =  (\phi_j(\hat{\mathbf{c}}_i)\Rightarrow \hat{c}_i)\wedge (\hat{c}_i\Rightarrow \phi_j(\hat{\mathbf{c}}_i)) = \nonumber\\   
    &=(\neg\phi_j(\hat{\mathbf{c}}_i)\vee \hat{c}_i)\wedge (\neg \hat{c}_i\vee \phi_j(\hat{\mathbf{c}}_i)) = \nonumber\\   
    &=\min\{\max\{1-\phi_j(\hat{\mathbf{c}}_i), \hat{c}_{i}\}, \max\{1-\hat{c}_{i},\phi(\hat{\mathbf{c}}_i)\}\}
\end{array}
\]
and Equation~\ref{eq:ruletarget} as: $
% \begin{equation*}
    \hat{y}_j = \min_{i=1}^k\{\max\{1 - \psi_j(\hat{\mathbf{c}}_i), \ell_{ji}\}\}
% \end{equation*}
$

\subsection{Global and counterfactual explanations}
\label{sec:ruleadv}

\paragraph{Interpreting global behaviour}
% Our model builds a simple parameterized fuzzy rule for each sample and it uses this rule to predict the final class. 
In general, DCR rules may have different weights and concepts for different samples. However, we can still globally interpret the predictions of our model without the need for an external post-hoc explainer. To this end, we collect a batch of (or all) fuzzy rules generated DCR on the training data $\mathcal{X}_{\text{train}}$. Following~\citet{barbiero2022entropy}, we then Booleanize the collected rules and aggregate them with a global disjunction to get a single logic formula valid for all samples of class $j$:
\begin{equation} \label{eq:global-explanation}
    \hat{y}^C_j = \bigvee_{\mathbf{x} \in \mathcal{X}_{\text{train}}} \hat{y}_j(\mathbf{x})
    % \bigwedge_{i \in \{i^\prime \; | \; r_{j i^\prime q} = 1 \} }{\mathbb{I}_{s_{jiq} \geq 0.5}}
\end{equation}
This way we obtain a global overview of the decision process of our model for each class.

% \subsection{Causality}
% \paragraph{Concept Interventions} As is the case for Concept Bottleneck Models~\cite{koh2020concept,zarlenga2022concept}, DCR is receptive to test-time concept interventions where an expert can improve the performance of our model by correcting mispredicted concepts during inference. This can be done by allowing experts to examine the concept probability $\hat{c}_i$ of a specific concept being fed into DCR and manually correcting it by setting it to $\hat{c}_i := 1$ if they believe the concept represented by embedding $\hat{c}_i$ is on or setting it to $\hat{c}_i := 0$ otherwise. This change is then propagated into our DCR module, possibly triggering a change in the prediction made by DCR before such an intervention was performed and increasing its accuracy. These sorts of interventions allow DCR to be able to be deployed in a human-in-the-loop setup where its performance can be drastically improved with the help of experts that can correct mispredicted concept probabilities given to DCR at training time.


\paragraph{Counterfactual explanations}
Logic rules clearly reveal which concepts play a key role in a prediction. This transparency, typical of interpretable models, facilitates the extraction of simple counterfactual explanations without the need for an external algorithm as in~\citet{abid2021meaningfully}. 
In DCR we extract simple counter-examples $x^\star$ using the logic rule as guidance. Following~\citet{wachter2017counterfactual}, we generate counter-examples as close as possible to the original sample $|x - x^\star|< \epsilon$. In particular, \citet{wachter2017counterfactual} proposes to perturb the input features of a model starting from the most relevant features. As the decision process depends mostly on the most relevant features, perturbing a small set of features is usually enough to find counter-examples. To this end, we first rank the concepts present in the rule according to their relevance scores. Then, starting from the most relevant concept, we invert their truth value until the prediction of the model changes. The new rule represents a counterfactual explanation for the original prediction.

% Our model makes local inferences by learning a simple logic rule. This logic rule can be seen as a hypothetical causal graph linking concepts to tasks. We can then play directly with this causal graph to intervene on concepts and/or on rule weights. Each modification of the rule is necessary and sufficient to obtain a specific prediction (as opposed to a black-box predictor). 
% This allows us to generate counterfactuals by design (causality level-3 according to Pearl\todo{Missing citation?}). 
% In Figure \ref{fig:counterfactual_trig}, we test the capability of providing counterfactual explanations. Counter-examples $x^\star$ are crafted following the provided explanation, while remaining . 


% \paragraph{Rule Interventions}
% As LENs may infer incorrect rules from data, human experts can intervene on the structure of the rules, deleting or adding new terms to the rule to align with their expectations.

% \paragraph{Prior Knowledge Injection}
% We just need to compile prior knowledge in form of logic rules and add those rules to the set of rules learnt from LENs.


\section{Experiments}
\label{sec:dcr-exp}

\subsection{Research Questions}
In this section, we analyze the following research questions:
\begin{itemize}
    \item \textbf{Generalization ---} How does DCR generalize on unseen samples compared to interpretable and neural-symbolic models? How does DCR generalize when concepts are unsupervised?
    \item \textbf{Interpretability ---} Can DCR discover meaningful rules? Can DCR re-discover ground-truth rules? How stable are DCR rules under small perturbations of the input compared to interpretable models and local post-hoc explainers?
    % \item \textbf{Causality ---} 
    % Are concept interventions effective on DCR? 
    How long does it take to extract a counterfactual explanation from DCR compared to a non-interpretable model?
\end{itemize}
% Moreover, we are specifically interested in evaluating DCR with respect to state-of-the-art interpretable and black-box models in real-world extreme settings where concept supervisions are not available for training or some of the training concepts are not available at test time.


\begin{figure*}[t]
    \centering
%    \includegraphics[width=0.98\linewidth]{figs/results_auc.pdf}
    \caption{Mean ROC AUC for task predictions for all baselines across all tasks (the higher the better). DCR often outperforms interpretable concept-based models. \emph{CE} stands for concept embeddings, while \emph{CT} for concept truth degrees. Models trained on concept embeddings are not interpretable as concept embeddings lack a clear semantic for individual embedding dimensions.
    % We report the mean and $95\%$ CI of the task's ROC AUC.
    %\todo{Change to Concept Embedding+X and Concept Scores+X because we don't always use CEM or CBM}
    }
    \label{fig:accuracy}
\end{figure*}


\subsection{Experimental Setup}
\paragraph{Data \& task setup}
We investigate our research questions using six datasets spanning three of the most common data types used in deep learning: tabular, image, and graph-structured data.
We use the three benchmark datasets (\textit{XOR}, \textit{Trigonometry}, and \textit{Dot}) proposed by~\citet{zarlenga2022concept} as they capture increasingly complex concept-to-label relationships, therefore challenging concept-based models. To test the DCR's ability to re-discover ground-truth rules we use the \textit{MNIST-Addition} dataset~\cite{manhaeve2018deepproblog}, a standard benchmark for neural-symbolic systems where one aims to predict the sum of two digits from the MNIST's dataset. 
% To further increase the complexity of this task, we train DCR end-to-end with a convolutional network without supervising concepts corresponding to digits. 
Furthermore, we evaluate our methods on two real-world benchmark datasets: the Large-scale CelebFaces Attributes (\emph{CelebA},~\citep{liu2015deep}) and the \emph{Mutagenicity}~\cite{morris2020tudataset} dataset. In particular we define a new \emph{CelebA} task to simulate a real-world condition of concept ``shifts'' where train and test concepts are correlated (e.g., ``beard'' and ``mustaches'') but do not match exactly. To this end, we split the set of \emph{CelebA} attributes defined by~\citet{zarlenga2022concept} in two partially disjoint sets and use one set of attributes for training models and one for testing.
Finally, we use \emph{Mutagenicity} as a real-world scenario the concept encoder is unsupervised.
As \emph{Mutagenicity} does not have concept annotations, we first train a graph neural network (GNN) on this dataset, and then we use the Graph Concept Explainer (GCExplainer, ~\cite{magister2021gcexplainer}) to extract a set of concepts from the embeddings of the trained GNN.
For dataset with concept labels instead, we generate concept embeddings and truth degrees by training a Concept Embedding Model~\cite{zarlenga2022concept}.
% We then train DCR on the discovered concepts and evaluate the correctness of a DCR rule indirectly by checking whether rules use concepts corresponding to functional groups known for their harmful effects.
Further details on these datasets and their properties are provided in Appendix~\ref{app:datasets}.


\paragraph{Baselines}
We compare DCR against interpretable models, such as logistic regression~\cite{verhulst1845resherches}, decision trees~\citep{breiman2017classification}, as well as state-of-the-art black-box classifiers, such as extreme gradient boosting (\mbox{XGBoost})~\citep{chen2016xgboost}, and locally-interpretable neural models, such as the Relu Net \citep{ciravegna2023logic}. 
% The latter is a deep neural network-based model, providing logic explanations of its prediction. 
We train all baseline models in two different conditions mapping concepts to tasks either using concept truth degrees or using concept embeddings (baselines marked with \emph{CT} and \emph{CE} in figures, respecitvely). We consider interpretable only baselines trained on concept truth degrees only, as concept embeddings lack of clear semantics assigned to each dimension. However, baselines trained on concept embeddings still provide a strong reference for task accuracy w.r.t. interpretable models.
% \mbox{(1)} extracted concept embeddings (i.e., embeddings learnt by CEMs~\cite{zarlenga2022concept}) to the output label, and \mbox{(2) by} training each model to map concept truth degrees to the task's output label. 
% We remark that because specific dimensions of a concept embedding lack known semantics, decision trees trained on these embedding will not be fully interpretable.
On the \emph{MNIST-Addition} dataset we compare DCR with state-of-the-art neural-symbolic baselines including: DeepProbLog \cite{manhaeve2018deepproblog}, DeepStochLog \cite{winters2022deepstochlog}, Logic Tensor Networks \cite{badreddine2022logic}, and Embed2Sym \cite{aspis2022embed2sym}. This is possible as the \emph{MNIST-Addition} dataset provides access to the full set of ground-truth rules, allowing us to train these neural-symbolic systems.
% When evaluating concept interventions, we compare DCR against Concept Bottleneck Models (CBMs)~\cite{koh2020concept} with sigmoidal bottlenecks and against Concept Embedding Models (CEMs)~\cite{zarlenga2022concept}.
Finally, we compare DCR interpretability with interpretable models, such as logistic regression and decision trees, and with local post-hoc explainers, such as the Local Interpretable Model-agnostic Explanations (LIME,~\cite{ribeiro2016should}) applied on XGBoost. 
% Finally, following~\cite{wachter2017counterfactual}, we compare counterfactual examples extracted from DCR counterfactuals extracted from interpretable models and with LIME applied on XGBoost.
% \todo{x sensitivity and counterfactuals ADD: We compared with the interpretable methods used in Figure \ref{fig:accuracy} with the addition of XGBoost when explained by LIME ('Lime') \citep{ribeiro2016should} and a recently proposed interpretable neural network ('XReluNN') \citep{ciravegna2023logic}.}


% As some of these models are not easily differentiable (e.g., decision trees), end-to-end training might be unfeasible or more difficult w.r.t.\ DCR. To avoid this bias, we train all models using a sequential training~\citep{koh2020concept} using exactly the same input. As a further comparison, we train each baseline model once on concept embeddings and once on concept truth degrees only. 
% The first experimental condition allows us to compare the generalisation of different models when they have access to the same amount of information. However, in this setting, all models but DCR are black-boxes as their inference depend on the embeddings. For example, a decision tree could learn the formula ``male $\leftarrow$ bald$>0.5$ $\wedge$ $e_{[\text{long-hair},3]}<32.1$'' where $e_{[\text{long-hair},3]}$ is the third dimension of the embedding of concept ``long-hair''. This will never happen with DCR as it uses concept embeddings to compute attention scores only. For this reason, we also train all baselines on concept truth degrees only as in this scenario we can consider them as white boxes. We consider XGBoost as a black-box despite being a tree-based classifier its inference for a single sample is a highly non-linear function of the input.



% % Please add the following required packages to your document preamble:
% % \usepackage{graphicx}
% \begin{table*}[!t]
% \centering
% \caption{Here we need a table showing examples of how our model makes predictions. Our model generates for each sample a weighted fuzzy rule whose literals are concepts. So, for each sample our model is interpretable and does not require to any explanation.}
% \label{tab:global-rules}
% \resizebox{\textwidth}{!}{%
% \begin{tabular}{lllll}
% \toprule
%  &
% \multicolumn{1}{l}{\textbf{\textsc{XOR}}} & 
% \multicolumn{1}{l}{\textbf{\textsc{Trigonometry}}} & 
% % \multicolumn{1}{l}{\textbf{\textsc{Vector}}} & 
% \multicolumn{1}{l}{\textbf{\textsc{BAGraph}}} & 
% \multicolumn{1}{l}{\textbf{\textsc{MNIST addition}}} \\ 
% \midrule
% Ground-truth rule & 
% $y_1 \leftarrow \neg c_0 \wedge c_1$ & 
% $y_1 \leftarrow \neg c_0 \wedge \neg c_1 \wedge \neg c_2$ & 
% % $y_0 \leftarrow c_0 \wedge c_1$ & 
% & $y_{13} \leftarrow c^{first}_9 \land  c^{second}_4$\\
% Example of DCR rule & 
% $y_1 \leftarrow 1.0_{\neg c_0} \wedge 1.0_{c_1}$ & 
% $y_1 \leftarrow 1.0_{\neg c_0} \wedge 1.0_{\neg c_1} \wedge 1.0_{\neg c_2}$ & 
% % $y_0 \leftarrow 1.0_{c_0} \wedge 1.0_{c_1}$ & 
% & $y_{13} \leftarrow 1.0_{c^{first}_9} \land  1.0_{c^{second}_4}$\\
% \bottomrule
% \end{tabular}%
% }
% \end{table*}
% \begin{figure*}[t]
%     \centering
%     \todo{}
%     % \includegraphics[width=0.24\linewidth]{figs/explanations/id_0_explanation.png}
%     % \includegraphics[width=0.24\linewidth]{figs/explanations/id_69_explanation.png}
%     % \includegraphics[width=0.24\linewidth]{figs/explanations/id_144_explanation.png}
%     % \includegraphics[width=0.24\linewidth]{figs/explanations/id_18_explanation.png}
%     \caption{Here we need an image showing examples of how our model makes predictions. Our model generates for each sample a weighted fuzzy rule whose literals are concepts. So, for each sample our model is interpretable and does not require to any explanation.}
%     \label{fig:exact_explanations}
% \end{figure*}


\paragraph{Evaluation}
We assess each model's performance and interpretability based on four criteria. First, we measure task generalization using the Area Under the Receiver Operating Characteristic Curve (ROC AUC) from prediction scores~\cite{hand2001simple} (the higher the better).
% We only consider task generalization as we evaluate all methods on the same concept representations, thus ruling out this confounding factor in our comparison. 
Second, we evaluate DCR interpretability by comparing the learnt logic formulae with ground-truth rules in \emph{XOR}, \emph{Trigonometry}, and \emph{MNIST-Addition} datasets, and indirectly on \emph{Mutagenicity} by checking whether the learnt rules involve concepts corresponding to functional groups known for their harmful effects, as done by~\citet{ying2019gnnexplainer}. Third, to further assess interpretability, we measure 
% the effect of concept interventions on task performance~\cite{koh2020concept} (the higher the better) and
the sensitivity of the predictions under small perturbations following~\citet{yeh2019fidelity} (the lower the better). Finally, we measure how receptive our model is to extracting meaningful counterfactual examples from its rules by computing the number of concept perturbations required to obtain a counterfactual example following~\citet{wachter2017counterfactual} (the lower the better).
For each metric, %unless otherwise specified, 
we report their mean and 95\% Confidence Intervals (CI) on our test sets using $5$ different initialization seeds.
% To evaluate counterfactual explanations, we compare how many perturbations are necessary to make DCR change its prediction w.r.t.\ interpretable models and LIME applied on XGBoost.
% \todo{add for counterfactuals: We compute the reduction of confidence of the originally predicted class on the counter-example $f_(x^\star)$ as a function of the number of modified features.}

\subsection{Task Generalisation}
\paragraph{DCR outperforms interpretable models (Figure~\ref{fig:accuracy})} 
Our experiments show that DCR generalizes significantly better than interpretable benchmarks in our most challenging datasets. This improvement peaks when concept embeddings hold more information than concept truth degrees, as in  the \emph{CelebA} and \emph{Dot} tasks where this deficit of information is imposed byconstruction~\cite{zarlenga2022concept}. This grants DCR a significant advantage (up to $\sim 25\%$ improvement in ROC-AUC) over the other interpretable baselines. This phenomenon confirms the findings by~\citet{mahinpei2021promises} and~\citet{zarlenga2023towards}. In particular, the concept shift in \emph{CelebA} causes interpretable models to behave almost randomly as the set of test concept is different from the set of train concepts (despite being correlated). DCR however still generalizes well as the mechanism generating rules only depends on concept embeddings and the embeddings hold more information on the correlation between train and test concepts w.r.t. concept truth degrees.
% because concept embeddings may hold relevant information to solve unrelated tasks. 
To further test this hypothesis, we compare DCR against XGBoost, decision trees (DTs), and logistic regression trained on concept embeddings. In most cases, concept embeddings allow DTs and logistic regression to improve task generalization, but the predictions of such models are no longer interpretable. In fact, even a logic rule whose terms correspond to dimensions of a concept embedding is not semantically meaningful as discussed in Section~\ref{sec:back}. In contrast, DCR uses concept embeddings to assemble rules whose terms are concept truth degrees, which makes it possible to keep the rules semantically meaningful.
% Neural-Symbolic Concept Reasoning outperforms even state-of-the-art classifiers such as XGBoost in out-of-distribution tasks where the number of concepts available at train and test time are different (up to $+25\%$ test AUC on CelebA). DCR achieves this as the model does not depend on the number of concepts used at training time, as opposed to some of the most common machine learning classifiers. This allows DCR to work in environments where the number of concepts is different without retraining.

\paragraph{DCR matches the accuracy of neural-symbolic systems trained using human rules (Table~\ref{tab:mnist-addition-accuracy})}
% Neural-symbolic systems using human rules represent the gold standard to benchmark rule learners.
Our experiments show that DCR generates rules that, when applied, obtain accuracy levels close to neural-symbolic systems trained using human rules, currently representing the gold-standard to benchmark rule learners.
% Our experiments show that DCR induces rules almost as accurate as neural-symbolic systems using human rules. 
We show this result on the \emph{MNIST-Addition} dataset~\cite{manhaeve2018deepproblog}, a standard benchmark in neural-symbolic AI, where the labels on the concepts are not available. We learn concepts without supervision by adding another task classifier, which only uses very crisp $\hat{c}_i$ to make the task predictions (see Appendix \ref{app:mnist}). 
DCR achieves similar performance to state-of-the-art neural-symbolic baselines (within $1\%$ accuracy from the best baseline). However, DCR is the only system discovering logic rules directly from data, while all the other baselines are trained using ground-truth rules. Therefore, this experiment indicates how DCR can learn meaningful rules also without concepts supervision while still maintaining state-of-the-art performance.
% In fact, we observe that if we increase the complexity of the task even further, by not providing any concept supervision during training time to DCR, our model is still able to learn meaningful rules in an unsupervised manner.
% Please add the following required packages to your document preamble:
% \usepackage{graphicx}
\begin{table}[]
\centering
\caption{Task accuracy on the \emph{MNIST-addition} dataset. The neural-symbolic baselines use the knowledge of the symbolic task %(i.e., the addition) 
to distantly supervise the image recognition task. DCR achieves similar performances even though it learns the rules from scratch.}
\label{tab:mnist-addition-accuracy}
\resizebox{0.7\columnwidth}{!}{%
\begin{tabular}{ll}
\hline
\textbf{\textbf{\textsc{Model}}} & \textbf{\textbf{\textsc{Accuracy} (\%)}} \\ \hline
\multicolumn{2}{c}{With ground truth rules} \\
DeepProbLog & $97.2 \pm 0.5$ \\
DeepStochLog & $97.9 \pm 0.1$ \\
Embed2Sym & $97.7 \pm 0.1$ \\
LTN & $98.0 \pm 0.1$ \\ \hline
\multicolumn{2}{c}{Without ground truth rules} \\
DCR(ours) & $97.4 \pm 0.2$ \\ \hline
\end{tabular}%
}
\end{table}
% with an unsupervised criterion to learn meaningful concepts distributions. In this paper, we used a simple solution, where tasks are predicted from a peaked distributions on the concepts obtained through a Gumbel softmax. 
% Neural-Symbolic Concept Reasoning attains high prediction performances and logic rules even when concept labels are not available at training time. \todo{discuss experiments on MNIST addition (relational) and graph classification}
% In particular, in absence of supervisions on the concepts, we can couple DCR with an unsupervised criterion to learn meaningful concepts distributions. In this paper, we used a simple solution, where tasks are predicted from a peaked distributions on the concepts obtained through a Gumbel softmax. 
% DCR achieves similar performance to state-of-the-art Neural Symbolic frameworks, despite being the only system without access to the underling symbolic rules.  On the contrary, DCR induces them automatically from data.  
% \todo{move the following to experimental set up}
% We show the results of our experiments on the MNIST Addition task \cite{manhaeve2018deepproblog} in Table \ref{tab:mnist-addition}, where we compare  DCR with the following Neural-Symbolic baselines: DeepProbLog \cite{manhaeve2018deepproblog}, DeepStochLog \cite{winters2022deepstochlog}, Logic Tensor Networks \cite{badreddine2022logic} and Embed2Sym \cite{aspis2022embed2sym}. 
% \begin{table}[t]
%     \centering
%     \label{tab:mnist-addition-accuracy}
%     \caption{Task accuracy on the \emph{MNIST-addition} dataset. The neural-symbolic baselines use the knowledge of the symbolic task (i.e., the addition) to distantly supervise the image recognition task. DCR achieves similar performances even though it has to learn the rules from scratch.}
%     % \vspace{0.2cm}
%     \begin{tabular}{ll}
%         \hline
%         \textbf{\textsc{Model}} & \textbf{\textsc{Accuracy} (\%)} \\
%         \hline
%         \multicolumn{2}{c}{\textit{With ground truth rules}} \\
%         DeepProbLog & $97.2 \pm 0.5$ \\
%         DeepStochLog & $97.9 \pm 0.1$ \\
%         Embed2Sym & $97.7 \pm 0.1$ \\
%         LTN  & $98.0 \pm 0.1$ \\
%         \hline
%         \multicolumn{2}{c}{\textit{Without ground truth rules}} \\
%         DCR(ours) & $97.4 \pm 0.2$ \\
%         \hline
%     \end{tabular}
% \end{table}
% \begin{table*}[]
%     \centering
%     \caption{Mean accuracy ($\pm$ standard deviation) on the MNIST addition dataset. The neural-symbolic baselines use the knowledge of the symbolic task (i.e. the addition) to distantly supervise the image recognition task. DCR achieves similar performances even though it has to learn the rules during the process.}
%     \label{tab:mnist-addition}
%     \begin{tabular}{l|llll|l}
%         \hline
%         & \multicolumn{4}{c}{\textit{With symbolic rules}} &
%         {\textit{Without symbolic rules}} \\
%         \textbf{\textsc{Model}} & DeepProbLog & DeepStochLog & Embed2Sym & LTN & DCR(ours) \\
%         \hline
%          \textbf{\textsc{Accuracy}} & $0.972 \pm 0.005$ & $0.979 \pm 0.001$ & $0.977 \pm 0.001$ & $0.980 \pm 0.001$ & $0.971 \pm 0.002$
%     \end{tabular}
%     \label{tab:my_label}
% \end{table*}
\subsection{Interpretability}
\paragraph{DCR discovers semantically meaningful logic rules (Table~\ref{tab:global-rules})}
Our experiments show that DCR induces logic rules that are both accurate in predicting the task and formally correct when compared to ground-truth logic rules. We evaluate the formal correctness of DCR rules on the \emph{XOR}, \emph{Trigonometry}, and \emph{MNIST-Addition} datasets where we have access to ground-truth logic rules. We report a selection of Booleanized DCR rules with the corresponding ground truth rules in Table~\ref{tab:global-rules}. 
Our results indicate that DCR's rules align with human-designed ground truth rules, making them highly interpretable. For instance, DCR predicts that the sum of two MNIST digits is $17$ if either the first image is a 
%\includegraphics[scale=0.4]{figs/img_9.jpg} (i.e., $c'_9$) 
``9''
and the second is an 
%\includegraphics[scale=0.4]{figs/img_8.jpg} (i.e., $c''_8$)
``8''
 or vice-versa which we can interpret globally using Equation~\ref{eq:global-explanation} as: $y_{17} \Leftrightarrow (c'_9 \land  c''_8) \vee (c'_8 \land  c''_9)$. We list all logic rules discovered by DCR on the \emph{MNIST-Addition} dataset in Appendix~\ref{app:mnist}.
It is interesting to investigate the potential of DCR also in settings where we do not have access to the ground-truth logic rules, such as the \emph{Mutagenicity} dataset. Here, unlike the \textit{MNIST addition} dataset, not only
there is no supervision on the concepts, but we don't even know which are the concepts.
We use GCExplainer~\cite{magister2021gcexplainer} to generate a set of concepts embeddings from the embeddings of a trained GNN. We then use these embeddings to train DCR.
% To test this, we use GCExplainer's concepts extracted from the embeddings of a GNN trained on the \emph{Mutagenicity} dataset to train a DCR model on the same task.
% following a similar procedure~\citet{ghorbani2019interpretation} proposed for convolutional neural networks.
In this setting, we can only evaluate the correctness of a DCR rules indirectly by checking whether the concepts appearing in the rules correspond to functional groups known for their harmful effects within the \emph{Mutagenicity} dataset following~\citet{ying2019gnnexplainer}. Interestingly, many of DCR's rules predicting mutagenic effects include functional groups such as phenols~\cite{hattenschwiler2000role} and dimethylamines~\cite{acgih2016american}, which can be highly toxic when combined in molecules such as \mbox{3-Dimethylaminophenols}~\cite{sabry2011synthesis}. This suggests that DCR has potential to unveil semantically meaningful relations among concepts and to make them explicit to humans by means of the learnt rules. 
%The experiments on \emph{Mnist-Addition} and \emph{Mutagenicity} also demonstrate how DCR can make these relations explicit and semantically meaningful even when concepts are unknown during training. 
We provide experimental details with the full list of concepts and rules discovered in \emph{Mutagenicity} in Appendix~\ref{app:mutag}.
% Indeed, DCR predicts each sample using a weighted fuzzy rule whose literals are the learnt concepts. 
 % As in other interpretable models, such as logistic regression and decision trees, DCR infers the prediction using an interpretable model which makes the use of local explainers such as LIME~\cite{ribeiro2016should} superfluous/obsolete.
% \paragraph{DCR global behavior is self-explainable (Table~\ref{tab:global-rules})}
% Our experiments show that the global behaviour of DCR is self-explainable. Indeed, The global behaviour of DCR is coherent and matches the ground-truth expected behaviour as shown in Figure~\ref{tab:global-explanations}. Here similar local logic expressions are aggregated to provide a global overview of DCR reasoning which matches ground-truth expressions.


% Please add the following required packages to your document preamble:
% \usepackage{graphicx}
\begin{table}[!t]
\centering
\caption{Error rate of Booleanised DCR rules w.r.t.\ ground truth rules. Error rate represents how often the label predicted by a Booleanised rule differs from the fuzzy rule generated by our model. The error rate is reported with the mean and standard error of the mean. A full list of logic rules for MNIST is in Appendix~\ref{app:mnist}.}
% \medskip
\label{tab:global-rules}
\resizebox{\columnwidth}{!}{%
\begin{tabular}{lll}
\hline
\multicolumn{1}{l}{\textbf{\textsc{Ground-truth Rule}}} & \multicolumn{1}{l}{\textbf{\textsc{Predicted Rule}}} & \multicolumn{1}{l}{\textbf{\textsc{Error (\%)}}} \\ 
\hline
\multicolumn{3}{c}{\textbf{XOR}} \\
$y_0 \leftarrow \neg c_0 \wedge \neg c_1$ & $y_0 \leftarrow \neg c_0 \wedge \neg c_1$ & $0.00 \pm 0.00$ \\
$y_0 \leftarrow c_0 \wedge c_1$ & $y_0 \leftarrow c_0 \wedge c_1$ & $0.00 \pm 0.00$ \\
$y_1 \leftarrow \neg c_0 \wedge c_1$ & $y_1 \leftarrow \neg c_0 \wedge c_1$ & $0.02 \pm 0.02$ \\
$y_1 \leftarrow c_0 \wedge \neg c_1$ & $y_1 \leftarrow c_0 \wedge \neg c_1$ & $0.01 \pm 0.01$ \\
\multicolumn{3}{c}{\textbf{Trigonometry}} \\
$y_0 \leftarrow \neg c_0 \wedge \neg c_1 \wedge \neg c_2$ & $y_0 \leftarrow \neg c_0 \wedge \neg c_1 \wedge \neg c_2$ & $0.00 \pm 0.00$ \\
$y_1 \leftarrow c_0 \wedge c_1 \wedge c_2$ & $y_1 \leftarrow c_0 \wedge c_1 \wedge c_2$ & $0.00 \pm 0.00$ \\ 
\multicolumn{3}{c}{\textbf{MNIST-Addition}} \\
$y_{18} \leftarrow c'_9 \land  c''_9$ & $y_{18} \leftarrow c'_9 \land  c''_9$ & $0.00 \pm 0.00$ \\
$y_{17} \leftarrow c'_9 \land  c''_8$ & $y_{17} \leftarrow c'_9 \land  c''_8$ & $0.00 \pm 0.00$ \\
$y_{17} \leftarrow c'_8 \land  c''_9$ & $y_{17} \leftarrow c'_8 \land  c''_9$ & $0.00 \pm 0.00$ \\
\hline
\end{tabular}%
}
\end{table}

\paragraph{
DCR rules are stable under small perturbations (Figure~\ref{fig:sensitivity})}
An important characteristic of local explanations is to be stable under small perturbations \citep{yeh2019fidelity}. Indeed, users do not trust explanations if they change significantly on very similar inputs for which the model make the same prediction. This metric, also known as explanation sensitivity, is generally computed as the maximum change in the explanation of a model $\Phi(f)$ on a slightly perturbed input ($x^{\star}$), that is, $|\Phi(f(\mathbf{x}^\star)) - \Phi(f(\mathbf{x}))|,  |\mathbf{x}-\mathbf{x}^\star|_\infty< \epsilon$. We compare the DCR explanations w.r.t. our interpretable baselines as well as w.r.t. LIME~\cite{ribeiro2016should} explaining the output of XGBoost. Since we are using different types of models, we use a normalised version of the sensitivity $ |\Phi(f(\mathbf{x}^\star)) - \Phi(f(\mathbf{x}))| / |\Phi(f(\mathbf{x}))|$. 
% Both the norm and the distance between any two explanations are computed by taking into consideration the feature importance vector that is provided to locally explain the model. 
We compute the distance between two explanations considering the feature importance of the original explanation w.r.t. to the feature importance of the explanation for the perturbed example. For decision tree's rules we consider distance between original path and the path of the perturbed example. 
% For a decision tree's Boolean explanations, we take into consideration the presence and the sign of the Boolean atom associated with each feature. 
As highlighted in Figure \ref{fig:sensitivity}, in all datasets the explanations provided by DCR are very stable, particularly w.r.t. LIME and ReluNet. Notice that the figure does not report the explanation sensitivity of logistic regression and decision tree because it is trivially zero as they learn fixed rules for the entire dataset. The area under the sensitivity curves of all methods together with further details concerning this experiment has been reported in Appendix~\ref{app:sensitivity}.
%In contrast, and as one may expect, the explanation sensitivity of logistic regression and decision trees is zero as these models learn fixed rules for the entire dataset. For this reason, their explanation sensitivities have been only reported in Table~\ref{tab:exp_sensitivity} in Appendix~\ref{app:sensitivity}.  

% % \subsection{Causality}
% \paragraph{DCR supports effective human interventions (Figure~\ref{fig:interventions})}
% As discussed in Section~\ref{sec:DCR}, DCR was designed to enable human concept interventions in its predicted explanations that enable users to improve the model's performance when deployed in a human-in-the-loop setup. We test their receptiveness to expert interventions by intervening on a randomly selected subset of concepts across all concept-annotated datasets as we vary the size of the set of intervened concepts. Our results, shown in Figure~\ref{fig:interventions}, show that DCR is highly receptive to test-time concept interventions, with a high increase in performance as the number of test-time intervened concepts increases. Furthermore, we observe that DCR's performance is competitive against that observed in competing models such as CBMs and CEMs. Although we observe that DCR's performance tends to be below that of CEM, our model is capable of enabling a deeper level of human inspection and interpretability by offering experts the possibility of inspecting a set of rules from which they can infer novel knowledge in the task of interest. Both of these properties, that is DCR's positive reaction to interventions and its ability to allow expert inspection of its learnt rules, enable our method to be a good candidate for practical applications in which an expert needs to certify, evaluate, or interact with the model's explanations for its predictions.

% \begin{figure}[h!]
%     \centering
%     % \includegraphics[width=0.75\columnwidth]{figs/interventions/intervention_results.pdf}
%     \todo{to add}
%     \caption{Task ROC-AUC (y-axis) as a function of the number of concepts intervened on at test-time for the XOR, Trig, Dot, and CelebA datasets. For each method, we show the mean AUC values computer across 3 randomly selected groups of concepts to intervene for 5 models trained with different initialization seeds. Their respective standard errors are therefore shown in the shaded areas. \todo{MATEO to add plot here once all experiments have been completed.}}
%     \label{fig:interventions}
% \end{figure}
% More precise


\begin{figure}[t]
    \centering
%    \includegraphics[trim=0 0 0 0, clip, width=0.9\columnwidth]{figs/counterfactual/results_sensitivity.pdf}
    \caption{Sensitivity of model explanation when changing the radius of the input perturbation. The lower, the better. DCR explanations engender trust as they are stable under small perturbations of the input. The same does not hold generally for LIME explanations of XGBoost or Relu Net decision rules.}
    \label{fig:sensitivity}
\end{figure}

\begin{figure}[t]
    \centering
%    \includegraphics[width=.965\columnwidth]{figs/counterfactual/results_counterfactuals.pdf}
    \caption{Model confidence as a function of the number of perturbed features on counterfactual examples. The lower, the better. Similarly to interpretable methods, DCR prediction confidence quickly drops after inverting the truth degree of a small set of relevant concepts, facilitating the discovery of counterfactual examples. }
    \label{fig:counterfactuals}
\end{figure}

% \begin{table}[t]
%     \caption{AUC of the explanation sensitivity curves when increasing the perturbation radius $\epsilon$. The lower, the better. }
%     \label{tab:exp_sensitivity}
%     \centering
%     \resizebox{\columnwidth}{!}{
%     \begin{tabular}{llllll}
%     \toprule
%     Model &                        XOR &                       Trig &                        Vec &                      Mutag &                     CelebA \\
%     \midrule
%     DT              &\bf0.000{\tiny $\pm 0.000$ } &\bf0.000{\tiny $\pm 0.000$ } &\bf0.000{\tiny $\pm 0.000$ } &\bf0.000{\tiny $\pm 0.000$ } &\bf0.000{\tiny $\pm 0.000$ } \\
%     LR              &\bf0.000{\tiny $\pm 0.000$ } &\bf0.000{\tiny $\pm 0.000$ } &\bf0.000{\tiny $\pm 0.000$ } &\bf0.000{\tiny $\pm 0.000$ } &\bf0.000{\tiny $\pm 0.000$ } \\
%     ReluNet          &   0.939{\tiny $\pm 1.301$ } &   0.110{\tiny $\pm 0.181$ } &   0.148{\tiny $\pm 0.247$ } &   0.995{\tiny $\pm 1.480$ } &   0.106{\tiny $\pm 0.169$ } \\
%     LIME            &   0.984{\tiny $\pm 0.885$ } &   0.013{\tiny $\pm 0.009$ } &   0.592{\tiny $\pm 0.534$ } &   1.900{\tiny $\pm 0.969$ } &       nan{\tiny $\pm nan$ } \\
%     DCR             &\bf0.000{\tiny $\pm 0.000$ } &\bf0.000{\tiny $\pm 0.000$ } &   0.165{\tiny $\pm 0.614$ } &\bf0.000{\tiny $\pm 0.000$ } &   1.292{\tiny $\pm 1.652$ } \\
%     \bottomrule
%     \end{tabular}
%     }
% \end{table}

\paragraph{DCR enables discovering counterfactual examples (Figure~\ref{fig:counterfactuals})}
Besides being stable, DCR rules can be used to find simple counterfactual examples, as introduced in Section~\ref{sec:ruleadv}. In Figure~\ref{fig:counterfactuals} we show a model's confidence in its predictions as we increase the number of concept perturbations. In making perturbations, we sort concepts from the most relevant to the least using DCR rules, as suggested by~\citet{wachter2017counterfactual}. 
Our results show that DCR confidence in its predictions drops quickly when we perturb the most relevant concepts according to a given rule. 
% perturb the input sample according to DCR explanation, 
This enables us to discover counterfactual examples where the concept literals are very similar to the original one rule. 
%As it can be seen  in most cases inverting the truth values of the 2 most relevant concepts is enough to cause a significant drop in DCR's confidence in the predicted class. 
This behaviour is emblematic of interpretable models such as decision trees and logistic regression, for which similar conclusions can be drawn. 
% This property of DCR is emphasized in datasets where concepts truth are crisp and stable. 
We also observe how in \emph{Mutagenicity} DCR confidence is a bit higher than interpretable baselines. We can explain this behavior as for this challenging dataset DCR rules give equal relevance to a larger set of concepts. Still DCR confidence is much lower than a black box such as XGBoost.
%In fact, black boxes such as XGBoost typically generate highly non-linear combinations of concepts. %These non-linearities make the effect of a simple truth degree inversion highly unpredictable as shown in Figure~\ref{tab:counterfactuals}. 
% In contrast, naively inverting the truth degree of some concepts in 
% as in black boxes input perturbations lead to an unpredictable effect on the model's prediction, hindering the search for a counterfactual example. 
Local explainers such as LIME can only partially explain the decision process of black box models such as XGBoost: LIME areas under the model confidence curve are generally higher than the other methods. The actual values for all methods are reported in Table \ref{tab:counterfactuals} in Appendix~\ref{app:counterfactual} together with further details and counterfactual examples.

% \begin{table}
% \caption{AUC of the model confidence against counterfactual explanations when increasing the number of perturbed features. The lower, the better.}
% \resizebox{\columnwidth}{!}{
%     \begin{tabular}{lrrrrr}
%     \toprule
%     Model &       XOR &      Trig &       Vec &     Mutag &    CelebA \\
%     \midrule
%     DT          &\bf0.339{\tiny $\pm 0.468$ } &\bf0.395{\tiny $\pm 0.380$ } &\bf0.443{\tiny $\pm 0.442$ } &\bf0.185{\tiny $\pm 0.311$ } &\bf0.333{\tiny $\pm 0.471$ } \\
%     LR          &   0.992{\tiny $\pm 0.015$ } &   0.451{\tiny $\pm 0.402$ } &   0.530{\tiny $\pm 0.391$ } &   0.347{\tiny $\pm 0.351$ } &\bf0.334{\tiny $\pm 0.471$ } \\
%     ReluNet      &   0.622{\tiny $\pm 0.476$ } &   0.469{\tiny $\pm 0.429$ } &   0.448{\tiny $\pm 0.457$ } &\bf0.279{\tiny $\pm 0.387$ } &   0.338{\tiny $\pm 0.468$ } \\
%     LIME        &   0.674{\tiny $\pm 0.462$ } &   0.424{\tiny $\pm 0.422$ } &   0.450{\tiny $\pm 0.438$ } &   0.249{\tiny $\pm 0.372$ } &   0.667{\tiny $\pm 0.471$ } \\
%     XGBoost     &   0.680{\tiny $\pm 0.460$ } &   0.739{\tiny $\pm 0.431$ } &   0.804{\tiny $\pm 0.426$ } &   0.924{\tiny $\pm 0.226$ } &   1.000{\tiny $\pm 0.000$ } \\
%     DCR         &\bf0.344{\tiny $\pm 0.505$ } &\bf0.255{\tiny $\pm 0.436$ } &\bf0.394{\tiny $\pm 0.489$ } &   0.705{\tiny $\pm 0.467$ } &   0.754{\tiny $\pm 0.447$ } \\
%     \bottomrule
%     \end{tabular}
% }
% \end{table}
% Only decision trees provide similar performances, with the other methods requiring to modify 2–3 features to reach the same confidence level.

% Our model makes local inferences by learning a simple logic rule. This logic rule can be seen as a hypothetical causal graph linking concepts to tasks. We can then play directly with this causal graph to intervene on concepts and/or on rule weights. Each modification of the rule is necessary and sufficient to obtain a specific prediction (as opposed to a black-box predictor). 
% This allows us to generate counterfactuals by design (causality level-3 according to Pearl\todo{Missing citation?}). 
% In Figure \ref{fig:counterfactual_trig}, we test the capability of providing counterfactual explanations. Counter-examples $x^\star$ are crafted following the provided explanation, while remaining as close as possible to the original sample $|x - x^\star|< \epsilon$ as proposed in \citet{wachter2017counterfactual}. We compute the reduction of confidence of the originally predicted class on the counter-example $f_(x^\star)$ as a function of the number of modified features. We compared with the interpretable methods used in Figure \ref{fig:accuracy} with the addition of XGBoost when explained by LIME ('Lime') \citep{ribeiro2016should} and a recently proposed interpretable neural network ('XReluNN') \citep{ciravegna2023logic}. As a baseline, we also compare with XGBoost when random feature permutations are performed on the original example('XGBoost').  

% \begin{figure}
%     \centering
%     \includegraphics[width=1.\columnwidth]{figs/counterfactual/counterfactual_trig.jpg}
%     \caption{Model confidence reduction of the originally predicted class as a function of the number of perturbed features on the Trig dataset. More figures are provided in Appendix \ref{app:counterfactual}.}
%     \label{fig:counterfactual_trig}
% \end{figure}

% More precisely, in Figure \ref{fig:counterfactual_trig} we reported the average confidence reduction on all the perturbed test samples of the Trig dataset and across 5 seed initializations. We can clearly observe how the confidence of the proposed method is drastically reduced after modifying only one feature. Only the Decision Tree provides similar performances, with the other methods requiring to modify 2–3 features to reach the same confidence level. Analogous results occur on the other datasets and are reported in Appendix \ref{app:counterfactual}. 


% \subsection{Reasoning and generalizing out-of-distribution}
% Here we show that we can train the model on a set of concepts and test it OOD by removing or adding new concepts on CelebA.

% \subsection{Distant supervisions}
% Here we show that we can train the model without supervising all concepts.

% \subsection{Explaining global behavior}
% Here we show the global behavior of the model by looking at Boolean global rules and comparing them with ground-truth logic rules.


\section{Key Findings \& Significance}
\label{sec:dcr-disc}
\paragraph{Relations with the state-of-the-art}
Interpretable concept-based models~\cite{koh2020concept} address the lack of human trust in AI systems as they allow their users to understand their decision process~\cite{rudin2019stop}.
% and to improve their performance by controlling and interacting with the learnt concepts~\cite{shen2022trust}. 
These approaches come with several advantages over other explainability methods as they circumvent the brittleness of post-hoc methods~\citep{adebayo2018sanity, kindermans2019reliability} and provide a semantic advantage in settings where input features are naturally hard to reason about (e.g., raw image pixels) by providing explanations in terms of human-interpretable concepts~\cite{ghorbani2019interpretation}.
% ; (iii) Engender human trust by concept interventions as opposed to other concept-based interpretable models such as Self-Explainable Neural Networks~\citep{alvarez2018towards} and Concept Whitening~\cite{chen2020concept}. 
However,~\citet{zarlenga2022concept} and~\citet{mahinpei2021promises} emphasise how state-of-the-art concept-based models
% : (i) require a minimum set of concept labels for training which might be expensive or intractable to obtain in some contexts,  (ii) 
either struggle to efficiently solve real-world tasks using concept truth-values only or they weaken their interpretability using concept embeddings to increase their learning capacity. This is true even when concept-based models use a simple logistic regression or decision tree to map concepts embeddings to tasks because concept embedding dimensions do not have a clear semantic meaning, and models composing such dimensions generate prediction rules that are not human interpretable. Our work solves this issue by introducing the first interpretable concept-based model that learns logic rules from concept embeddings. Our approach draws from t-norm fuzzy logic learning paradigms~\cite{diligenti2021constraint,badreddine2022logic,van2022analyzing} to obtain high generalisation across tasks and provide meaningful logic rules even in the absence of concept labels during training.
% \todo{ADD: Several neural-symbolic approaches attempts to combine logic knowledge with deep neural architectures \cite{garcez2022neural,hitzler2022neuro} to get advantages from both of the representations. One of the main paradigm for this integration consists in relaxing the logic rules with a t-norm fuzzy logic \cite{diligenti2021constraint,badreddine2022logic,van2022analyzing} to set the task in a purely differentiable learning setting.}


% \paragraph{Limitations}
% \todo{for authors and reviewers of this paper: please populate this section with ideas and comments
% \begin{itemize}
%     \item 
% \end{itemize}
% }


% \paragraph{Broader Impact}

\paragraph{Conclusion}
This work presents the \textit{Deep Concept Reasoner} (DCR), the new state-of-the-art of interpretable concept-based models. To achieve this, DCR builds for each sample a weighted logic rule combining neural and symbolic algorithms on concept embeddings in a unified end-to-end differentiable system. In our experiments, we compare DCR with state-of-the-art interpretable concept-based models and black-box models using datasets spanning three of the most common data types used in deep learning: tabular, image, and graph data. 
Our experiments show that Deep Concept Reasoners: (i) attain better task accuracy w.r.t.\ state-of-the-art interpretable concept-based models, (ii) discover meaningful logic rules, and (iii) facilitate the generation of counterfactual examples. 
% attain better task accuracy w.r.t.\ state-of-the-art black-box models in out-of-distribution settings where some of the training concepts are not available at test time, (iii) make interpretable concept-based predictions for the task labels which do not need to be explained in a post-hoc manner, 
% (iv) support effective 
% human interventions and
% counterfactual explanations, and \mbox{(v) provide} meaningful logic rules even in the absence of concept labels at training time. 
While the global behaviour of the model is still not directly interpretable, our results show how aggregating Boolean DCR rules provides an approximation for the global behaviour of the model which matches known ground truth relationships. As a result, our experiments indicate that DCR represents a significant advance over the current state-of-the-art of interpretable concept-based models, and thus makes progress on a key research topic within the field of explainability.


%%%%%%%%%%%%%%%%%%%%%%%%%%%%%%%%%%%%%%%%%%%%%%%%%%%%%%%%%%%%%%%%%%%%%%%%%%%%%%%%%
%%% Applications:
%%%
%\chapter{Applications}

\section{Graph domain}

\section{Vision}

\section{Medicine and tabular data}


%
%%%%%%%%%%%%%%%%%%%%%%%%%%%%%%%%%%%%%%%%%%%%%%%%%%%%%%%%%%%%%%%%%%%%%%%%%%%%%%%%
%% Conclusion:
%%
\chapter{Conclusion} \label{chapter:conclusion}
\textbf{Research: completed. Status: drafted. Difficulty: low. Priority: low.}

\textit{I will conclude my thesis with a summary of my inventions. I will then discuss the impact of my contributions to the deep learning field as well as to broader research communities.}

\section{Summary of the Contributions}

\section{Potential Impact on Research and Society}

\section{The Next Decades in AI}



\chapter*{Other papers}
\nobibliography*

\bibentry{barbiero2020modeling}

\bibentry{barbiero2020computational}

\bibentry{georgiev2021algorithmic}

\bibentry{barbiero2021predictable}


%%%%%%%%%%%%%%%%%%%%%%%%%%%%%%%%%%%%%%%%%%%%%%%%%%%%%%%%%%%%%%%%%%%%%%%%%%%%%%%%
%% References:
%%
% If you include some work not referenced in the main text (e.g. using \nocite{}), consider changing "References" to "Bibliography".
%

% \renewcommand to change default "Bibliography" to "References"
\renewcommand{\bibname}{References}
\cleardoublepage
\phantomsection
\addcontentsline{toc}{chapter}{References}
%\bibliographystyle{plainnat}
\bibliography{thesis.bib}



%%%%%%%%%%%%%%%%%%%%%%%%%%%%%%%%%%%%%%%%%%%%%%%%%%%%%%%%%%%%%%%%%%%%%%%%%%%%%%%%
%% Appendix:
%%
\appendix


%%%%%%%%%%%%%%%%%%%%%%%%%%%%%%%%%%%%%%%%%%%%%%%%%%%%%%%%%%%%%%%%%%%%%%%%%%%%%%%%
%% Concept Quality:
%%
\chapter{Concept Quality} \label{chapter:metrics}
% \textbf{Research: completed. Status: drafted. Difficulty: low. Priority: low.}

% \textit{In this chapter I will discuss how to measure the quality of concept representations. In particular I will focus on my contribution in inventing the niche impurity score~\citep{zarlenga2021quality} which generalizes concept completeness~\citep{yeh2020completeness} to concept subsets. I will demonstrate how this metric is computationally efficient and does not require concept labels thus making it applicable in real-world supervised and unsupervised scenarios. I will conclude the chapter with experiments showing how the niche impurity score can be used in practice to evaluate the robustness of concept representations generated by state-of-the-art supervised and unsupervised concept learning methods.}

\textbf{Motivation---} Very few metrics available to assess concept quality. Hard to understand whether to trust concept-based models explanations based on learnt concepts.

\textbf{Solution---} Two new metrics to assess concept quality and robustness.

The \textbf{key innovation} consists in generalizing existing metrics to subset of concepts (niching) and to concept embeddings (alignment).

\section{Concept completeness}


\section{Concept niches and concept impurity}
\begin{definition}[Concept nicher] \label{def:nicher}
Given a set of concept representations $\hat{C} \subseteq \mathbb{R}^{d \times k}$, we define a concept nicher as a function $\nu: \{1, \cdots k\} \times \{1, \cdots k\} \mapsto [0, 1]$ that returns $\nu(i, j) \approx 1$ if the $i$-th concept $\mathbf{\hat{c}}_{(:, i)}$ is entangled with the $j$-th ground truth concept $c_j$, and $\nu(i, j) \approx 0$ otherwise.
\end{definition}

Our definition above can be instantiated in various ways, depending on how entanglement is measured. In favour of efficiency, we measure entanglement using absolute Pearson correlation $\rho$, as this measure can efficiently discover (a linear form of) association between variables~\cite{altman2015points}. We call this instantiation  \emph{concept-correlation nicher} (CCorrN) and define it as
$\text{CCorrN}(i, j) := \big| \rho\big(\{\mathbf{\hat{c}}^{(l)}_{(:, i)}\}_{l=1}^N, \{c^{(l)}_j\}_{l=1}^N\big) \big|$.

% The above definition is affected by how entanglement is defined. One efficient way of measuring the entanglement is to use the absolute Pearson correlation, denoted as $\rho$. We call such an instantiation a \emph{concept-correlation nicher} (CCorrN) and define it as:
% \[
%     \text{CCorrN}(i, j) := \big| \rho\big(\{\mathbf{\hat{c}}^{(l)}_{(:, i)}\}_{l=1}^N, \{\mathbf{\hat{c}}^{(l)}_j\}_{l=1}^N\big) \big|
% \]
If $\mathbf{\hat{c}}_{(:, i)}$ is not a scalar representation (i.e., $d > 1$), then for simplicity we use the maximum absolute correlation coefficient between all entries in $\mathbf{\hat{c}}_{(:, i)}$, and the target concept label $c_j$ as a representative correlation coefficient for the entire representation $\mathbf{\hat{c}}_{(:, i)}$. We then define a concept niche as: 
\begin{definition}[Concept niche]
The concept niche $N_j(\nu, \beta)$ for target concept $j$, determined by concept nicher $\nu(\cdot, \cdot)$ and threshold $\beta \in [0,1]$, is defined as $N_j(\nu, \beta) := \big\{i \ \ | \ \ i \in \{1, \cdots, k\} \text{ and } \nu(i, j) > \beta \big\}$.
\end{definition}

From this, the Niche Impurity (NI) measures the predictive capacity of the complement of concept niche $N_i(\nu, \beta)$, referred to as $\neg N_i(\nu, \beta) := \{1, \cdots, k\} \; \backslash \; N_i(\nu, \beta)$, for the $i$-th ground truth concept:
%Given the complement of concept niche $N_i(\nu, \beta)$, which we refer to as $\neg N_i(\nu, \beta) := \{1, \cdots, k\} \; \backslash \; N_j(\nu, \beta)$, the Niche Impurity (NI) measures its predictive capacity for the $i$-th ground truth concept.

\begin{definition}[Niche Impurity (NI)] \label{def:niche_impurity}
Given a classifier $f: \hat{C} \mapsto C$, concept nicher $\nu$, threshold $\beta \in [0, 1]$, and labeled concept representations $\{(\mathbf{\hat{c}}^{(l)}, \mathbf{c}^{(l)})\}_{l = 1}^n$, the Niche Impurity of the $i$-th output of $f(\cdot)$ is defined as $\text{NI}_i(f, \nu, \beta) := \text{AUC} \big( \{( f|_{\neg N_i(\nu, \beta)} \big( \mathbf{\hat{c}}^{(l)}_{(:, \neg N_i(\nu, \beta))} \big), c^{(l)}_i) \}_{l=1}^n \big)$, where $f|_{\neg N_j(\nu, \beta)}$
% : \hat{C} \mapsto C$
is the classifier resulting from masking all entries in $\neg N_j(\nu, \beta)$ when feeding $f$ with concept representations. 
\end{definition}

Although $f$ can be any classifier, in our experiments we use a ReLU MLP with hidden layer sizes $\{ 20, 20 \}$.
Intuitively, a NI of $1/2$ (random AUC of niche complement) indicates that the concepts inside the niche $N_i(\nu)$ are the only concepts predictive of the $i$-th concept, that is, concepts outside the niche do not hold any predictive information of the $i$-th concept.
% In contrast, a NI of $1$ suggests that concepts outside the nice $N_i(\nu)$ are still fully predictive of concept $i$.
Finally, the \textit{Niche Impurity Score} metric measures how much information apparently disentangled concepts
% (target concepts and their niche complements)
are actually sharing:

\begin{definition}[Niche Impurity Score (NIS)] \label{def:niche_impurity_score}
Given a classifier $f: \hat{C} \mapsto C$ and concept nicher $\nu$, the niche impurity score $\text{NIS}(f,\nu) \in [0,1]$ is defined as the summation of niche impurities across all concepts for different values of $\beta$: $\text{NIS}(f,\nu) := \int_{0}^{1} (\sum_{i=1}^{k} \text{NI}_i(f, \nu, \beta)/k) d\beta$.
\end{definition}

In practice, we estimate this integral using the trapezoid method with values in $\beta \in \{ 0.0, 0.05, \cdots, 1\}$. For efficiency, we parameterise $f$ as an MLP,
% one can very efficiently compute the NI for different concepts and values of $\beta$,
leading to a tractable impurity metric which easily scales as the number of concepts $k$ increases. Intuitively, a NIS of $1$ means that all the information to perfectly predict each ground truth concept is spread on many different and disentangled concept representations. In contrast, a NIS around $1/2$ (random AUC) indicates that no concept can be predicted by any concept representation subset.
% Intuitively, a NIS score of $1$ conveys perfect purity and means that the set of learnt concepts can be divided into disentangled sets, each of which is related to predicting a single ground truth concept, while the NIS score around $1/2$ conveys maximum impurity and indicates that information related to each ground truth concept is scattered across all assumed disentangled sets of learnt concepts.

\section{Concept alignment}
The Concept Alignment Score (CAS) aims to measure how much learnt concept representations can be trusted as faithful representations of their ground truth concept labels. Intuitively, CAS generalises concept accuracy by considering the predictions' homogeneity within groups of similar samples. More specifically, CAS applies a clustering algorithm $\kappa$ to find $\rho > 2$ clusters, assigning to each sample $\mathbf{x}^{(j)}$, for each concept $c_i$, a cluster label $\pi_i^{(j)} \in \{1, \cdots, \rho\}$, using the concept representation $\hat{\textbf{c}}_i$. Given $N$ test samples, the homogeneity score $h(\cdot)$~\citep{rosenberg2007v} then computes the conditional entropy $H$ of ground truth labels $C_i = \{c_i^{(j)}\}_{j=1}^{N}$ w.r.t. cluster labels $\Pi_i = \{\pi_i^{(j)}\}_{j=1}^{N}$, i.e., $h = 1$ when $H(C_i,\Pi_i)=0$ and $h = 1 - H(C_i, \Pi_i)/H(C_i)$ otherwise. The higher the homogeneity, the more a learnt concept representation is ``aligned'' with its labels, and can thus be trusted as a faithful representation. CAS averages homogeneity scores over all concepts, providing a normalised score $\text{CAS} \in [0,1]$:
\begin{equation}
    \text{CAS}(\mathbf{\hat{c}}_1, \cdots, \mathbf{\hat{c}}_k) \triangleq \frac{1}{N - 2}\sum_{p=2}^N \Bigg(\frac{1}{k} \sum_{i=1}^k h(c_i, \kappa_p(\hat{\textbf{c}}_i)) \Bigg)
    % \text{CAS}(\mathbf{\hat{c}}_1, \cdots, \mathbf{\hat{c}}_k) := \frac{1}{k(N-2)} \sum_{p=2}^N \sum_{i=1}^k h(c_i, \kappa(\ha  t{\textbf{c}}_i, p)))
\end{equation}
% Notice how when the number of clusters $p$ equals the number of samples, CAS and concept accuracy are identical.
% The concept alignment score is therefore maximal (i.e., $\text{CAS} = 1$) when all clusters contain only data points which are members of a single concept class (i.e., for all samples within a cluster, the label of any concept $i$ is either always $c_i = True$ or always $c_i = False$). 
To tractably compute CAS in practice, we sum homogeneity scores by varying $p$ across $p \in \{2, 2 + \delta, 2 + 2 \delta, \cdots, N\}$ for some $\delta > 1$ (details in Appendix). Furthermore, we use k-Medoids~\citep{kaufman1990partitioning} for cluster discovery, similarly to~\citet{ghorbani2019interpretation} and~\citet{magister2021gcexplainer}, and use concept logits when computing the CAS for Boolean and Fuzzy CBMs. For Hybrid CBMs, we use $\hat{\mathbf{c}}_i \triangleq [\hat{\mathbf{c}}_{[k:k + \gamma]}, \hat{\mathbf{c}}_{[i:(i + 1)]}]^T$ as the concept representation for $c_i$ (i.e., the extra capacity is a shared embedding across all concepts).
% Cluster and concept labels are matched by homogeneity score.
% and use a simple majority class count for labeling a cluster.

\section{Trade-offs in concept learning}
\begin{itemize}
    \item Accuracy vs. scalability/noise
    \item Accuracy vs. explainability
    \item Accuracy vs. interpretability
\end{itemize}


\section*{Papers}
\nobibliography*
\begin{itemize}
    \item \bibentry{zarlenga2021quality}
\end{itemize}



%%%%%%%%%%%%%%%%%%%%%%%%%%%%%%%%%%%%%%%%%%%%%%%%%%%%%%%%%%%%%%%%%%%%%%%%%%%%%%%%
%% Robust Concept Discovery:
%%
\chapter{Concept Self-Awareness} \label{chapter:unsupervised}
\textbf{Research: in progress. Status: drafted. Difficulty: medium. Priority: high.}

\textit{In this chapter I will illustrate how to make models ``aware'' of concepts they discover during training eliminating the need for expensive (and manual) concept annotations. I will first describe how to make networks learn concept encodings without using any concept label during training. Next I will show how to avoid \textit{shortcut learning}~\citep{geirhos2020shortcut} which can prevent the network to learn a ``complete'' and robust concept representation. I have already obtained preliminary results on graph neural networks~\citep{magister2022encoding} which I submitted to the Neural Information and Processing Systems conference. In the next few months I will focus on extending the approach to other common architectures such as convolutional networks. I will conclude the chapter showing the results on real-world settings comparing supervised and unsupervised concept learning methods.}


Summary to this point: up to here we demonstrated how to design self-explainable models which are as accurate as black boxes (or more) without sacrificing interpretability and the effectiveness of causual human interactions. In this chapter I will show a trick to train concept bottleneck modes for GNNs without expensive and sometimes unknown concept annotations.

\section{Motivation}
Knowledge gap/motivation: CEMs and LENs are robust self-explaining models going beyond the current accuracy-vs-explainability trade-off. However, these models require concept supervisions which might be expensive to generate to train the model, but in some cases might not even be known a priori, which makes impossible to train a concept bottleneck in a supervised way. In this setting there are papers (ACE, GCExplainer) showing how latent concepts can be extracted from trained architectures post-hoc with the assumption that: 1 concept == 1 cluster. However, the existing concept-based unsupervised approaches are post-hoc, while we argue against this family of approaches because they do not increase human trust in the model itself allowing for interventions at test time, nor they try to make the model itself more explainable. So, here we try to find a way to exploit the natural clustering performed in hidden layers of a neural network to make the NN aware of concepts and use them to predict the classification targets, making the architecture self-explainable.

Contribution: a self-explainable GNN model distilling unsupervised concepts at train time and using these concepts to solve the task.

Key innovation: a concept distillation layer based on hard cluster encodings.

Expected outcome: the concept distillation layer generates hard cluster encodings (labels) in the hidden layers of a GNN. Each cluster can be thought as a concept representing a spacific motif/subgraph. A LENs model is trained on top of cluster encodings to generate logic explanations for the predictions of the GNN based on the concepts learnt without concept annotations. The model is evaluated in terms of:

- performance: model accuracy

- interpretability: concept purity (edit distance of motifs in each cluster), concept completeness, logic explanations' accuracy and complexity

Research questions: how would this approach compare with existing approaches (post-hoc such as black-box GNN + GCExplainer)? is it interpretable as white boxes in terms of rule complexity? is it accurate as black boxes? what's the accuracy-vs-explainability trade-off of the approach? is there any advantage of using this approach over existing methods?


\section{The Costs of Supervised (Concept) Learning}

\section{Fantastic Concepts and Where to Find Them}

\section{Encoding Concepts in Neural Networks}
\paragraph{Concept distillation}
The first CDM step consists in extracting node-level clusters corresponding to concepts from the GNN's latent space. This is based on the observation that the arrangement of the activation space shows similarities to human perceptual judgement~\citep{zhang2018unreasonable}, as shown by GCExplainer~\citep{magister2021gcexplainer} for GNNs. However, in contrast to GCExplainer, in CDM this step is differentiable and integrated in the network architecture, allowing gradients to optimize clusters in GNN embeddings. Specifically, we implement this differentiable clustering using a normalized softmax activation on the node-level embeddings $\mathbf{h}_i$, associating each node with one cluster/concept. This operation returns for each node a fuzzy encoding $\mathbf{q}_i \in [0,1]^s$:
\begin{equation} \label{eq:diffGCExp}
    \tilde{\mathbf{q}}_i = \frac{\exp({\mathbf{h}_i})}{\sum_{u=1}^s \exp(\mathbf{h}_{iu})}, \qquad \mathbf{q}_i = \frac{\tilde{\mathbf{q}}_i}{\max_i \tilde{\mathbf{q}}_i + \epsilon}
\end{equation}
where $s$ is the size of the encoding vector. CDM then clusters nodes considering the similarity of their fuzzy encodings $\mathbf{q}_i$. Specifically, CDM groups the samples together depending on their Booleanized encoding $\mathbf{r}_i \in \{0,1\}^s$:
\begin{equation}
    \mathbf{r}_{iu} = 
    \begin{cases}
    1 \quad \text{ if } \mathbf{q}_{iu} \geq \tau\\
    0 \quad \text{ otherwise }
    \end{cases}
    % \mathbb{I}_{\mathbf{q}_i \geq \epsilon}
\end{equation}
where $\tau \in [0,1]$ is conventionally set to $0.5$.
In particular, two samples $a$ and $b$ belong to the same cluster if and only if their encodings $\mathbf{r}_a$ and $\mathbf{r}_b$ match.  For example, consider the two node embeddings $\mathbf{h}_a = [-1.2, 2.3]$ and $\mathbf{h}_b = [2.2, 1.8]$. For these inputs, the normalized softmax would return the fuzzy encodings $\mathbf{q}_a = [0.0293, 0.9707]$ and $\mathbf{q}_b = [0.5987, 0.4013]$, respectively. As their Booleanizations $\mathbf{r}_a = [0, 1]$ and $\mathbf{r}_b = [1, 0]$ do not match, we can then conclude that the two nodes belong to different clusters. Notice how our concept encoding is highly efficient, as it allows to learn up to $2^s$ different concepts on GNN embeddings $\mathbf{h}_i$ of size $s$. This way the GNN can dynamically find the optimal number of concepts/clusters, thus relieving users from this burden. In fact, users just need to choose an upper bound to the number of concepts $s$ rather than an exact value, as when using k-Means like in GCExplainer. In order to account for graph classification, the concept encodings for a graph are pooled before being passed to the interpretable model predicting the task, as explained in the next paragraph.
% On the other hand, the softmax function puts $\mathbf{q}_i$ elements in competition, thus encouraging the model to identify only a few concepts.

\paragraph{Interpretable predictions}
The second CDM step consists of using the distilled concepts to make interpretable predictions for downstream tasks. In particular, the presence of concepts enables pairing GNNs with existing concept-based methods which are explainable by design, such as Logic Explained Networks (LENs,~\citep{ciravegna2021logic}). LENs are neural models providing simple concept-based logic explanations for their predictions. Specifically, LENs can provide class-level explanations which makes our approach the first at providing unique global explanations for GNNs. CDM uses a LEN as the readout function $f$ for the classification, applying it on top of concept representations $\mathbf{q}_i$. For graph classification tasks, the input data is composed of a set of $t$ graphs $G^j \in \{(V^j, E^j)\}_{j=1}^t$, where each graph is associated with a task label $y^j \in Y$. In this setting, GNN-based models predict a single label for each graph $G^j$ by pooling its node-level encodings $\mathbf{q}_i^j$ to aggregate over multiple concepts:
\begin{equation} \label{eq:lens}
    \hat{y}_i = \text{LEN}_{\text{node}} ( \mathbf{q}_i), \qquad
    \hat{y}^j = \text{LEN}_{\text{graph}} \Bigg(\frac{1}{n_j} \sum_{i=1}^{n_j} \mathbf{q}_i^j \Bigg)
\end{equation}
where $n_j$ is the number of nodes associated with graph $j$. In our implementation, we use the entropy-based layer to implement LENs~\citep{barbiero2021entropy}) as it can provide high classification accuracy with high-quality logic explanations. This entropy-based layer implements a sparse attention layer designed to work on top of concept activations.
% \begin{equation} \label{eq:alpha}
%     \alpha^i = \frac{e^{\gamma^i/\tau}}{\sum_{l=1}^k e^{\gamma^i_l/\tau}}
% \end{equation}
The attention mechanism allows the model to focus on a small subset of concepts to solve each task. It also introduces a parsimony principle in the architecture corresponding to an intuitive human cognitive bias~\citep{miller1956magical}. This parsimony principle allows the extraction of simple logic explanations from the network, thus making these models explainable by design. 

% \paragraph{Interactive Concept-based Graph Layer}
% The Interactive Concept-based Graph Layer combines DGCExplainer and E-LENs. Conceptually, the layer first performs concept discovery using the normalised softmax activation function to associate each node embedding with a concept encoding. In order to account for graph classification, the concept encodings for a graph are pooled using global mean pooling before being passed to E-LENs. We propose global mean pooling over minimum or maximum pooling of concept representations, as it will give an average idea of the activation of a feature. We implement the Interactive Concept-based Graph Layer in the following way:
% \begin{equation} \label{eq:lens}
%     (y, \phi) = LEN(\frac{1}{n} \sum_{i=1}^{n}q_i)
% \end{equation}
% where $n$ is the number of nodes associated with graph $j$. 
% The components of the layer are fully differentiable, allowing to train the model using classical methods, and making it explainable-by-design. Furthermore, the concept encoding component of the layer allows for human intervention, as the concept encoding may be corrected after the training of the network to improve model performance.

\paragraph{Concept-based and logic-based explanations}
The proposed method provides two types of explanations: concept-based and logic-based explanations. Global concept-based explanations can be extracted in a similar manner as in GCExplainer: a concept for a node or graph is extracted by finding the cluster with which a node's embedding is associated, and visualising the samples closest to the cluster centroid. The logic-based formula provided per class broadens the explanation scope, as it indicates which neurons of the concept encoding $\mathbf{q}_i$ are activated and representative of a class. This provides a more comprehensive explanation since a class can be associated with multiple concepts. 

\paragraph{Concept interventions}
As in Concept Bottleneck Models~\citep{koh2020concept}, our approach supports human interaction at concept level. In fact, in contrast to existing post-hoc methods, an explainable-by-design approach creates an explicit concept layer which can positively react to test-time human interventions. For instance, consider a misclassified node with concept encoding $\mathbf{q}_a = [0.21, 0.93]$. Assume that the vast majority of nodes with the binary encoding $\mathbf{r}_{\text{grid\_node}} = [0, 1]$ are nodes of a grid-like structure, which allows a human to label this cluster as ``grid nodes''. Now, a human expert can inspect the neighborhood of the misclassified node and realize that this node belongs to a circle-like structure and not to a grid structure. As the binary encoding for the concept ``circle nodes'' is $\mathbf{r}_{\text{circle\_node}} = [1, 1]$, the user can easily apply an intervention to correct the misclassified concept by changing its encoding to $\mathbf{q}_a:=[1, 1]$. Such an update allows the interpretable readout function to act on information related to the corrected concept, thus improving the original model prediction.

\section{Experiments and results}

\begin{table*}[!t]
\centering
\resizebox{\textwidth}{!}{
\begin{tabular}{lllllll}
\toprule     & \multicolumn{2}{c}{\textbf{\begin{tabular}[c]{@{}c@{}}Model Accuracy (\%)\end{tabular}}} & \multicolumn{2}{c}{\textbf{\begin{tabular}[c]{@{}c@{}}Concept  Completeness (\%)\end{tabular}}} & \multicolumn{2}{c}{\textbf{\begin{tabular}[c]{@{}c@{}}Concept  Purity\end{tabular}}} \\
                       & \multicolumn{1}{c}{\textbf{CGN}}      & \multicolumn{1}{c}{\textbf{Vanilla GNN}}      & \multicolumn{1}{c}{\textbf{CGN}}          & \multicolumn{1}{c}{\textbf{Vanilla GNN}}         & \multicolumn{1}{c}{\textbf{CGN}}      & \multicolumn{1}{c}{\textbf{Vanilla GNN}}      \\ \midrule
\textbf{BA-Shapes}     & \textbf{98.11 (97.04, 99.18)}         & 98.02 (96.40, 99.65)                          & \textbf{98.11 (96.85, 99.36)}             & 93.69 (86.21, 100.00)                            & \textbf{0.00 (0.00, 0.00)}            & 0.00 (0.00, 0.00)                             \\
\textbf{BA-Community}  & 85.67 (81.38, 89.95)        & \textbf{87.50 (85.56, 89.45)}                 & \textbf{83.10 (78.90, 87.29)}             & 75.74 (72.85, 78.64)                             & 1.70 (0.43, 3.83)            & \textbf{1.60 (0.49, 2.71)}                    \\
\textbf{BA-Grid}       & 99.51 (98.75, 100.00)        & \textbf{99.71 (99.38, 100.00)}                & 99.61 (98.80, 100.00)                     & \textbf{99.71 (99.38, 100.00)}                   & \textbf{0.20 (0.00, 0.76)}            & 2.40 (0.00, 6.48)                             \\
\textbf{Tree-Cycle}    & \textbf{94.97 (92.50, 97.44)}         & 86.26 (58.58, 100.00)                         & \textbf{91.98 (83.71, 100.00)}            & 91.16 (84.47, 97.86)                             & \textbf{0.00 (0.00, 0.00)}            & 0.60 (0.00, 2.27)                              \\
\textbf{Tree-Grid}     & \textbf{95.17 (93.59, 96.75)}         & 94.54 (93.61, 95.46)                          & \textbf{91.37 (84.58, 98.16)}             & 78.48 (76.17, 80.79)                             & \textbf{0.00 (0.00, 0.00)}            & 0.00 (0.00, 0.00)                              \\
\textbf{Mutagenicity}  & \textbf{82.40 (81.31, 83.48)}         & 82.35 (81.64, 83.06)                          & 63.40 (58.84, 67.96)            & \textbf{63.95 (60.14, 67.77)}                    & 1.00 (0.00, 3.78)                     & \textbf{0.60 (0.00, 2.27)}                    \\
\textbf{Reddit-Binary} & 90.55 (87.95, 93.15)                  & \textbf{91.20 (88.82, 93.58)}                 & \textbf{75.91 (61.16, 90.66)}             & 73.10 (58.44, 87.75)                  & 0.40 (0.00, 1.51)                     & \textbf{0.00 (0.00, 0.00)}                   \\ \bottomrule
\end{tabular}%
}
\caption{Model accuracy and concept completeness for the Concept-based Graph Network (CGN) and an equivalent vanilla GNN. For these results, and those that follow, we compute all metrics on test sets across five seeds and report their mean and $95\%$ confidence intervals.}
    \label{fig:accuracy}
\end{table*}


% \newcommand\conceptsizef{30}
% \newcommand\vfigsf{-.5\height}

% % \newcommand\conceptsizef{60}
% % \newcommand\vfigsf{-.3\height}
% % Please add the following required packages to your document preamble:
% % \usepackage{graphicx}
% \begin{table*}[!t]
% \centering
% \renewcommand{\arraystretch}{1}
% \resizebox{\textwidth}{!}{%
% \begin{tabular}{cccccccc}
% \toprule
%  & \textbf{BA-Shapes} & \textbf{BA-Grid} & \textbf{Tree-Grid} & \textbf{Tree-Cycle} & \textbf{BA-Community} & \textbf{Mutagenicity} & \textbf{Reddit-Binary} \\
% \midrule
% \textbf{Ground Truth} & \raisebox{\vfigsf}{\includegraphics[height=\conceptsizef pt]{fig/logic_expl_graphs/house.pdf}} & \raisebox{\vfigsf}{\includegraphics[height=\conceptsizef pt]{fig/logic_expl_graphs/grid.pdf}} & \raisebox{\vfigsf}{\includegraphics[height=\conceptsizef pt]{fig/logic_expl_graphs/grid.pdf}} & \raisebox{\vfigsf}{\includegraphics[height=\conceptsizef pt]{fig/logic_expl_graphs/ring.pdf}} & \raisebox{\vfigsf}{\includegraphics[height=\conceptsizef pt]{fig/logic_expl_graphs/house.pdf}} & \raisebox{\vfigsf}{\includegraphics[height=\conceptsizef pt]{fig/logic_expl_graphs/ring.pdf}}  \raisebox{\vfigsf}{\includegraphics[height=\conceptsizef pt]{fig/logic_expl_graphs/no2.pdf}} & \raisebox{\vfigsf}{\includegraphics[height=\conceptsizef pt]{fig/logic_expl_graphs/star.pdf}}  \\
% \textbf{Extracted Concept} & \raisebox{\vfigsf}{\includegraphics[height=\conceptsizef pt]{fig/BA_Shapes_concept_5.pdf}} & \raisebox{\vfigsf}{\includegraphics[height=\conceptsizef pt]{fig/logic_expl_graphs/BA_Grid_concept_2.pdf}} & \raisebox{\vfigsf}{\includegraphics[height=\conceptsizef pt]{fig/logic_expl_graphs/Tree_Grid_concept_21.pdf}} & \raisebox{\vfigsf}{\includegraphics[height=\conceptsizef pt]{fig/logic_expl_graphs/Tree_Cycle_concept_8.pdf}} & \raisebox{\vfigsf}{\includegraphics[height=\conceptsizef pt]{fig/logic_expl_graphs/BA_Community_concept_30.pdf}} & \raisebox{\vfigsf}{\includegraphics[height=\conceptsizef pt]{fig/logic_expl_graphs/Mutagenicity1.pdf}} & \raisebox{\vfigsf}{\includegraphics[height=\conceptsizef pt]{fig/logic_expl_graphs/Reddit_Binary_concept_24.pdf}}  \\
% \bottomrule\\
% \end{tabular}%
% }
% \caption{The Concept Distillation Module detects meaningful concepts matching the expected ground truth. Blue nodes are the instances being explained, while orange nodes represent their $p$-hop neighbors. Similar motifs are identified by GCExplainer.}
% \label{tab:concept_visuals}
% \end{table*}


%\subsection{Task Accuracy and Completeness}
\paragraph{Concept Graph Networks are as accurate as vanilla GNNs (Table \ref{fig:accuracy})}
Our results show that CDM allows GNNs to achieve better or comparable task accuracy w.r.t. equivalent GNN architectures. Specifically, our approach outperforms vanilla GNNs on the Tree-Cycle dataset, having a higher test accuracy (plus $\sim 8\%$ on average) and less variance across different parameter initializations. We hypothesize that this effect is due to more stable and pure concepts being learnt thanks to CDM, as we will see later when discussing the concept purity scores. We do not observe any significant negative effect of using CDM on the generalization error of GNNs.


\paragraph{The Concept Distillation Module discovers complete concepts (Table \ref{fig:accuracy})}
Our experiments show that overall CDM discovers a more complete set of concepts w.r.t. the concept set extracted by GCExplainer on equivalent GNN architectures. This is particularly emphasized in the Tree-Grid, BA-Shapes and BA-Community datasets, where CDM significantly outperforms GCExplainer by up to $\sim 13\%$. For the other datasets, the proposed approach matches the concept completeness scores of GCExplainer. The completeness scores on the BA-Grid and Mutagenicity datasets are only slightly lower, however, within the margins of the confidence interval. In absolute terms, CDM discovers highly complete sets of concepts with completeness scores close to the model accuracy for the synthetic datasets. 
% As there is a 1-to-1 mapping between clusters and concepts, concept completeness can be visualized as a clear separation in the node embeddings, as exemplified in \ref{fig:ba_shapes_clustering}.


%\subsection{Concept Interpretability}
\paragraph{The Concept Distillation Module identifies meaningful concepts (Table~\ref{tab:concept_visuals})}
CDM discovers high-quality concepts, which are meaningful to humans. Similarly to GCExplainer, our results demonstrate that CDM can discover concepts corresponding to the ground truth motifs embedded in the toy datasets. For example, our approach recovers the ``house motif'' in BA-Shapes. Moreover, CDM proposes plausible concepts for the real-world datasets where ground truth motifs are lacking. In this case, the extracted concepts match the desirable motifs suggested by~\citet{ying2019gnnexplainer}, corresponding to ring structures and the nitrogen dioxide compound in Mutagenicity, and a star-like structure in Reddit-Binary. 
% Figure~\ref{fig:concept_visuals} also reports for each discovered concept the corresponding binary encoding representing the concept signature learnt by the Concept Distillation Module. 
As we use the same visualization technique as GCExplainer the merit of our contribution lies in the discovery of a more descriptive set of concepts, which includes rare and fine-grained concepts. As a thorough qualitative comparison with GCExplainer requires exhaustive visualization, we refer the reader to the Appendix for a complete set of results.

% \begin{figure}[!ht]
%     \centering
%     \includegraphics[width=\textwidth]{fig/fine_concept_visualization.pdf}
%     \caption{The Concept Graph Module detects concepts more fine-grained than the simple ground truth motif encoded.}
%     \label{fig:outliers}
% \end{figure}


% \raisebox{\vfigsf}{\includegraphics[height=\conceptsizef pt]{fig/house_concepts.pdf}}

% \renewcommand\conceptsizef{50}

% \begin{table}[!t]
% \resizebox{\columnwidth}{!}{%
% \begin{tabular}{ccccc}
% \toprule
% \textbf{Ground Truth} & \multicolumn{2}{c}{\textbf{Fine-Grained Concepts}} & \multicolumn{2}{c}{\textbf{Rare Concepts}} \\
% \midrule
% \raisebox{\vfigsf}{\includegraphics[height=\conceptsizef pt]{fig/house_concepts.pdf}} & \raisebox{\vfigsf}{\includegraphics[height=\conceptsizef pt]{fig/BA_Shapes_finegrained2.pdf}} & \raisebox{\vfigsf}{\includegraphics[height=\conceptsizef pt]{fig/BA_Shapes_finegrained5.pdf}} & \raisebox{\vfigsf}{\includegraphics[height=\conceptsizef pt]{fig/BA_Shapes_rare_motif17.pdf}} & \raisebox{\vfigsf}{\includegraphics[height=\conceptsizef pt]{fig/BA_Shapes_rare_motif20.pdf}} \\
% \bottomrule\\
% \end{tabular}%
% }
% \captionsetup{width=\columnwidth}
% \caption{The Concept Distillation Module detects concepts more fine-grained than the simple ground truth motif encoded as well as rare motifs. Blue nodes are the instances being explained, while orange nodes represent their $p$-hop neighbors. Notably, GCExplainer gives no indication of rare concepts.}
% \label{fig:rare_concepts}
% \end{table}


% \begin{figure}[!t]
%     \centering
%     \includegraphics[width=\textwidth]{fig/explanation_visualization.pdf}
%     \caption{Visualisations of an example concept per dataset, as well as the binary concept encoding generated by the Concept Distillation Module.}
%     \label{fig:concept_visuals}
% \end{figure}

% Figure \ref{fig:purity} visualises the minimum, maximum and median purity score of concepts extracted using our model versus those extracted from the vanilla GNN using GCExplainer. In regards to the minimum purity score, both our method and GCExplainer obtain a perfect score of 0 for the BA-Shapes and Tree-Grid dataset. Reviewing the maximum concept purity score across datasets, it can be stated that our method outperforms GCExplainer on the BA-Shapes, BA-Grid and Tree-Cycle dataset, however, performs worse on the remaining datasets. Regarding the median, it can be stated that our method and GCExplainer are of similar quality with each method extracting more pure concepts on different datasets.

\paragraph{The Concept Distillation Module identifies rare and fine-grained concepts (Table~\ref{fig:rare_concepts})}
CDM discovers more fine-grained concepts than just the ``house motif'' suggested by GNNExplainer, as it can differentiate whether a middle or bottom node is on the far or near side of the edge attaching to the BA graph. This matches the quality of concepts extracted by GCExplainer. In contrast to GCExplainer, CDM also identifies rare concepts. Rare motifs are present in toy datasets through the insertion of random edges. As the proposed approach can find the optimal number of clusters/concepts dynamically, clusters of a very small size possibly represent rare motifs. To check the presence of rare concepts, we visualize the $p$-hop neighbors of nodes found in small clusters. For example, CDM identifies a rare concept represented as a ``house'' structure attached to the BA graph via the top node of the house in the BA-Shapes dataset. This represents a rare concept as it is generated by the insertion of a random edge. We confirm this observations in other toy datasets, such as BA-Community and Tree-Cycle, where motifs with random edges are clearly identified. We have not identified rare concepts in BA-Grid or Tree-Grid, which may be attributed to the random edges being distributed within the base graph, which has a less definite structure. Due to the lack of expert knowledge, we cannot confirm whether the rare motifs found in Mutagenicity and Reddit-Binary align with human expectations.



% \begin{figure}[!ht]
%     \centering
%     \includegraphics[width=\textwidth]{fig/outlier_visualization.pdf}
%     \caption{Concepts corresponding to rare motifs detected by the proposed Concept Graph Module.}
%     \label{fig:outliers}
% \end{figure}




%\subsection{Explanation performance}
\paragraph{The Concept Distillation Module identifies pure concepts (Table~\ref{fig:accuracy})}
CDM discovers high-quality concepts, which are coherent across samples, as measured by concept purity. In terms of purity, our approach discovers concepts with nearly optimal scores in toy datasets, with a graph edit distance close to zero. For these datasets, CDM provides either better or comparable purity scores when compared to GCExplainer. CDM provides slightly worse purity scores in both the Mutagenicity and Reddit-Binary datasets. However, also in this case the absolute purity of CDM is almost optimal.

% \vspace{-0.5cm}
% \begin{wrapfigure}{r}{0.6\textwidth}
%   \begin{center}
%     \includegraphics[width=0.6\textwidth]{fig/purity.pdf}
%     \caption{Purity scores for the concept extracted by Concept Graph Module and GCExplainer. Notice how the optimal purity score is zero, as it measures the graph edit distance between concept instances~\citep{magister2021gcexplainer}.}
%     \label{fig:purity}
%     \end{center}
% \end{wrapfigure}

% \begin{minipage}{0.4\textwidth}
% \paragraph{The Concept Distillation Module identifies pure concepts (Figure~\ref{fig:purity})}
% CDM discovers high-quality concepts, which are coherent across samples, as measured by concept purity. In terms of purity, our approach discovers concepts with nearly optimal scores in toy datasets, with a graph edit distance close to zero. For these datasets, CDM provides either better or comparable purity scores when compared to GCExplainer. CDM provides slightly worse purity scores in both the Mutagenicity and Reddit-Binary datasets. However, also in this case the absolute purity of CDM is almost optimal.
% \end{minipage}
% \hspace{0.02\textwidth}
% \begin{minipage}{0.55\textwidth}
% % \begin{figure}[!h]
%     \includegraphics[width=\textwidth]{fig/purity.pdf}
%     \captionof{figure}{Purity scores for the concept extracted by Concept Graph Module and GCExplainer. Notice how the optimal purity score is zero, as it measures the graph edit distance between concept instances~\citep{magister2021gcexplainer}.}
%     \label{fig:purity}
% % \end{figure}
% \end{minipage}

% \paragraph{The Concept Distillation Module identifies pure concepts (Figure~\ref{fig:purity})}
% CDM discovers high-quality concepts, which are coherent across samples, as measured by concept purity. In terms of purity, our approach discovers concepts with nearly optimal scores in toy datasets, with a graph edit distance close to zero. For these datasets, CDM provides either better or comparable purity scores when compared to GCExplainer. CDM provides slightly worse purity scores in both the Mutagenicity and Reddit-Binary datasets. However, also in this case the absolute purity of CDM is almost optimal.
% \end{minipage}
% \hspace{0.02\textwidth}
% \begin{minipage}{0.55\textwidth}
% % \begin{figure}[!h]
%     \includegraphics[width=\textwidth]{fig/purity.pdf}
%     \captionof{figure}{Purity scores for the concept extracted by Concept Graph Module and GCExplainer. Notice how the optimal purity score is zero, as it measures the graph edit distance between concept instances~\citep{magister2021gcexplainer}.}
%     \label{fig:purity}
% % \end{figure}

% \begin{figure}[!ht]
%     \centering
%     \includegraphics[width=0.5\textwidth]{fig/purity.pdf}
%     \caption{The minimum purity across concepts discovered by the proposed Concept Graph Module (CGM) and GCExplainer for an equivalent vanilla GNN.}
%     \label{fig:purity}
% \end{figure}

\paragraph{The Concept Distillation Module provides accurate logic explanations (Table \ref{fig:lens}, Table~\ref{tab:logic_explanations})}
LEN allows CDM to provide simple and accurate logic explanations for task predictions. The accuracy of the logic explanations extracted reaches at least $90\%$ for the BA-Shapes, BA-Grid and Tree-Cycle datasets, indicating that CDM derives a precise description of the model decision process. Relating the accuracy of explanations back to the model accuracy, we observe that the explanation accuracy is bounded by task performance, as already noticed by~\citet{ciravegna2021logic}. This explains the slightly lower logic explanation accuracy on the real-world datasets, which can be ascribed to the absence of definite ground-truth concepts and to the classification task being more complex. Besides being accurate, logic explanations are very short, with a complexity below $4$ terms. In conjunction with the explanation accuracy, this means that CDM finds a small set of predicates which accurately describes the most relevant concepts for each class. % We refer the reader to appendix \ref{appendix:homogenity} for a further evaluation on concept homogeneity.

% \begin{figure*}[!h]
%     \centering
    
%     % \includegraphics[width=0.8\textwidth]{fig/lens.pdf}
%     \caption{Accuracy and complexity of logic explanations provided by the proposed Concept Graph Module. The accuracy is computed using logic formulas to classify samples based on their concept encoding. Explanation complexity measures the number of minterms in logic formulas.}
%     \label{fig:lens}
% \end{figure*}

\begin{table}[]
\centering
\resizebox{\columnwidth}{!}{%
\begin{tabular}{lll}
\toprule
     & \textbf{\begin{tabular}[c]{@{}l@{}}Logic Explanation\\ Accuracy (\%)\end{tabular}} & \textbf{\begin{tabular}[c]{@{}l@{}}Logic Explanation\\ Complexity\end{tabular}} \\ \midrule
\textbf{BA-Shapes}   & 96.56 (92.17, 100.95)                                                         & 3.10 (2.75, 3.45)                                                               \\
\textbf{BA-Community}    & 81.43 (78.20, 84.66)                                                          & 3.85 (3.09, 4.61)                                                               \\
\textbf{BA-Grid}     & 99.61 (98.86, 100.36)                                                         & 1.30 (0.74, 1.86)                                                               \\
\textbf{Tree-Cycle}  & 90.49 (78.43, 102.55)                                                         & 1.90 (1.22, 2.58)                                                               \\
\textbf{Tree-Grid}   & 89.66 (82.71, 96.62)                                                          & 2.20 (1.07, 3.33)                                                               \\
\textbf{Mutagenicity}      & 59.94 (44.99, 74.90)                                                          & 2.60 (0.88, 4.32)                                                               \\
\textbf{Reddit-Binary} & 71.84 (54.10, 89.59)                                                          & 1.60 (1.08, 2.12)                                                               \\ \bottomrule
\end{tabular}%
}
    \caption{Accuracy and complexity of logic explanations provided by the proposed Concept Distillation Module. The accuracy is computed using logic formulas to classify samples based on their concept encoding. Explanation complexity measures the number of minterms in logic formulas.}
    \label{fig:lens}
\end{table}

% \begin{figure}[!ht]
%     \centering
%     \includegraphics[width=\textwidth]{fig/formulas.pdf}
%     \caption{An example of a concept-based logic explanations discovered by the Concept Distillation Module for each dataset.}
%     \label{fig:logic_explanations}
% \end{figure}

% \scalerel*{\includegraphics{fig/logic_expl_graphs/BA_Shapes_concept_3.pdf}}{B}

% \resizebox{\textwidth}{!}{%
% \input{logic_expl}
% }

% \newcommand\conceptsize{20}
% \newcommand\vfigs{-.3\height}
% % Please add the following required packages to your document preamble:
% % \usepackage{graphicx}
% \begin{table*}[!h]
% \centering
% \renewcommand{\arraystretch}{1}
% \resizebox{0.8\textwidth}{!}{%
% \begin{tabular}{llll}
% \toprule
% \textbf{Dataset} & \textbf{Concept-based Logic Explanation} & \multicolumn{2}{l}{\textbf{Ground Turth Concepts}} \\
% \midrule
% \textbf{BA-Shapes} & $y =2\rightarrow$ \raisebox{\vfigs}{\includegraphics[height=\conceptsize pt]{fig/logic_expl_graphs/BA_Shapes_concept_3.pdf}} OR \raisebox{\vfigs}{\includegraphics[height=\conceptsize pt]{fig/logic_expl_graphs/BA_Shapes_concept_5.pdf}} & \raisebox{\vfigs}{\includegraphics[height=\conceptsize pt]{fig/logic_expl_graphs/house.pdf}}  & \textit{Node in house motif} \\
% \textbf{BA-Grid} & $y=1\rightarrow$ \raisebox{\vfigs}{\includegraphics[height=\conceptsize pt]{fig/logic_expl_graphs/BA_Grid_concept_2.pdf}} & \raisebox{\vfigs}{\includegraphics[height=\conceptsize pt]{fig/logic_expl_graphs/grid.pdf}}  & \textit{Node in grid motif}  \\
% \textbf{Tree-Grid} & $y=1\rightarrow$ \raisebox{\vfigs}{\includegraphics[height=\conceptsize pt]{fig/logic_expl_graphs/Tree_Grid_concept_21.pdf}} OR \raisebox{\vfigs}{\includegraphics[height=\conceptsize pt]{fig/logic_expl_graphs/Tree_Grid_concept_32.pdf}} & \raisebox{\vfigs}{\includegraphics[height=\conceptsize pt]{fig/logic_expl_graphs/grid.pdf}}  & \textit{Node in grid motif} \\
% \textbf{Tree-Cycle} & $y=1\rightarrow$ \raisebox{\vfigs}{\includegraphics[height=\conceptsize pt]{fig/logic_expl_graphs/Tree_Cycle_concept_0.pdf}} OR \raisebox{\vfigs}{\includegraphics[height=\conceptsize pt]{fig/logic_expl_graphs/Tree_Cycle_concept_8.pdf}} OR \raisebox{\vfigs}{\includegraphics[height=\conceptsize pt]{fig/logic_expl_graphs/Tree_Cycle_concept_11.pdf}} & \raisebox{\vfigs}{\includegraphics[height=\conceptsize pt]{fig/logic_expl_graphs/ring.pdf}}  & \textit{Node in circle motif} \\
% \textbf{BA-Community} & $y=3\rightarrow$ \raisebox{\vfigs}{\includegraphics[height=\conceptsize pt]{fig/logic_expl_graphs/BA_Community_concept_29.pdf}} OR \raisebox{\vfigs}{\includegraphics[height=\conceptsize pt]{fig/logic_expl_graphs/BA_Community_concept_30.pdf}} OR \raisebox{\vfigs}{\includegraphics[height=\conceptsize pt]{fig/logic_expl_graphs/BA_Community_concept_33.pdf}} & \raisebox{\vfigs}{\includegraphics[height=\conceptsize pt]{fig/logic_expl_graphs/house.pdf}}  & \textit{Node in house motif} \\
% \textbf{Reddit-Binary} & $y =\text{``Q/A''}\rightarrow$  \raisebox{\vfigs}{\includegraphics[height=\conceptsize  pt]{fig/logic_expl_graphs/Reddit_Binary_concept_24.pdf}} OR \raisebox{\vfigs}{\includegraphics[height=\conceptsize  pt]{fig/logic_expl_graphs/Reddit_Binary_concept_27.pdf}} & \raisebox{\vfigs}{\includegraphics[height=\conceptsize pt]{fig/logic_expl_graphs/star.pdf}}  & \textit{Star motifs} \\
% \textbf{Mutagenicity} & $y =\text{``mutagenic''}\rightarrow$ \raisebox{\vfigs}{\includegraphics[height=\conceptsize pt]{fig/logic_expl_graphs/Mutagenicity1.pdf}} & \raisebox{\vfigs}{\includegraphics[height=\conceptsize pt]{fig/logic_expl_graphs/ring.pdf}} \raisebox{\vfigs}{\includegraphics[height=28 pt]{fig/logic_expl_graphs/no2.pdf}}  & \textit{Ring motifs or NO$_{\text{2}}$} \\
% \bottomrule\\
% \end{tabular}%
% }
% \caption{An example of a concept-based logic explanations discovered by the Concept Distillation Module per dataset. Blue nodes are the instances being explained, while orange nodes represent their $p$-hop neighbors. For Mutagenicity the color of each node represents a different chemical element. The logic formulae describe how the presence of concepts can be used to infer task labels. For example, the first logic rule states that the task label ``middle nodes in house motifs'' ($y=2$) can be inferred from the concepts: ``middle node with attaching edge on the near side'' or ``middle node with attaching edge on the far side''.}
% \label{tab:logic_explanations}
% \end{table*}

%\subsection{Interventions}
\paragraph{The Concept Distillation Module supports human interventions (Figure~\ref{fig:interventions})}
Supporting human interventions is one of the main benefits of more interpretable architectures that learn tasks as a function of concepts. In contrast to vanilla GNNs, CDM enables interventions at concept-level, which allows human experts to correct mispredicted concepts. Similarly to Concept Bottleneck Models~\citep{koh2020concept}, our results show that correcting concept assignments significantly improves the model test accuracy to over $98\%$ for the synthetic datasets, achieving $100\%$ test accuracy on BA-Grid and BA-Shapes. We also observe an increase in task accuracy in BA-Community, however, the increase is much more gradual. Most notably, in both real-world datasets CDMs allow GNNs to improve their task accuracy by up to $\sim + 10\%$ with less than $10$ interventions.

% % \vspace{-0.5cm}
% \begin{figure*}[!t]
%     \centering
%     \includegraphics[width=0.45\textwidth]{fig/expl_intervention.pdf}
%     \includegraphics[width=0.25\textwidth]{fig/interventions.pdf}
%     \caption{The Concept Distillation Module supports interventions at concept-level, allowing human experts to correct mispredicted concepts (left), increasing human trust in the model~\citep{shen2022trust}. This interaction  significantly improves task performance, achieving almost $100\%$ accuracy on synthetic datasets (right).}
%     \label{fig:interventions}
% \end{figure*}



\section{Removing Shortcuts to Concepts}


Summary of contributions: a self-explaining GNN which explains its own predictions based on concepts learnt automatically in an unsupervised way.

Summary of results: the model accuracy of the self-explaining GNN is comparable to a standard GNN. The concept purity is as high as in post-hoc methods such as GCExplainer. Motifs/subgraphs are coherent within each cluster. The learnt set of concepts is complete w.r.t. the task i.e., it is possible to accurately predict the task given the information in learnt concepts. The logic explanation accuracy is just a bit lower than model accuracy (as expected, because that's a boolean mapping). The complexity is really low ~3-4 terms which makes the explanations simple and quickly interpretable.

Impact/significance: this work alleviates the problem of generating expensive concept annotations which might even be unfeasible in some fields where humans have not yet accumulated enough knowledge to allow a robust a large-scale concept annotations. So, this work has an impact on:

- the concept-based and the XAI field allowing the expansion of these approaches in tackling problems where concept annotations are too scarce/expensive or not existing.

- other research disciplines especially those where learning new concepts can help humans accumulate knowledge and build an ontology

\section{Architectural biases}
Next steps: scale to non-GNN models which might be tricky because in GNNs clustering works and has a strong association to motifs, while other architectures do not have such a strong bias.

\section*{Papers}
\nobibliography*
\begin{itemize}
    \item Steve Azzolin, Pietro Barbiero, ..., and Pietro Lio' Global GNN Interpretability via Logic Explanations. \textit{arXiv preprint arXiv:XXXX.YYYYY}, 2023
    \item Han Xuanyuan, Pietro Barbiero, ..., and Pietro Lio' Analysing the Neurons of Graph Neural Networks: Towards Concept-Based Global Interpretability. \textit{arXiv preprint arXiv:XXXX.YYYYY}, 2023
    \item \bibentry{magister2022encoding}
\end{itemize}



%%%%%%%%%%%%%%%%%%%%%%%%%%%%%%%%%%%%%%%%%%%%%%%%%%%%%%%%%%%%%%%%%%%%%%%%%%%%%%%%
%% Applications to Medical Digital Twins:
%%
\chapter{Applications} \label{chapter:applications}
\textbf{Research: in progress. Status: drafted. Difficulty: medium. Priority: medium.}

\textit{In this chapter I will showcase how my inventions significantly improved medical digital twin models~\citep{laubenbacher2021using}. First, I will descibe existing digital twin approaches in medicine and the main challenges of the field. Next, I will demonstrate how concept-based neural models can significantly improve the flexibility and robustness of existing equation-based approaches. In particular I will show how concept-based models allow the discovery of multi-omic patterns explaining drug responses in asthma and down syndrome.}

\section{Concept Learning for Biomedical Data}

\section{mRNA Expression Profiles in Asthma}

\section{Mouse Models of Down Syndrome}


\section*{Papers}
\nobibliography*
\begin{itemize}
    \item Pietro Barbiero, and Pietro Lio'. Logic-based Deep Learning Clinical models: Down Syndrome case study. \textit{arXiv preprint arXiv:XXXX.YYYYY}, 2023
    \item \bibentry{kidwai2023forecasts}
    \item \bibentry{barbiero2021graph}
\end{itemize}


%%%%%%%%%%%%%%%%%%%%%%%%%%%%%%%%%%%%%%%%%%%%%%%%%%%%%%%%%%%%%%%%%%%%%%%%%%%%%%%%
%% Index:
%%
\printthesisindex

\end{document}
