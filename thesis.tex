%%%%%
%%
%% Sample document ``thesis.tex''
%%
%% Version: v0.2
%% Authors: Jean Martina, Rok Strnisa, Matej Urbas
%% Date: 30/07/2008
%%
%% Copyright (c) 2008-2011, Rok Strniša, Jean Martina, Matej Urbas
%% License: Simplified BSD License
%% License file: ./License
%% Original License URL: http://www.freebsd.org/copyright/freebsd-license.html
%%%%%

% Available documentclass options:
%
%   <all `report` document class options, e.g.: `a5paper`>
%   withindex   - enables the index. New index entries can be added through `\index{my entry}`
%   glossary    - enables the glossary.
%   techreport  - typesets the thesis in the technical report format.
%   firstyr     - formats the document as a first-year report.
%   times       - uses the `Times` font.
%   backrefs    - add back references in the Bibliography section
%
% For more info see `README.md`
% \documentclass[withindex,glossary]{cam-thesis}
\documentclass[withindex,glossary,secondyr]{cam-thesis}

% Citations using numbers
\usepackage[numbers]{natbib}

%%%%%%%%%%%%%%%%%%%%%%%%%%%%%%%%%%%%%%%%%%%%%%%%%%%%%%%%%%%%%%%%%%%%%%%%%%%%%%%%
%% Thesis meta-information
%%

%% The title of the thesis:
\title{The Trustworthy Language\\of Self-Explainable Deep Learning}

%% The full name of the author (e.g.: James Smith):
\author{Pietro Barbiero}

%% College affiliation:
\college{Clare College}

%% College shield [optional]:
% \collegeshield{CollegeShields/Christs}
% \collegeshield{CollegeShields/Churchill}
\collegeshield{CollegeShields/Clare}
% \collegeshield{CollegeShields/ClareHall}
% \collegeshield{CollegeShields/CorpusChristi}
% \collegeshield{CollegeShields/Darwin}
% \collegeshield{CollegeShields/Downing}
% \collegeshield{CollegeShields/Emmanuel}
% \collegeshield{CollegeShields/Fitzwilliam}
% \collegeshield{CollegeShields/Girton}
% \collegeshield{CollegeShields/GonCaius}
% \collegeshield{CollegeShields/Homerton}
% \collegeshield{CollegeShields/HughesHall}
% \collegeshield{CollegeShields/Jesus}
% \collegeshield{CollegeShields/Kings}
% \collegeshield{CollegeShields/LucyCavendish}
% \collegeshield{CollegeShields/Magdalene}
% \collegeshield{CollegeShields/MurrayEdwards}
% \collegeshield{CollegeShields/Newnham}
% \collegeshield{CollegeShields/Pembroke}
% \collegeshield{CollegeShields/Peterhouse}
% \collegeshield{CollegeShields/Queens}
% \collegeshield{CollegeShields/Robinson}
% \collegeshield{CollegeShields/Selwyn}
% \collegeshield{CollegeShields/SidneySussex}
% \collegeshield{CollegeShields/StCatharines}
% \collegeshield{CollegeShields/StEdmunds}
% \collegeshield{CollegeShields/StJohns}
% \collegeshield{CollegeShields/Trinity}
% \collegeshield{CollegeShields/TrinityHall}
% \collegeshield{CollegeShields/Wolfson}
% \collegeshield{CollegeShields/CUniNoText}
% \collegeshield{CollegeShields/FitzwilliamRed}

%% Submission date [optional]:
% \submissiondate{November, 2042}

%% You can redefine the submission notice [optional]:
% \submissionnotice{A badass thesis submitted on time for the Degree of PhD}

%% Declaration date:
\date{My Month, My Year}

%% PDF meta-info:
\subjectline{Computer Science}
\keywords{one two three}



%%%%%%%%%%%%%%%%%%%%%%%%%%%%%%%%%%%%%%%%%%%%%%%%%%%%%%%%%%%%%%%%%%%%%%%%%%%%%%%%
%% Abstract:
%%
\abstract{%
  Bla
}



%%%%%%%%%%%%%%%%%%%%%%%%%%%%%%%%%%%%%%%%%%%%%%%%%%%%%%%%%%%%%%%%%%%%%%%%%%%%%%%%
%% Acknowledgements:
%%
\acknowledgements{%
  My acknowledgements ...
}



%%%%%%%%%%%%%%%%%%%%%%%%%%%%%%%%%%%%%%%%%%%%%%%%%%%%%%%%%%%%%%%%%%%%%%%%%%%%%%%%
%% Glossary [optional]:
%%
\newglossaryentry{HOL}{
    name=HOL,
    description={Higher-order logic}
}



%%%%%%%%%%%%%%%%%%%%%%%%%%%%%%%%%%%%%%%%%%%%%%%%%%%%%%%%%%%%%%%%%%%%%%%%%%%%%%%%
%% Contents:
%%
\begin{document}



%%%%%%%%%%%%%%%%%%%%%%%%%%%%%%%%%%%%%%%%%%%%%%%%%%%%%%%%%%%%%%%%%%%%%%%%%%%%%%%%
%% Title page, abstract, declaration etc.:
%% -    the title page (is automatically omitted in the technical report mode).
\frontmatter{}



%%%%%%%%%%%%%%%%%%%%%%%%%%%%%%%%%%%%%%%%%%%%%%%%%%%%%%%%%%%%%%%%%%%%%%%%%%%%%%%%
%% Thesis body:
%%
\chapter{Progress}
\textit{A report on progress made in relation to that described in the first-year PhD Proposal. This should include an indication of where the student is relative to their original timetable, discussion of any significant changes to the original ideas and their implications for the research as a whole.}

\chapter{Thesis Outline}
\textit{In this chapter I provide a tentative chapter-by-chapter outline of the thesis. I include a tentative title for each of the main chapters with a brief paragraph summary of their content. Chapter summaries also acknowledge the work I have already completed and the work I still need to do during the third year (for example, ``chapter written'', ``chapter drafted'', ``research complete but not written'', ``research in progress'', ``research not started'').}

\section{Introduction}

\section{Related Work}
\begin{itemize}
    \item Neural networks: MLPs, CNNs and GNNs
    \item explainability~\citep{duran2021afraid,lo2020ethical,wachter2017counterfactual,gdpr2017,rudin2019stop}
    \item concepts-based XAI~\citep{ghorbani2019towards,kim2018interpretability,shen2022trust}
    \item limitations: weak metrics, no clear how concepts should be combined, accuracy-vs-interpretability trade-off, concept annotations are expensive
\end{itemize}

\section{Concept Quality Metrics}
\begin{itemize}
    \item Existing metrics~\citep{yeh2020completeness}
    \item Oracle purity~\citep{zarlenga2021quality}
    \item Niching~\citep{zarlenga2021quality}
\end{itemize}


\section{Concept Embedding Models}
\begin{itemize}
    \item Concept bottleneck models~\citep{koh2020concept}
    \item CEMs~\citep{zarlenga2022concept} break the accuracy-vs-interpretability trade-off
\end{itemize}


\section{Logic Explained Networks}
\begin{itemize}
    \item LENs framework~\citep{ciravegna2021logic}
    \item Entropy-based Logic Explanations of Neural Networks~\citep{barbiero2021entropy} provide excellent accuracies and explanations
\end{itemize}


\section{Explainability by Design}
\begin{itemize}
    \item Encoding Concepts in Graph Neural Networks~\citep{magister2022encoding} discovers concepts in an unsupervised way (no need for expensive annotations!)
    \item Encoding Concepts in any network
    \item Multiple concepts
\end{itemize}


\section{The ``PyTorch, Explain!'' Library}


\section{Applications}
\begin{itemize}
    \item asthma
    \item down syndrome
    \item digital twin
    \item automated reasoning
\end{itemize}




\chapter{Timetable}
\textit{A timetable that schedules the remaining work and indicates when the draft and final versions of the thesis will be produced.}

\chapter{Papers}
\textit{A list of any papers published (with URLs so that the assessors can read the papers), a list of any papers in press, submitted, or in preparation, and a list of any presentations given, whether or not the presentation is associated with a paper.}




%%%%%%%%%%%%%%%%%%%%%%%%%%%%%%%%%%%%%%%%%%%%%%%%%%%%%%%%%%%%%%%%%%%%%%%%%%%%%%%%
%% References:
%%
% If you include some work not referenced in the main text (e.g. using \nocite{}), consider changing "References" to "Bibliography".
%

% \renewcommand to change default "Bibliography" to "References"
\renewcommand{\bibname}{References}
\cleardoublepage
\phantomsection
\addcontentsline{toc}{chapter}{References}
\bibliographystyle{plainnat}
\bibliography{thesis}



%%%%%%%%%%%%%%%%%%%%%%%%%%%%%%%%%%%%%%%%%%%%%%%%%%%%%%%%%%%%%%%%%%%%%%%%%%%%%%%%
%% Appendix:
%%

\appendix

\chapter{Extra Information}
Some more text ...



%%%%%%%%%%%%%%%%%%%%%%%%%%%%%%%%%%%%%%%%%%%%%%%%%%%%%%%%%%%%%%%%%%%%%%%%%%%%%%%%
%% Index:
%%
\printthesisindex

\end{document}
